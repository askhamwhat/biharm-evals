\documentclass{article} 

\usepackage{amsmath,amsfonts,amssymb}
\usepackage{cleveref}

\usepackage{color}
%%%%%%%%%%%%%%%%%%%%%%%%%%%%%%%%%%%%%%%%%%%%%%%%%%%%%%%%%%%%

\Crefname{equation}{}{}
\crefname{equation}{}{}

%%%%%%%%%%%%%%%%%%%%%%%%%%%%%%%%%%%%%%%%%%%%%%%%%%%%%%%%%%%%

\DeclareMathOperator*{\argmin}{arg\,min}
\DeclareMathOperator{\dn}{dn}
\DeclareMathOperator{\cn}{cn}
\DeclareMathOperator{\sn}{sn}

\newcommand*\C{\mathbb{C}}
\newcommand*\R{\mathbb{R}}
\def\bx{{\mathbf x}}
\def\bX{{\mathbf X}}
\def\bf{{\mathbf f}}
\def\bg{{\mathbf g}}
\def\bc{{\mathbf c}}
\def\by{{\mathbf y}}
\def\bxi{{\boldsymbol \xi}}
\def\bPhi{{\boldsymbol \Phi}}
\def\bSigma{{\boldsymbol \Sigma}}
\def\btheta{{\boldsymbol \theta}}
\def\bTheta{{\boldsymbol \Theta}}
\def\bC{{\mathbf C}}
\def\balpha{{\boldsymbol \alpha}}
\def\bB{{\mathbf B}}
\def\bigo{\mathcal{O}}

\newtheorem{remark}{Remark}
\newtheorem{lemma}{Lemma}
\newtheorem{theorem}{Theorem}

\title{Response to referees}
\author{Travis Askham, Manas Rachh}

%%%%%%%%%%%%%%%%%%%%%%%%%%%%%%%%%%%%%%%%%%%%%%%%%%%%%%%%%%%%%

\begin{document}

\maketitle

In this document, we respond to the referees' comments
for the paper ``A boundary integral equation approach to computing
eigenvalues of the Stokes operator''.
We believe that the referees' suggestions improve the
quality of the paper and that we have been able to
address their concerns. Changes made to address these concerns
are marked in {\color{red} red} in the revised manuscript.

We have also marked in {\color{red} red} a couple of
changes around the explanation of the order of convergence
of the generalized Gaussian quadrature rule we used.
The rule is designed to be 16th order, i.e. it is designed
to integrate $p(y)\log|x-y| + q(y)$ for $p$ and
$q$ polynomials of degree up to 15. Our kernel is actually
smoother than $\log|x-y|$ (with leading order
singularity $|x-y|^2\log|x-y|$) and in our experience the
convergence rate is typically higher (20th order in
the example in the paper). In the revised text, we have clarified
that the rule is not designed to be 20th order but that this is
the behavior we observed. 


\section{Reviewer 1}

A general concern of the first reviewer was the discussion
of what we originally called ``spurious double roots''.
We appreciate the reviewer pressing us on this issue, as it
has clarified the results of the paper.

To explain what had happened, we computed the roots using a rootfinding
utility in another software package, \texttt{chebfun}. In our
numerical experiments, we were puzzled that the output of the
utility consistently gave us pairs of roots which seemed to
be identical. Yet, they did not actually seem to correspond to
a double root of the Fredholm determinant in any meaningful sense
(indeed, the derivative of our interpolant was not particularly
close to zero for these points). We performed some other tests
(including checking whether there truly was a corresponding
two-dimensional eigenspace) and no other tests suggested a double
root existed. Thus, in the original submission, we made a note
suggesting that rounding errors may explain this strange behavior
and we described our ad hoc methods for determining true roots.
To confuse matters, we mentioned the existence of roots that
were computed and had large imaginary part as being part of
the same phenomenon (they are not).

It turns out that this behavior can be suppressed rather
simply. By default, the \texttt{roots}
utility in \texttt{chebfun} first divides the interval on
which the interpolant is defined into smaller intervals on which
the interpolant can be represented by a shorter Chebyshev series.
The roots on these subintervals can then be computed more efficiently.
However, the roots for neighboring intervals can be found twice
by this procedure and there is some internal processing to eliminate
double roots. This processing seems to fail in our case (perhaps because
we are using the complex roots mode, which is more experimental) and
we observe the fake double roots described above. Fortunately, the
subdivision procedure can be turned off. When this was done, such
roots disappeared (aside from double roots introduced by our
own splitting into subintervals). Once aware of this issue,
a much simpler procedure for identifying the roots which are actually
contained on a real subinterval can be employed.

We describe this simplified root finding procedure in the updated
manuscript. The eigenvalues/eigenvectors in the examples
did not change.

We respond to other comments from Reviewer 1 below. Original
reviewer comments in {\color{blue} blue}

\begin{itemize}
\item[1.] {\color{blue} In different parts of the paper the connection between the eigenvalue
problem of the Stokes operator and the biharmonic eigenvalue problem
1is addressed and utilized. Therefore for me it is strange that just in
the last section of the paper (“Conclusions”) this connection is elabo-
rated. It would be better that this elaboration is done somewhere at
the beginning of the paper and not in the end.}

We appreciate this suggestion and have moved this elaboration to
the introduction. 

\item[2.] {\color{blue}
  In the abstract the Stokes operator is denoted as fourth-order operator.
This is unusual and in my opinion not the case (even if there is an
equivalence to a forth-order operator).}

We see how this could be confusing and have adjusted this phrasing.

\item[3.] {\color{blue} I would appreciate it if the following issue would be addressed in the
paper: under which weaker conditions (not analytic boundary, non-
simple eigenvalues) convergence of the discretization of the eigenvalues
can be proven. (Are there results concerning the convergence of the
eigenfunctions?)}

The question of addressing this problem under weaker assumptions of smoothness of the boundary
is a very interesting one. To the best of our knowledge, this
question hasn't even been addressed for the eigenvalues of the Laplacian.
We have a sketch of a proof of convergence for the eigenvalues and eigenfunctions
for the Laplace operator where we can show that the eigenvalues and eigenfunctions
converge at an algebraic rate (dependent on the smoothness of curves) for $C^{\ell}$,
with $\ell\geq 2$, boundaries. The procedure of using the Fredholm determinant as
stated does not work on Lipchitz boundaries, since the corresponding operators are
not second kind. 

However, we feel that the discussion would be fairly technical and distract
from the main results of the paper --- demonstrating the use of Fredholm determinants
and second kind formulations for computing the Stokes operator eigenvalues and
eigenfunctions.  For this reason we feel that this analysis should be carried out in
a separate work and done for both the Laplace and Stokes cases.

We've included a sketch of the proof below. 

The main estimate needed is to show that the error in computing
the Fredholm determinant at an eigenvalue of the Stokes operator
is directly proportional to the quadrature error in 
evaluating the integral operator applied to the null vector (A proof of this result for the 
Laplace case is contained in~\cref{sec:appb}). 
Based on the second-kind nature of the operator from $C^{m,\alpha} \to C^{m,\alpha}$
for $m<\ell$ and $\alpha>0$ (when the boundary is $C^{\ell}$) one can show
that the resulting null vector must be at least in $C^{m,\alpha}$.
The quadrature error using Kress quadratures converges algebraically
when both the boundary and density being approximated are $C^{m,\alpha}$. 

\item[4.] {\color{blue} In the paper it is not explicitly mentioned that
  both presented boundary integral equations may have non-trivial null-spaces
  for certain $k$ with ${\rm Im}(k) < 0$. These $k$s correspond to the eigenvalues of some
  related eigenvalue problems in the exterior domain. It is possible that
  these $k$s are also be detected by the root search method and that these
  eigenvalues may be some of those “spurious eigenvalues too far from
  real-valued” which are blamed to be caused by “noisy numerical determinant
  evaluation” but are actually caused by the used boundary integral formulation.}

Thank you for pointing this out. This is indeed the case, as can be seen
by the dependence of many of the results on ${\rm Im}(k) \geq 0$. We have
added a remark about this after Theorem 10 and updated the theorems and lemmas
as suggested in other comments below. These are indeed detected by the
root finding procedure, though they are easy to identify as
being distinct from the real-valued roots which correspond to
eigenvalues (the imaginary part of the real-valued roots is several
orders of magnitued smaller than the imaginary part of the
spurious roots we observed).

\item[5.] {\color{blue}
  In general it would be of interest which kind of spurious eigenvalues may
occur from a“noisy numerical determinant evaluation”. In the paper
only two kinds are specified, namely spurious eigenvalues with “large”
imaginary part (which may result also from the boundary integral for-
mulation itself as pointed out above) and spurious “double roots” on
the real line. Isn’t it possible that for example spurious single roots
occur on the real line?}

Again, thank you for pressing this point. We believe that this
is mostly resolved by the discussion above. In the case of a spurious
single root on the real line, we imagine that this could happen, though it
was not observed, i.e. one can imagine a pathological domain for which a
true spurious root exists that is within, say, $10^{-7}$ of the real line.
Such a root would be hard to distinguish from a true root.
In all cases, we also computed the smallest singular
value of the BIE corresponding to the real part of a given root.
This singular value was found to be quite small. Because we believe
that the integral equation was well-resolved
by our discretization, this is rather convincing evidence that an
eigenvalue of the PDE has been found.

\item[6.] {\color{blue}
  For me it is not clear why with the suggested procedure spurious ``double roots''
  may be detected. What is actually the argument that this
  procedure detects spurious double roots?}

The original procedure was designed to test whether a root was
an eigenvalue in a manner independent of the interpolant we had
computed for the Fredholm determinant. Thus, we checked the several smallest
singular values to see if any eigenspaces of dimension larger
than one were associated
with the weird double roots. None were found. We thus chose what
seemed to be the better root in an ad hoc way. As noted above, this
procedure is no longer necessary when calling \texttt{roots} with
the subdivision procedure suppressed.

\item[7.] {\color{blue}In the caption of figure 5 the given interval for $k$
  does not match with the plots. Moreover, in each plot of figure
  5 two functions are given but in the caption only one function is described.}

Apologies about that. The caption for figure 5 has now been fixed. The two curves
correspond to the real and the imaginary part of the Fredholm determinant. 

\item[8.] {\color{blue}Subsection 1.1 (”Relation to other work“) may
  lead to the false impression that boundary integral equations of the
  second kind combined with the proposed root finding method is the only
  profound and reasonable boundary integral equation approach for the Stokes
  eigenvalue problem. The authors seem not be aware of works which deal with
  boundary integral equation methods based on integral equations of the
  first kind (for several types of eigenvalue problems(Laplacian, Maxwell,
  fluid-solid interaction, plasmonics, ....)) which are well analyzed and for
  which a rigorous and comprehensive convergence theory is established.
  Such kind of an approach could be also applied to the Stokes eigen-
  value problem. Moreover, in these works instead of the root search for
  the determinant the contour integral method for nonlinear eigenvalue
  problem is used for which multiple and spurious eigenvalues seem not
  to be a problematic issue. In some way these alternatives should be
  briefly mentioned in the paper.}

  This comment is well taken.
  We did not mean to give this impression but had intended this section
  to be focused on methods specifically for the Stokes eigenvalue problem.
  We have added some references to applications treated using integral
  equation methods in the literature and reworked the section
  a little.

\item[9.] {\color{blue}
  I cannot follow the ”formal“ derivation of the fundamental solution from
  equation (8) to (10). Is it possible to describe the derivation in more
  detail? (It would be also good to mention the reference [7] between
  equation (8) and (10)).}

  We've added some more steps and added the suggested
  reference in that section.

\item[10.] {\color{blue}
  Lemma 4 only seems to hold for $k$ with ${\rm Im}(k) \geq 0$,
  see the asymptotic expansion of the Hankel function.}

  Thank you for pointing out this ommission. It's been fixed
  in the revision.

\item[11.] {\color{blue} In the proof of Theorem 3 and 4 only the
  uniqueness of the solution is shown but not the existence.
  (The existence follow for example from the Fredholm theory.)}

  Thank you for pointing this out. We adjusted the
  phrasing of these.

\item[12.] {\color{blue}
  Theorem 8, 9 and 10 only seem to be valid for $k$ with
  ${\rm Im}(k) \geq 0$.}

  Again, thank you for this. This is certainly true for theorems
  8 and 10. We think that the demonstrated null vector in
  theorem 9 works for general $k$.

\end{itemize}

We also appreciate the reviewer alerting us to some typos, which have
been fixed.

\section{Reviewer 2}


Comments by the second reviewer. Original reviewer comments
in {\color{blue} blue}.

\begin{itemize}
\item[1.] {\color{blue} In the abstract and introduction the authors talk about the so-called pollution effect
with no reference to the Helmholtz problem or what high frequency regime means for
this problem. Thus unless the reader has experience with Helmholtz problems (may
not be the case for people interested in Stokes problems), they will not understand the
associated numerical challenges. This should be clarified.}

  We appreciate this suggestion and agree it may have been confusing.
  We have added a brief description of the phenomenon to the
  introduction.
  
\item[2.] {\color{blue} On page 3 line 31, what is meant by “current grid?”}

  Thank you for pointing out this vagueness. The current grid referred
  to the current boundary discretization. We have reworked this section
  in response to comments by the other reviewer and have dropped this more
  philosophical point altogether.
  
\item[3.] {\color{blue} On page 5 line 36, should $\exists \phi_0$ read $\exists \phi_0 \in X$?}

  Thank you for pointing this out. It's been fixed.
  
\item[4.] {\color{blue} On page 28 line 38, is the linear system $C_k^N$
  also discretized with generalized Gaussian quadrature?}

  Yes, this has been clarified in the revision.
  
\item[5.] {\color{blue} In Figure 6, a plot of $O(N \log N )$
  would be more helpful than $O(N )$ since that is the expected scaling.}

  Good point. This has been adjusted.


\section{Proof of convergence for general case}
Let $T_{n}$ denote the space of trignometric polynomials
$\{ e^{-inx}, \ldots e^{inx} \}$. 
Let $\mathcal{K}$ denote the Helmholtz double layer integral operator
given by
\begin{equation}
\mathcal{K}[\phi] = \frac{i}{2}\int_{\Gamma} k H_{1}^{1}(k|x-y|) \frac{(x-y)\cdot n(y)}{|x-y|} \phi(y) dS_{y} \,
\end{equation}
where $k$ the Helmholtz parameter is assumed to be a dirichlet eigenvalue of the simply connected
domain $\Omega$ whose boundary is given by $\Gamma$, and $H_{1}^{1}$ is the Hankel function
of the first kind of order $1$.
It is well known that if $\Gamma$ is of class $C^{\ell,\alpha}$, then the
integral operator $\mathcal{K}: C^{\ell-1,\alpha} \to C^{\ell-1,\alpha}$ is a compact operator.
Furthermore, as shown in Zhao and Barnett, at the simple Dirichlet eigenvalues of the operator
$I + \mathcal{K}$ is not invertible and has a one dimensional null-space. Let $\phi_{0}$ denote
the null vector. 
By Fredholm theory, the adjoint $I + \mathcal{K}^{\ast}$ also has a one-dimensional null space.
Let $\psi_{0}$ denote the null vector of $I+\mathcal{K}^{\ast}$.


Let $K_{n}: T_{n} \to T_{n}$ 
denote a discretization of $\mathcal{K}$ 
using kress quadrature. 
Let $P_{n}$ denote the orthogonal 
projection operator from $L^{2} \to T_{n}$.
Let $P_{n,\phi_{0}^{\perp}}: T_{n} \to T_{n}$ denote the orthogonal
projection on to the orthogonal complement of $P_{n} \phi_{0}$.
Let $P_{n,\psi_{0}^{\perp}}: T_{n} \to T_{n}$ denote the
analogous operator for $\psi_{0}$. 

The matrix $I+K_{n}$ can be rewritten as a map from the
following two bases for $T_{n}: \{ P_{n} \phi_{0}, \text{basis for }
P_{n}\phi_{0}^{\perp} \}$, and for the range the basis:
$\{ P_{n} \psi_{0}, \text{basis for } P_{n} \psi_{0}^{\perp} \}$.
In these bases, the linear operator $I_{n} + K_{n}$ is given by
\begin{equation}
\begin{bmatrix}
A_{n} & B_{n} \\
C_{n} & D_{n} 
\end{bmatrix} \, .
\end{equation}

The operator $A_{n}$
is the restriction of $I_{n} + K_{n}$ as a map from $P_{n} \phi_{0}$ 
to $P_{n} \psi_{0}$. Thus, it is just a number and it is given by
\begin{equation}
A_{n} = \left(P_{n} \psi_{0}, (I_{n} + K_{n}) P_{n} \phi_{0} \right) \,.
\end{equation}

The operator $D_{n}$ is the restriction of $I_{n} + K_{n}$ 
as a map from $P_{n} \phi_{0}^{\perp}$ 
to $P_{n} \psi_{0}^{\perp}$.
The following statement requires a more careful proof.
{\color{red} As $n \to \infty$, $D_{n}$ converges pointwise to the restriction
of $I + K: \phi_{0}^{\perp} \to R(I+K)$. Moreover, the collection
of operators $K_{n}$ are collectively compact, and hence their restrictions
are also collectively compact. 
Since the operator $I+K: \phi_{0}^{\perp} \to R(I+K)$ is an invertible
operator, by standard estimates, for $n$ large enough $D_{n}$ is also 
an invertible operator.}

The operator $B_{n}$ is the restriction of $I_{n} + K_{n}$ as a map from 
$P_{n} \phi_{0} ^{\perp} \to P_{n} \psi_{0}$ and is given by
\begin{equation}
B_{n} \phi = \left( P_{n} \psi_{0}, (I_{n} + K_{n}) 
(P_{n} \phi -( P_{n} \phi, P_{n} \phi_{0}) P_{n} \phi_{0}) \right) P_{n} \psi_{0}  \, .
\end{equation}

And finally the operator $C_{n}$ is the restriction of $I_{n} + K_{n}$ as a map from
$P_{n} \phi_{0} \to P_{n} \psi_{0}^{\perp}$ and is its adjoint is given by
\begin{equation}
C_{n}^{\ast}\psi = \left( (I_{n} + K_{n}^{\ast}) \left(P_{n} \psi - (P_{n} \psi, P_{n} \psi_{0}) P_{n} \psi_{0} \right), P_{n} \phi_{0} \right)
\end{equation}

In this setting, $\phi_{0}$ and $\psi_{0}$ are known to be linearly independent.
Thus for large enough $n$, 
$f_{n} = P_{n} \phi_{0} - (P_{n} \phi_{0}, P_{n} \psi_{0}) P_{n} \psi_{0} \neq 0$
and $f_{n} \in P_{n} \psi_{0}^{\perp}$.
Similarly $g_{n} = P_{n} \psi_{0} - (P_{n}\phi_{0}, P_{n} \psi_{0}) P_{n} \phi_{0} \neq 0$
and $g_{n} \in P_{n} \phi_{0}^{\perp}$.

By block linear algebra, the determinant of the matrix above is also given by
\begin{equation}
\det{
\begin{bmatrix}
A_{n} & B_{n} \\
C_{n} & D_{n} 
\end{bmatrix} 
}
= \det{D_{n}} \cdot (A_{n} - C_{n} D_{n}^{-1} B_{n}) \,.
\end{equation}
Note that $C_{n} D_{n}^{-1} B_{n}$ is a number and can be evaluated as 
\begin{equation}
\begin{aligned}
C_{n} D_{n}^{-1} B_{n} &= 
\left( f_{n} \cdot  C_{n} D_{n}^{-1} B_{n} g_{n} \right) \\
&= \left( C_{n}^{\ast} f_{n}, D_{n}^{-1} B_{n} g_{n} \right) \\
&= \left( \left((I_{n} + K_{n}^{\ast})f_{n}, P_{n} \phi_{0}\right) P_{n} \phi_{0}, D_{n}^{-1} B_{n} g_{n} \right) \\
&= \left(f_{n}, (I_{n} + K_{n})P_{n} \phi_{0} \right) \cdot (P_{n} \phi_{0}, D_{n}^{-1} B_{n} g_{n}) \\
&= \left(f_{n}, (I_{n} + K_{n})P_{n} \phi_{0} \right) \cdot (P_{n} \phi_{0}, D_{n}^{-1} P_{n} \psi_{0})  \cdot 
\left((I_{n} + K_{n}^{\ast})P_{n} \psi_{0}, g_{n}  \right) \\
\end{aligned}
\end{equation}

First, we observe that $D_{n}$ is the restriction of an operator in trace class
with a kernel whose leading order singularity is $r^{2} \log{r}$. 
Thus, $D_{n}$ is itself in trace class, and extending the proof of Bornemann to this case,
we observe that the discrete determinant of $D_{n}$ converges to the Fredholm determinant
of $D$ at a rate of at least $O(1/n)$. This in particular implies that
$\sup_{n} |\det{D_{n}}| <\infty$.
Using Cauchy-Schwarz, the quantities $A_{n}, C_{n}D_{n}^{-1}B_{n}$ are bounded by
$|(I_{n} + K_{n})\phi_{n}|$.
Combining these results together, we get that
\begin{equation}
\left| \det{\begin{bmatrix} A_{n} & B_{n} \\ C_{n} & D_{n} \end{bmatrix}} \right| \leq C |(I_{n} + K_{n})\phi_{n}|
\end{equation}

\end{itemize}

\end{document}
