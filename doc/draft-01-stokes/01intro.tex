\section{Introduction}

The planar incompressible Stokes equations describe
creeping flows in two dimensions.
%
Let $\Omega \subset \R^2$
be a bounded domain with $C^2$ boundary.
%
The Stokes eigenvalue problem is to find
values $k^2$ such that 

\begin{equation}
\begin{aligned}
  -\Delta \uu + \nabla p &= k^2 \uu \quad \textrm{in} \quad
  \Omega \label{eq:ostokes} \; , \\
  \nabla \cdot \uu &= 0 \; ,
\end{aligned}
\end{equation}
subject to boundary conditions, has a non-trivial solution $(\uu,p)$.
%
In this work, we consider the eigenvalue problem subject to 
the Dirichlet boundary condition,
\begin{equation}
  \uu = \bz \quad \textrm{on} \quad \Gamma \label{eq:ostokes_dir} \; .
\end{equation}
It is well known that the values $k^2$ are necessarily
real and positive and that there is a countable collection of such
values $0 < k_{1}^{2} \leq k_{2}^2 \leq \ldots \uparrow \infty$,
counting multiplicities.

\begin{remark}
  When $k = i\alpha$, the differential equation
  \cref{eq:ostokes} is known as the modified Stokes
  equation. As there appears to be no preferred
  name for the equation with real-valued $k$,
  we will refer to \cref{eq:ostokes} as the
  oscillatory Stokes equation.
\end{remark}

The eigenvalues (and eigenfunctions)
of the Stokes operator have applications in the
stability analysis of stationary solutions of the
Navier--Stokes equations \cite{osborn1976approximation},
in the study of decaying two dimensional turbulence
\cite{schneider2008final}, and as a trial basis for
numerical simulations of the Navier--Stokes
equations~\cite{batcho1994generalized}.
%
The eigenvalues and eigenfunctions of the Stokes
operator are also the subject of intense analytical
investigation
\cite{taylor1933buckling,szego1950membranes,
  polya1951isoperimetric,bramble1963pointwise,
  ashbaugh1996fundamental,leriche2004stokes,
  kelliher2009eigenvalues,antunes2011buckling},
especially as they relate to the eigenvalues and
eigenfunctions of the Laplacian.

Historically, the Stokes eigenvalue problem serves as a
common model problem for numerical eigenvalue analysis
with a fourth order operator (here, the bi-Laplacian).
%
Further, numerical simulation has long played an
important role in the analyses cited above --- both for
computing the eigenvalues and eigenfunctions
on domains of practical interest and in forming
new conjectures.

Borrowing the language of~\cite{zhao2015robust},
which concerns the eigenvalues of the Laplacian
(also known as the membrane or ``drum'' problem),
the numerical treatment of the Stokes eigenvalue
problem can be divided into two basic approaches.
%
The first class of methods
directly discretize the
Stokes operator, typically with a
finite element basis, and the eigenvalues are found
as the eigenvalues of the discrete system.
%
The second class of methods reformulate the 
oscillatory Stokes equations as a boundary integral
equation (BIE) which is discretized.
The eigenvalues are then found
by a nonlinear search for the values of
$k$ where the BIE is not invertible.
%

There is a large body of research on the first class of methods
for the Stokes eigenvalue problem.
%
We do not seek to review this literature here,
but point to \cite{johnson1974beam,
  rannacher1979nonconforming,
  mercier1981eigenvalue,bjorstad1999high,
  jia2009approximation,chen2006approximation,
  lovadina2009posteriori,huang2011numerical,
  carstensen2014guaranteed}
for some representative examples.

As noted in~\cite{zhao2015robust}, integral
equation based methods provide several advantages.
%
Because the BIE is defined on the
boundary alone, there is a reduction in the
dimension of the domain to be discretized.
%
This reduces the number of unknowns over finite
element discretizations, particularly given
that finite element methods suffer from
high-frequency ``pollution'' when $k$ is large
\cite{babuska1997pollution}.

Further, Zhao and Barnett~\cite{zhao2015robust}
show how to alleviate some of the costliness of the
nonlinear optimization introduced by formulating the 
problem as an integral equation.
%
The standard approach searches for ``V''-shaped minima
of the singular values of the BIE; see, for
instance, \cite{trefethen2006computed}.
%
Instead, Zhao and Barnett utilize the Fredholm
determinant (see \cref{sec:dets}) which, for certain
BIEs, is an analytic function of $k$ with roots
precisely when $k^2$ is an eigenvalue.
%
The Fredholm determinant can be estimated using
a Nystr\"{o}m discretization of the BIE
\cite{bornemann2010numerical,zhao2015robust}.
%
Then, the eigenvalues can be estimated efficiently
by using high order root finding methods applied
to the discretized determinant.

With the efficiency of the approach of
\cite{zhao2015robust} for the drum problem in mind,
we develop an integral equation based method for the
Stokes eigenvalue problem.
%
This requires that a layer
potential representation of the solution
of \cref{eq:ostokes} be given and that the resulting BIE
is not invertible precisely when $k^2$ is an eigenvalue.
%
The first requirement is straightforward to
satisfy because the
well-known layer potential representations for the
modified Stokes equation~\cite{Pozrikidis1992,biros2002embedded,
  jiang2013second,ladyzhenskaya1969mathematical}
are directly applicable.
%
Proving the invertibility of the associated operators
away from the eigenvalues is a more involved task
and forms the bulk of the theoretical component
of this paper.

\subsection{Relation to other work}

While integral equation based methods for the
related ``buckling'' eigenvalue problem
(which is equivalent  on simply connected domains
\cite{kelliher2009eigenvalues})
have been considered previously,
these typically relied on first-kind integral
equation formulations of the underlying PDE,
i.e. formulations in which the BIE operator is
compact \cite{kitahara2014boundary,antunes2011buckling}.
%
This is unsatisfying from a numerical
perspective, because the spectrum of a
compact operator either contains zero
or has zero as a limit point
(by design, the representations in
\cite{kitahara2014boundary,antunes2011buckling}
are not injective precisely when
$k^2$ is an eigenvalue).
%
This obscures the relation between
the non-invertibility of discrete approximations
of the operator and the eigenvalues;
in particular, common measures of the
``non-invertibility'' of a matrix, like the
smallest singular value or the determinant,
converge rapidly to zero for all values
of $k^2$ as the boundary is refined.
The measure of whether $k^2$ is an
approximate eigenvalue is then {\em relative to the
  current grid} for first kind formulations.

The classical single and double layer representations
for oscillatory Stokes considered in this paper
result in second kind equations, i.e. integral equations
of the form $\mathcal{I} - \mathcal{K}_k$ where
$\mathcal{K}_k$ is compact.
%
Such equations have a more satisfying theory
\cite{reed1972methods,colton1983integral,kress1989linear},
which translates well to numerical implementation
\cite{atkinson2009numerical,bornemann2010numerical,
  hackbusch2012integral,zhao2015robust}.
%
The use of a second kind representation is standard
for the drum problem \cite{backer2003numerical,zhao2015robust}
and was used recently to compute the vibrating
modes of thin, clamped plates~\cite{lindsay2018boundary}.

\subsection{Paper outline and contributions}

The rest of this paper proceeds as follows.
%
In \cref{sec:prelim}, we set the notation, provide some
mathematical preliminaries, and review the
properties of single and double layer potentials
for the oscillatory Stokes equations.
%
Then, in \cref{sec:analysis}, we develop the necessary
theory for proving the main results of this work 
(\cref{thm:dlmain,thm:cfmain}),
which show that the BIEs resulting from these
layer potential representations are not invertible
precisely when $k^2$ is an eigenvalue.
%
These theoretical developments include a detailed
discussion of the uniqueness of oscillatory Stokes
boundary value problems in exterior domains.
%
To the best of our knowledge, the invertibility
and uniqueness results are new to the literature.
%
\Cref{sec:dets} then outlines how the Fredholm determinant
can be used in the oscillatory Stokes context.
%
In \cref{sec:numerical}, we describe the numerical
methods we use to discretize the BIEs and to perform
determinant calculations.
%
At the moderate frequencies considered in this
paper, we find that standard fast-direct
solvers provide a reasonably efficient determinant
evaluation.
%
That section also includes numerical experiments
which demonstrate some of the paper's analytical
claims as well as the effectiveness of the overall
framework.
%
Finally, we provide some concluding thoughts,
describe plans for future research,
and outline some open questions in
\cref{sec:conclusion}.
