\section{Introduction}

The planar incompressible Stokes equations describe
creeping flows in two dimensions. Let $\Omega \subset \R^2$
be a bounded domain with $C^2$ boundary. The eigenvalue problem seeks
values $k^2$ such that 

\begin{equation}
\begin{aligned}
  -\Delta \uu + \nabla p &= k^2 \uu \quad \textrm{in} \quad
  \Omega \label{eq:ostokes} \; , \\
  \nabla \cdot \uu &= 0 \; ,
\end{aligned}
\end{equation}
subject to boundary conditions, has a non-trivial solution $(\uu,p)$.
In this work, we consider the eigenvalue problem subject to 
the Dirichlet boundary condition,
\begin{equation}
  \uu = \bz \quad \textrm{on} \quad \Gamma \label{eq:ostokes_dir} \; .
\end{equation}
It is well known that there is a countable collection of such
values $0 < k_{1}^{2} \leq k_{2}^2 \leq \ldots \uparrow \infty$,
counting multiplicites.

\begin{remark}
  When $k = i\alpha$, the differential equation
  \cref{eq:ostokes} is known as the modified Stokes
  equation. As there appears to be no preferred
  name for the equation with real-valued $k$,
  we will refer to \cref{eq:ostokes} as the
  oscillatory Stokes equation.
\end{remark}

We closely follow the approach of Zhao and Barnett
for computing the Laplace eigenvalues~\cite{zhao2015robust}.
%
The fact that layer potentials can be used to compute
eigenvalues is based on the following sequence of
facts: 1) represent the solution 
as a layer potential with 
an unknown density, and choose the representation such that
you get a second kind integral equation for the unknown density;
2) establish that the intergral equation is not invertible 
only at the Dirichlet eigenvalues for Laplace's equation;
3) prove that the zeros of the Fredholm determinant 
(when the compact operator is in trace class) 
line up with the Dirichlet eigenvalues; 
4) and finally, show that when the integral equation
is discretized using a Nystr\"{o}m method, the zeros of 
the determinant 
of the discrete linear system 
converge exponentially
to the zeros of the Fredholm determinant. 

%
While the integral representations for the modified
Stokes equation are directly applicable to the oscillatory
Stokes equations, 
proving invertibility of the associated operators
away from the eigenvalues is a more involved task
and forms the bulk of this section.
%
This in turn requires proving uniqueness results
for various interior and exterior boundary value problems
for the oscillatory Stokes equations.
%
To the best of our knowledge, the uniqueness results,
particularly
for exterior domains are new to the literature.

