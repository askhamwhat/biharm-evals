\section{Conclusion}
\label{sec:conclusion}

In the preceding, we have demonstrated a
BIE framework for computing the eigenvalues
of the Stokes operator in the plane which
is robust and scalable.
%
To justify the
approach, we developed a uniqueness theory
for the oscillatory Stokes equations in
exterior domains in analogy with the
discussion of the Helmholtz equation in
\cite{colton1983integral}.
%
This lead to the primary theoretical
results of the paper which show that the
BIEs resulting from double layer and
combined-field representations of the
velocity field are 
not invertible precisely when $k^2$ is
an eigenvalue on simply connected
and multiply connected domains, respectively.
%
As in \cite{zhao2015robust}, the costliness
of performing the nonlinear minimization
associated with computing these eigenvalues
can be alleviated by computing instead the
approximate zeros of the discrete Fredholm
determinant.

The results of this paper can be
extended in a number of ways.
%
The theory
extends directly to three dimensions, where
computational efficiency and numerical
implementation will be the primary concern.
%
In the numerical examples above, we
consider only domains with differentiable
boundaries for simplicity.
%
However, there has been recent progress
toward efficient discretization of the
layer potentials of elliptic operators
on domains with corners
\cite{helsing2008corner,serkh2016solution,rachh2017solution,helsing2018integral}
which makes such problems tractable.
%
Of course, the theoretical considerations
are different for such domains.
%
Computing the Stokes eigenvalues of regions
with corners is the topic of ongoing
research.

As noted in the introduction, the
eigenvalues of the Stokes operator are
equivalent to the so-called
``buckling'' eigenvalues of the
biharmonic operator on simply connected
domains~\cite{kelliher2009eigenvalues}.
%
This can be seen through the stream function
formulation  of the oscillatory Stokes
equation, i.e. setting $\bu = \nabla^\perp \psi$
where $\psi$ now satisfies

\begin{equation*}
  -\Delta^2 \psi = k^2 \Delta \psi \quad \textrm{in} \quad \Omega\; .
\end{equation*}
Note that the buckling problem enforces the
clamped, or first Dirichlet, boundary
condition on $\psi$

\begin{equation*}
  \psi = \partial_\nu \psi = 0 \; \quad \textrm{on} \quad \Gamma.
\end{equation*}
On a multiply connected domain, it is possible for
an eigenfield to have no corresponding stream
function which is still clamped, so that the
buckling eigenvalues are a subset of the Stokes
eigenvalues.
By adapting the approach of \cite{rachh2017integral},
a suitable layer potential representation of the
buckling problem can be derived based on the
oscillatory Stokes layer potentials.
%
This is the subject of a follow-up paper which is in
preparation.

There are some interesting questions to answer
on the use of Fredholm determinants in numerical
calculations.
%
As observed in \cite{zhao2015robust}, the
combined-field representation causes some
difficulty in that the Fredholm determinant
is not defined when the single layer, which
is not trace-class, is included.
%
Zhao and Barnett~\cite{zhao2015robust}
suggest looking into
representations of the form $\cI-2\cD
-2i\eta\cS^2-2\cW$,
which would have a well-defined Fredholm
determinant.
%
The relative performance of
such an approach should be explored.
%
Further, as discussed above, the
convergence of the determinant of
integral equations discretized with
panel-corrected schemes (as described in
\cref{sec:numerical}) is yet to be proved.
%
Finally, it is worth exploring alternatives
to the Fredholm determinant which perform well
for layer potentials that are not trace class
on the boundary.

