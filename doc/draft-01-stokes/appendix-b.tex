\section{Sketch of proof}
\label{sec:appb}
Let $T_{n}$ denote the space of trigonometric polynomials
$\{ e^{-int}, \ldots e^{int} \}$. 
Let $\mathcal{K}$ denote the Helmholtz double layer integral operator
given by
\begin{equation}
\mathcal{K}[\phi] = \frac{i}{2}\int_{\Gamma} k H_{1}^{1}(k|x(t)-y(t)|) \frac{(x(t)-y(t))\cdot n(y(t))}{|x(t)-y(t)|} \phi(y(t)) dS_{y} \,
\end{equation}
where $k$ the Helmholtz parameter is assumed to be a dirichlet eigenvalue of the simply connected
domain $\Omega$ whose boundary is given by $\Gamma$, $(x(t),y(t)): [0,2\pi] \to \Gamma$ is
an arclength parameterization of the boundary, and $H_{1}^{1}$ is the Hankel function
of the first kind of order $1$.
It is well known that if $\Gamma$ is of class $C^{\ell,\alpha}$, then the
integral operator $\mathcal{K}: C^{\ell-1,\alpha} \to C^{\ell-1,\alpha}$ is a compact operator.
Furthermore, as shown in Zhao and Barnett, at the Dirichlet eigenvalues, the operator
$I + \mathcal{K}$ is not invertible and has a $p-$dimensional null-space, where
$p$ is the multiplicity of the eigenvalue. 
By Fredholm theory, the adjoint $I + \mathcal{K}^{\ast}$ also has a $p$-dimensional null space.
For convenience, we restrict our attention to the case $p=1$. This
part of the analysis extends to non-simple eigenvalues as well. 
Let $\phi_{0}$ denote
the null vector. 
Let $\psi_{0}$ denote the null vector of $I+\mathcal{K}^{\ast}$.

In order to prove the regularity of the eigenfunctions, we need to analyze
the problem in $C^{\ell-1,\alpha}$ space. 
However in order to obtain a bound for the magnitude of the Fredholm
determinant, it is more convenient to work in $L^{2}$. It is well known
that the integral operator $I + \mathcal{K}$ is a second kind
Fredholm operator from $L^{2} \to L^{2}$ as well.
For the remainder of the section, we will work with $L^{2}$.


Let $K_{n}: T_{n} \to T_{n}$ 
denote a discretization of $\mathcal{K}$ 
using kress quadrature. 
Let $P_{n}$ denote the orthogonal 
projection operator from $L^{2} \to T_{n}$.
Let $P_{n,\phi_{0}^{\perp}}: T_{n} \to T_{n}$ denote the orthogonal
projection on to the orthogonal complement of $P_{n} \phi_{0}$.
Let $P_{n,\psi_{0}^{\perp}}: T_{n} \to T_{n}$ denote the
analogous operator for $\psi_{0}$. 

The matrix $I+K_{n}$ can be rewritten as a map from the
following two bases for $T_{n}: \{ P_{n} \phi_{0}, \text{basis for }
P_{n}\phi_{0}^{\perp} \}$, and for the range the basis:
$\{ P_{n} \psi_{0}, \text{basis for } P_{n} \psi_{0}^{\perp} \}$.
In these bases, the linear operator $I_{n} + K_{n}$ is given by
\begin{equation}
\begin{bmatrix}
A_{n} & B_{n} \\
C_{n} & D_{n} 
\end{bmatrix} \, .
\end{equation}

The operator $A_{n}$
is the restriction of $I_{n} + K_{n}$ as a map from $P_{n} \phi_{0}$ 
to $P_{n} \psi_{0}$. Thus, it is just a number and it is given by
\begin{equation}
A_{n} = \left(P_{n} \psi_{0}, (I_{n} + K_{n}) P_{n} \phi_{0} \right) \,.
\end{equation}

The operator $D_{n}$ is the restriction of $I_{n} + K_{n}$ 
as a map from $P_{n} \phi_{0}^{\perp}$ 
to $P_{n} \psi_{0}^{\perp}$.
The following statement requires a more careful proof.
{\color{red} As $n \to \infty$, $D_{n}$ converges pointwise to the restriction
of $I + K: \phi_{0}^{\perp} \to R(I+K)$. Moreover, the collection
of operators $K_{n}$ are collectively compact, and hence their restrictions
are also collectively compact. 
Since the operator $I+K: \phi_{0}^{\perp} \to R(I+K)$ is an invertible
operator, by standard estimates, for $n$ large enough $D_{n}$ is also 
an invertible operator.}

The operator $B_{n}$ is the restriction of $I_{n} + K_{n}$ as a map from 
$P_{n} \phi_{0} ^{\perp} \to P_{n} \psi_{0}$ and is given by
\begin{equation}
B_{n} \phi = \left( P_{n} \psi_{0}, (I_{n} + K_{n}) 
(P_{n} \phi -( P_{n} \phi, P_{n} \phi_{0}) P_{n} \phi_{0}) \right) P_{n} \psi_{0}  \, .
\end{equation}

And finally the operator $C_{n}$ is the restriction of $I_{n} + K_{n}$ as a map from
$P_{n} \phi_{0} \to P_{n} \psi_{0}^{\perp}$ and is given by
\begin{equation}
C_{n}P_{n} \phi_{0} = (I_{n} + K_{n}) P_{n} \phi_{0} - \left( (I_{n} + K_{n}) P_{n}\phi_{0}, P_{n} \psi_{0} \right)P_{n} \psi_{0}
\end{equation}

%Thus for large enough $n$, 
%$f_{n} = P_{n} \phi_{0} - (P_{n} \phi_{0}, P_{n} \psi_{0}) P_{n} \psi_{0} \neq 0$
%and $f_{n} \in P_{n} \psi_{0}^{\perp}$.
%Similarly $g_{n} = P_{n} \psi_{0} - (P_{n}\phi_{0}, P_{n} \psi_{0}) P_{n} \phi_{0} \neq 0$
%and $g_{n} \in P_{n} \phi_{0}^{\perp}$.

By block linear algebra, the determinant of the matrix above is also given by
\begin{equation}
\det{
\begin{bmatrix}
A_{n} & B_{n} \\
C_{n} & D_{n} 
\end{bmatrix} 
}
= \det{D_{n}} \cdot (A_{n} - B_{n} D_{n}^{-1} C_{n}) \,.
\end{equation}

First, we observe that $D_{n}$ is the restriction of an operator in trace class
with a kernel whose leading order singularity is $r^{2} \log{r}$. 
{\color{red} Thus, $D_{n}$ is itself in trace class, and extending the proof of Bornemann to this case,
we observe that the discrete determinant of $D_{n}$ converges to the Fredholm determinant
of $D$ at a rate of at least $O(1/n)$. This in particular implies that
$\sup_{n} |\det{D_{n}}| <\infty$}.


Using Cauchy-Schwarz, we get that
\begin{equation}
\begin{aligned}
|A_{n}| &\leq |P_{n} \psi_{0}| |(I_{n} + K_{n})P_{n} \phi_{0}| \\
|B_{n}D_{n}^{-1} C_{n}| &\leq |B_{n}||D_{n}^{-1}|(1+ |P_{n} \psi_{0}|^2) |(I_{n} + K_{n})P_{n} \phi_{0}| \,.
\end{aligned}
\end{equation}
$B_{n}$ is a discretized block of a bounded operator and hence is bounded, 
and $D_{n}^{-1}$ is a bounded operator by the discussion above. 

Combining these results together, we get that
\begin{equation}
\left| \det{\begin{bmatrix} A_{n} & B_{n} \\ C_{n} & D_{n} \end{bmatrix}} \right| \leq C |(I_{n} + K_{n})\phi_{n}| \, ,
\end{equation}
where $C = \det{D_{n}} |B_{n}||D_{n}^{-1}| (1 + |P_{n} \psi_{0}|^2)$.

