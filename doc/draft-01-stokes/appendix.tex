\section{Uniqueness of the Neumann problem in the exterior
  of several obstacles}
\label{app:neuuniqueness}
In this section, we prove the uniqueness result for the oscillatory
Stokes problem with Neumann boundary conditions on the exterior
of several simply connected obstacles. 
\begin{thrm}[Uniqueness of the Exterior Neumann Problem]
  Suppose that $\Omega$ is the union of a finite collection 
  of simply connected domains, i.e. $\Omega = \bigcup_{i=1}^m \Omega_{i}$
  for some $m \in \N$, with $C^{2}$ boundaries, 
  and let $E = \R^{2} \setminus \bar{\Omega}$ denote its exterior;
  see~\cref{fig:ext_dom} for an example with $m=4$.
  Let $\Gamma_{i}$ denote the boundary of $\Omega_{i}$, 
  and $\Gamma = \bigcup_{i=1}^{m} \Gamma_{i}$ denote the boundary
  of $\Omega$.
  Suppose that $\Im(k)\geq 0$ and 
  that $(\bu,p)$ is a radiating solution to the oscillatory Stokes
  equation in $E$ with $\bt = 0$ on the boundary $\Gamma$, then
  $\bu \equiv 0$ in $E$.
\end{thrm}

\begin{proof}
Since $\bt = 0$ on $\Gamma$, it follows
from~\cref{eq:repinfest} that
\begin{equation}
\lim_{r\to\infty}
\int_{|\by|=r} \left( |\bt|^2 + |k|^2 |\bu|^2 \right) dS +
2 \Im(k) \int_{E \cap B_{r}(0)} \left(|k|^2 |\bu|^2 + |\be(\bu)|^2 \right)
dV = 0 \; . \nonumber
\end{equation}

Suppose that $\Im(k) > 0$. It is then immediate
that $\bu \equiv 0$ in $E$.

Suppose that $k$ is real. It is clear that
\begin{equation}
\lim_{r\to\infty} \int_{|\by|=r} |\bu|^2 dS = 0 \, . \nonumber
\end{equation}
Thus, the conditions on $\bu$ and $k$ in \cref{lem:rellich}
are satisfied, and each component of $\bu$ is a harmonic function
with $\bu \to 0$ as $r \to \infty$. Furthermore, as observed
in \cref{rmk:harmu}, $k^2 \bu = \nabla p$. Then, the boundary
condition becomes $0 = \bt = -p \bnu + 2 \nabla \partial_\nu p/k^2$.
Because $0 = \btau \cdot \bt = 2\partial_{\tau\nu} p/k^2$,
$\partial_\nu p  = c_{i}$ on $\Gamma_{i}$ for each $\Gamma_{i}$,
where $c_{i}$ is a constant. 
Observe that $|\bu|$ and $|\be(\bu)|$ must be $O(1/r)$ as
$r\to\infty$. Thus,
for a radiating pair $(\bu,p)$ with $p$ harmonic,
we have that $|p| = O(1/r)$ and $|\nabla p| = O(1/r^2)$
as $r\to\infty$.
Since the boundary is $C^{2}$ and the boundary data for $p$
is analytic, we conclude that $p$ is $C^{2}$ in
$\overline{E}$.
Furthermore, $\bt=0$ implies $p \bnu = 2\nabla \partial_{\nu} p/k^2$, 
and taking the dot
product with $\bnu$, we get
$p = 2\partial_{\nu \nu}p/k^2$. 
It then follows that
$p = -2\partial_{\tau \tau}p/k^2$ on $\Gamma$ since 
$p$ is harmonic in $E$ and $C^{2}$ in $\overline{E}$. 
Since $p$ satisfies the radiation condition at $\infty$,
we have
\begin{equation}
\begin{aligned}
\int_{E} |\nabla p|^2 dV &= 
\sum_{i=1}^{m} \int_{\Gamma_{i}} p \partial_{\nu}p \,dS \\
&= \sum_{i=1}^{m} c_{i} \int_{\Gamma_{i}} p \,dS \quad \text{(Since $\partial_{\nu} p = c_{i}$ on
$\Gamma_{i}$)} \\ 
&= -\sum_{i=1}^{m} \frac{2c_{i}}{k^2} \int_{\Gamma_{i}} \partial_{\tau \tau} p \,dS 
\quad \text{(Since $p = -\partial_{\tau \tau}p/k^2$ on $\Gamma$)} \\
&= 0
\end{aligned}
\end{equation}
Thus, $p$ is a constant in $E$. Furthermore, since $p\to 0$ at $\infty$, 
we conclude that $p\equiv 0$ in $E$. 
Finally, since $k^2 \bu = \nabla p$, we conclude that $\bu\equiv 0$ in 
$E$.
\end{proof}



\section{Neumann eigenvalues and eigenfunctions on the unit disk}
In this section, we analytically compute some of the Neumann eigenvalues
on the unit disk for the Stokes operator.

Suppose that $\bu$ is of the form
\begin{equation}
\bu = \nabla (\alpha r^{n} \sint - \beta r^{n} \cost) + \nabla^{\perp}(\gamma \jnkr \sint + \delta \jnkr \cost) \, ,
\end{equation}
and the pressure is given by
\begin{equation}
p = k^2 (\alpha r^{n} \sint - \beta r^{n} \cost) \, ,
\end{equation}
where $\alpha,\beta,\gamma,\delta$ are constants.

We first show that this pair satisfies the oscillatory stokes equation
in the unit disk.
\begin{lemma}
$(\bu,p)$ satisfy the oscillatory Stokes equation.
\end{lemma}
\begin{proof}
$\bu$ is divergence free since
\begin{equation}
\nabla \cdot \bu  = \Delta (\alpha r^{n} \sint - \beta r^{n} \cost) = 0 \, ,
\end{equation}
as $r^{n} \cost$,  $r^{n} \sint$ are harmonic functions. 
To show that $(\bu,p)$ satisfies the PDE, note that
$r^{n} \cost$ and $r^{n} \sint$ are harmonic
and $\jnkr \cost$, $\jnkr \sint$ satisfy Helmholtz's equation.
\begin{equation}
\begin{aligned}
(\Delta + k^2) \bu &= (\Delta + k^2) \nabla (\alpha r^{n} \sint - \beta r^{n} \cost) \, + \\
\quad & \hspace*{15ex} (\Delta + k^2) \nabla^{\perp} (\gamma \jnkr \sint + \delta \jnkr \cost) \, , \\
&= k^2 \nabla (\alpha r^{n} \sint - \beta r^{n} \cost) \\ 
&= \nabla p
\end{aligned}
\end{equation}
\end{proof}

Let $\hat{r}, \hat{\theta}$ denote the unit vectors in polar coordinates.
Then  $\hat{r}$ is the unit normal on the boundary of the unit disk.
The surface traction on the boundary of the unit disk is then given by 
\begin{equation}
\bt = -p \hat{r} + \nabla (\bu \cdot \hat{r}) + \partial_{r} \bu \, .
\end{equation}
If $(\bu,p)$ are as defined above, then 
a long but tedious calculation shows that
\begin{equation}
\begin{aligned}
t_{r} &= \cost \left( \beta k^2 r^{n} -2\beta n(n-1)r^{n-2} - 2\gamma n \ddr \frac{\jnkr}{r}  \right) \, + \\
& \hspace*{15ex} \sint \left(-\alpha k^2 r^{n} + 2 \alpha n(n-1)r^{n-2} + 2 \delta n \ddr \frac{\jnkr}{r} \right) \, , \\
t_{\theta} &= \cost \left( \alpha r^{n-2} (n^2 + n(n-1)) + \delta \left( \frac{n^2}{r^2} \jnkr + k^2 \jnppkr \right)  \right) + \\
& \hspace*{15ex} \sint \left( \beta r^{n-2} (n^2 + n(n-1)) + \gamma \left( \frac{n^2}{r^2} \jnkr + k^2 \jnppkr \right) \right) \, .
\end{aligned}
\end{equation}
Let $h_{k} = \ddr (\jnkr/r)|_{r=1}$.
The values of $k$ for which there exists a non-zero solution $\alpha,\beta,\gamma,\delta$ 
to the following system of equations
\begin{equation}
\begin{bmatrix}
0 & 0 & k^2 - 2n(n-1) & -2nh_{k} \\
0 & 0 & 2n^2 -n & n^2 \jnk + k^2 \jnppk \\
-k^2 + 2n(n-1) & 2nh_{k} & 0 & 0 \\
2n^2 -n & n^2 \jnk + k^2 \jnppk & 0 & 0
\end{bmatrix}
\begin{bmatrix}
\alpha \\ \delta \\ \beta \\ \gamma
\end{bmatrix}
=
\begin{bmatrix}
0 \\ 0 \\ 0 \\ 0
\end{bmatrix}
\, ,
\end{equation}
are the Neumann eigenvalues of the unit disk.

A collorary of the above statement and the structure of the matrix is that
the interior Neumann eigenvalues of the unit disk satisfy
\begin{equation}
2n^2(2n-1)h_{k} - (n^2 \jnk + k^2 \jnppk)(-k^2 + 2n(n-1)) = 0 \, .
\end{equation}
