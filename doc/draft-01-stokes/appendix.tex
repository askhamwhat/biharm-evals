\section{Dirichlet eigenvalues and eigenfunctions on the annulus \label{sec:annul_dir_exact}}
In this section, we compute some of the Dirichlet eigenvalues
corresponding to the a subset of radially symmetric eigenfunctions
on the annulus. 
In polar coordinates $(r,\theta)$, consider the annulus defined by 
$R_{1}<r<R_{2}$. 
Suppose that $\bu$ is of the form 
\begin{equation}
\bu = \nabla^{\perp}\left( \alpha H_{0}(kr) + \beta J_{0}(kr) \right) \, ,
\label{eq:annulus_direig}
\end{equation}
and $p=0$.

Clearly, this pair satisfies the osciallatory Stokes equation
with parameter $k$, since $J_{0}(kr)$ and $H_{0}(kr)$ 
satisfy the Helmholtz equation on the annulus.

Let $\hat{r},\hat{\theta}$ denote the unit vectors in polar coordinates.
A simple calculation shows that 
\begin{equation}
\begin{aligned}
u_{r} &= \bu \cdot \hat{r}  = 0 \, \\
u_{\theta} &= \bu \cdot \hat{\theta} = k(\alpha  H_{0}'(kr) + \beta J_{0}'(kr)) \, .
\end{aligned}
\end{equation}

This in particular implies that on $r=R_{1}$,
$u_{\theta}$ takes on the constant value,
$k(\alpha H_{0}'(kR_{1}) + \beta J_{0}'(kR_{1}))$.
Similarly, on $r=R_{2}$, 
$u_{\theta}$ takes on the constant value,
$k(\alpha H_{0}'(kR_{2}) + \beta J_{0}'(kR_{2}))$.

Thus, if $k$ satisfies, 
\begin{equation}
H_{0}'(kR_{1}) J_{0}'(kR_{2}) - H_{0}'(kR_{2})J_{0}'(kR_{1}) = 0 \, ,
\end{equation}
and for those values of $k$ if 
$\alpha,\beta$ are non-zero solutions to system of equations 
\begin{equation}
\begin{bmatrix}
H_{0}'(kR_1) & J_{0}'(kR_{1}) \\
H_{0}'(kR_{2}) & J_{0}'(kR_{2}) 
\end{bmatrix}
\begin{bmatrix}
\alpha \\
\beta
\end{bmatrix}
=
\begin{bmatrix}
0 \\
0
\end{bmatrix}
\, ,
\end{equation}
then $k$ is a Dirichlet eigenvalue and $\bu$ 
defined by~\cref{eq:annulus_direig}
is the corresponding eigenfunction.

\section{Neumann eigenvalues and eigenfunctions on the unit disk}
In this section, we analytically compute some of the radially symmetric 
Neumann eigenvalues on the unit disk for the Stokes operator.

Suppose that $\bu$ is of the form
\begin{equation}
\bu = \nabla^{\perp} \jkr \, ,
\end{equation}
and the pressure is given by $p=0$, since
$\bu$ satisfies $(\Delta + k^2)\bu = 0$.

Then, the surface traction $\bt$ on the disk of radius $r$ 
is given by
\begin{equation}
\bt = 
\left( -\frac{k}{r^2}\jpkr  + \frac{k^2}{r} \jppkr\right)
\begin{bmatrix}
\sint \\
-\cost
\end{bmatrix} \, .
\end{equation}

Thus, $k$ which satisfies
\begin{equation}
-k \jpk + k^2 \jppk = 0 \, ,
\end{equation}
are Neumann eigenvalues on the unit disk.
The first $4$ roots of the above equation are 
\begin{equation}
\begin{aligned}
k &= 5.135622301840683 \\
  &= 8.417244140399865 \\
  &= 11.61984117214906 \\
  &= 14.79595178235126
\end{aligned}
\end{equation}

