\documentclass[preprint,12pt]{article} 
\usepackage{b}
\usepackage[a4paper]{geometry}
\geometry{top=1.0in, bottom=1.0in, left=1.0in, right=1.0in}
\usepackage[small,it]{caption}
\usepackage{empheq}

\usepackage{subcaption}
\usepackage{amsmath}

\usepackage{cleveref}
\crefname{equation}{}{}
\Crefname{equation}{}{}
\crefname{eqnarray}{}{}
\Crefname{eqnarray}{}{}
\crefname{lemma}{lemma}{lemma}
\crefname{cor}{corollary}{corollary}

%%%%%%%%%%%%%%%%%%%%%%%%%%%%%%%%%%%%%%%%%%%%%%%%%%%%%%%%%%%%

\def\bz{{\boldsymbol 0}}
\def\br{{\boldsymbol r}}
\def\bc{{\boldsymbol c}}
\def\bh{{\boldsymbol h}}
\def\bn{{\boldsymbol n}}
\def\bT{{\boldsymbol T}}
\def\bt{{\boldsymbol t}}
\def\xx{{\boldsymbol x}}
\def\bx{{\boldsymbol x}}
\def\yy{{\boldsymbol y}}
\def\by{{\boldsymbol y}}
\def\ss{{\boldsymbol s}}
\def\tt{{\boldsymbol t}}
\def\uu{{\boldsymbol u}}
\def\bu{{\boldsymbol u}}
\def\be{{\boldsymbol e}}
\def\bl{{\boldsymbol \ell}}
\def\ff{{\boldsymbol f}}
\def\bg{{\boldsymbol g}}
\def\GG{{\bf G}}
\def\Gbh{G^{\textrm{BH}}}
\def\Glap{G^{\textrm{L}}}
\def\Ghelm{G^{\textrm{H}}}
\def\Kfark{K^{\textrm{F}}}
\def\II{{\bf I}}
\def\TT{{\bf T}}
\def\bS{{\bf S}}
\def\bD{{\bf D}}
\def\cN{{\mathcal N}}
\def\cS{{\mathcal S}}
\def\cD{{\mathcal D}}
\def\cW{{\mathcal W}}

\def\bsigma{{\boldsymbol \sigma}}
\def\bpsi{{\boldsymbol \psi}}
\def\bmu{{\boldsymbol \mu}}
\def\bnu{{\boldsymbol \nu}}
\def\brho{{\boldsymbol \rho}}
\def\btau{{\boldsymbol \tau}}
\def\bigo{\mathcal{O}}
\def\littleo{o}

\def\Im{{\textrm{\textup{Im}}}}

\newcommand\dwdn{\frac{\partial w}{\partial n}}
\newcommand{\cI}{\mathcal I}
\newcommand{\cC}{\mathcal C}
\newcommand{\bbmat}{\begin{bmatrix}}
\newcommand{\ebmat}{\end{bmatrix}}

\newtheorem{remark}{Remark}
\newtheorem{proposition}{Proposition}
\newtheorem{lem}{Lemma}
\newtheorem{cor}{Corollary}
\newtheorem{thrm}{Theorem}
\newtheorem{definition}{Definition}


%%%%%%%%%%%%%%%%%%%%%%%%%%%%%%%%%%%%%%%%%%%%%%%%%%%%%%%%%%%%%

\title{A note on representation for Helmholtz equations 
on multiply connected domains}
\author{Travis Askham and Manas Rachh}
\date{\today}

\begin{document}

\maketitle

%%%%%%%%%%%%%%%%%%%%%%%%%%%%%%%%%%%%%%%%%%%%%%%%%%%%%%%%%%%%

Suppose that $\Omega$ is a multiply connected domain.
The theorem extends for general regions, but for convenience let us assume
that the domain has one hole. Let $\Gamma_{0}$ denote the outer boundary
and let $\Gamma_{1}$ denote the obstacle boundary. 
Let $E$ denote the exterior of $\Gamma_{0}$ and let $D_{1}$ denote the 
interior of $\Gamma_{1}$ (see~\cref{fig:1}). 
Suppose that $x_{1}$ is in the interior of $D_{1}$.

Suppose that we are solving 
\begin{align} 
(\Delta + k^2) u &= 0\quad x \in \Omega \, ,\label{eq:pde} \\
u &= g \quad x \in \Gamma \, , \label{eq:bc}
\end{align}
using the following representation
\begin{equation}
u(x) = \int_{\Gamma} \frac{\partial G_{k}}{\partial n}(x,y) \sigma(y) dS_{y} + \left(\int_{\Gamma_{1}} \sigma(y) dS_{y}\right)
H_{0} (k |x-x_{1}|) \, .
= D_{k}[\sigma](x) + L[\sigma](x) \, .
\end{equation}

On imposing the boundary conditions, we obtain the following integral equation for $\sigma$
\begin{equation}
-\frac{1}{2}\sigma + D_{k} [\sigma] + L [\sigma] = A[\sigma] =  g \, ,
\end{equation}
where
\begin{equation}
A = -\frac{1}{2}I + D_{k} + L
\end{equation}

\begin{thrm}
Suppose that $k$ is not a Dirichlet eigenvalue of $\Omega$ but is a Neumann eigenvalue of $\Omega_{1}$, 
and suppose that $x_{1}$ is not on the nodal set for the correspdonding eigenfunction.
Then $\cN(A) \subset \cN(-\frac{1}{2}I + D_{k})$.
\end{thrm}

\begin{proof}
Suppose that $\sigma \in \cN(A)$. Then setting $u = D_{k} \sigma + L[\sigma]$ and imposing the boundary conditions, 
we note that $u$ satisfies the homogeneous Dirichlet boundary value problem for Helmholtz's equations. 
Since have assumed that $k$ is not a Dirichlet eigenvalue of $\Omega_{1}$, we conclude that
$u\equiv = 0$ on $\Omega$. 
Thus $\frac{\partial u}{\partial n} = 0$ when approached from the interior of $\Omega$. 
Standard arguments show that $\sigma =0$ on $\Gamma_{0}$ which follows from the uniqueness of solutions
to the exterior Neumann problem for Helmholtz equation with radiating boundary conditions.

Since the normal derivative of the double layer is continuous across the boundary and since L[$\sigma$] 
is smooth across the boundary as well, we conclude that $\frac{\partial u}{\partial n} = 0$ 
when viewed as the interior limit from $D_{1}$. 


Suppose that $v$ is the interior Neumann eigenfunction associated with eigenvalue $k$. 
Then applying Green's identity on $D_{1}$, we get
\begin{equation}
\int_{D_{1}} v (\Delta + k^2) u - u (\Delta+k^2) v =
\int_{\Gamma_{1}} v \frac{\partial u}{\partial n} - u \frac{\partial v}{\partial n} dS \, .
\end{equation}
$(\Delta + k^2)v = 0$ for $x\in D_{1}$ since $v$ is an interior Neumann eigenfunction.
$\frac{\partial v}{\partial n} = 0$ for $x\in \Gamma_{1}$ since $v$ is an interior Neumann eigenfunction.
$\frac{\partial u}{\partial n} = 0$ from above. 
$(\Delta+k^2)u =\int_{\Gamma_{1}} \sigma \delta_{x_{1}}$ due to the $H_{0}$ term, the double layer satisfies Helmholtz equation
in the interior as well, combining all of this, we get
\begin{equation}
\int_{\Gamma_{1}} \sigma v(x_{1}) = 0 \, ,
\end{equation}
since we have assumed that $x_{1}$ is not on the nodal set of an interior Neumann eigenfunction, we conclude that
$\int_{\Gamma_{1}} \sigma = 0$ which implies that $L\sigma = 0$.
This further implies that $\sigma \in \cN( -\frac{1}{2} I + D_{k})$ and proves the result. 
\end{proof}

This result means that if we can find a null vector of $-\frac{1}{2} I + D_{k}$ which satisfies
$\int_{\Gamma_{1}} \sigma = 0$, then that by tautology must've been a null-vector for our operator $A$.
Thus, if the dimension of the null-space is more than one-dimensional, then all hope is lost, as we can
always find a null-vector which integrates to $0$. So there is no hope on domains like squares or rectangles.
But if the null-space is one-dimensional, then there might be some hope by showing that 
the integral of the null vector must not equal to $0$. All in all it seems pretty grim that this theorem might
be true in general. 
\end{document} 
