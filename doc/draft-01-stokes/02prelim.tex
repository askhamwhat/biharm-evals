\section{Mathematical Preliminaries}

In this paper, vector-valued
quantities are denoted by bold, lower-case letters
(e.g. $\bh$), while tensor-valued quantities are bold
and upper-case (e.g. $\mathbf{T}$). 
Subscript indices of non-bold characters (e.g. $h_j$ or $T_{jkl}$)
are used to denote the entries within a vector ($\bh$) or tensor ($\bT$).
We use the standard Einstein summation convention; in other words, 
there is an implied sum taken over the repeated indices of 
any term (e.g. the symbol $a_{j} b_{j}$ is used to represent the sum
$\sum_{j} a_{j} b_{j}$).
If $\xx = (x_1,x_2)^\intercal$, then $\xx^\bot = (-x_2,x_1)^\intercal$.
Similarly, $\nabla^\bot = (-\partial_{x_2},\partial_{x_1})^\intercal$.
Upper-case script characters (e.g. $\mathcal{K}$) are reserved for
operators on Banach spaces, with $\mathcal{I}$ denoting the
identity.

For a given velocity field $\bu$ and pressure $p$, let $\bsigma(\bu,p)$
denote the Cauchy stress tensor, i.e.
\begin{equation}
\bsigma(\uu,p) = -p \II + 2 \be(\bu) \, ,
\end{equation}
where $\be(\bu)$ is the strain tensor given by
\begin{equation}
e_{ij}(\bu) = \frac{1}{2} \left( \partial_{x_j} u_i + \partial_{x_i} u_j \right) \; .
\end{equation}
When it is clear from context, we will drop the dependence of
$\bsigma$ on $\bu$ and $p$.


\subsection{Green's functions}

Let $\mathcal{L}_x$ denote a linear differential operator. A fundamental
solution $G(\xx,\yy)$ of $\mathcal{L}_x$ satisfies the equation
$\mathcal{L}_x G(\xx,\yy) = \delta_y(\xx)$ in the distributional sense, i.e.
for sufficiently smooth $f$
\begin{equation}
  \mathcal{L}_x \int_{\R^2} G(\xx,\yy) f(\yy) \, dy = f(\xx) \; .
  \nonumber
\end{equation}
We consider here
``free-space'' Green's functions, i.e. fundamental solutions which satisfy
some natural growth or decay conditions as $|\xx-\yy| \to \infty$.
The Green's function of the oscillatory biharmonic equation,
\begin{equation}
  \Delta ( \Delta + k^2 ) u = 0 \; , \label{eq:obiharm} \nonumber
\end{equation}
is given by 
\begin{equation}
  \Gbh(\xx,\yy) = \frac{1}{k^2}
  \left (\frac{1}{2\pi} \log |\xx-\yy| +
  \frac{i}{4} H_0^{(1)}(k|\xx-\yy|) \right ) \, ,
  \label{eq:Gbh}
\end{equation}
where $k$ is the Helmholtz paramater in the oscillatory biharmonic equation,
and $H_{0}^{1}(r)$ is the Hankel function of the first kind of order zero.
Note that this is a scaled difference of the Green's function for
Laplace, i.e.

\begin{equation}
  \Glap(\xx,\yy) = \frac{1}{2\pi} \log |\xx-\yy| \; , \nonumber
\end{equation}
and the Green's function for the Helmholtz equation

\begin{equation}
  \Ghelm(\xx,\yy) = -\frac{i}{4} H_0^{(1)}(k|\xx-\yy|) \; . \nonumber
\end{equation}

\subsection{An integral representation for the Stokes eigenvalue
  problem}

In the case that $k=i\alpha$ for some real-valued $\alpha$,
the equations \cref{eq:ostokes,eq:ostokescty}
are known as the modified Stokes equations and are of particular
interest for their application to the analysis and numerical
simulation of unsteady flow
\cite{pozrikidis1992boundary,biros2002embedded,
  jiang2013second,ladyzhenskaya1969mathematical}.
The equations are well-studied in that setting and
integral representations which lead to second kind
integral equations have been developed. We review
some of the relevant results here, translating to
the present, oscillatory setting.

\subsubsection{Oscillatory Stokeslets and stresslets}


In this section, we review the derivation of the
modified Stokeslet, but for the imaginary parameter
regime, which is the appropriate setting for the Stokes
eigenvalue problem.
The original modified Stokeslet is more well known,
and is of particular interest for its application to
numerical simulations of the unsteady Stokes equations
\cite{pozrikidis1992boundary,biros2002embedded}.

 Consider the solution of
\cref{eq:ostokes,eq:ostokescty} where a $\delta$-mass
centered at $\yy$ with strength $\ff$
has been added to the right-hand side of \cref{eq:ostokes}, i.e.

\begin{align}
  \nabla p - \Delta \uu - k^2 \uu &= \delta_\yy \ff \; ,
  \label{eq:ostokes_charge}  \\
  \nabla \cdot \uu &= 0 \; . \nonumber
\end{align}
Recall that

\begin{equation}
 \Delta \Glap(\xx,\yy) = \delta_\yy(\xx) \; . \label{eq:lapdelta}
\end{equation}
If we substitute \eqref{eq:lapdelta} into
\eqref{eq:ostokes_charge} and take the divergence,
we obtain

\begin{equation}
  p = \nabla \Glap(\xx,\yy) \cdot \ff \; . \nonumber
\end{equation}
We then have, formally,

\begin{align}
  \uu &= - (\Delta + k^2)^{-1} ( \Delta \Glap \ff
  - \nabla (\nabla \Glap \cdot \ff ) ) \nonumber \\
  &= \left ( -\Delta + \nabla \otimes \nabla \right )
  \Gbh \ff \; . \nonumber
\end{align}
The tensor

\begin{equation} \label{eq:ostokeslet}
  \GG = - \II \Delta \Gbh + \nabla \otimes \nabla \Gbh
\end{equation}
is then the equivalent of a Stokeslet
\cite{pozrikidis1992boundary} for the problem
\eqref{eq:ostokes}.

A related object is the stresslet, which is defined
in terms of the stress tensor of the velocity, pressure
pair induced by a Stokeslet. For these tensors, we find
that it is more convenient to express them in index notation
with the Einstein index summing convention.
Recall that the stress tensor $\bsigma$ is defined as 

\begin{equation}
  \sigma_{ij} = -p \delta_{ij} + \left ( \partial_{x_j}u_i
  +\partial_{x_i} u_j \right ) \; , \nonumber
\end{equation}
where $\delta_{ij}$ is the standard Kronecker delta notation.
The stresslet $\TT$ is defined to be

\begin{align}
  T_{ijk} &= - \partial_{x_j} \Glap \delta_{ik}
  + \partial_{x_k} \left ( -\Delta \Gbh \delta_{ij} +
  \partial_{x_i} \left(\partial_{x_j} \Gbh \right) \right)
  \nonumber \\
  & \qquad+ \partial_{x_i} \left ( -\Delta \Gbh \delta_{kj} +
  \partial_{x_k} \left(\partial_{x_j} \Gbh \right) \right)
  \; . \label{eq:ostress} \nonumber
\end{align}
Let $u_i = G_{ij} f_j$ and $p = \partial_{x_i} \Glap f_i$ be a
solution of the Stokes equations induced by a Stokeslet.
Then the corresponding stress tensor is given by
$\sigma_{ik} = T_{ijk} f_j$.

For the layer potentials of the next section, the following
formulas are useful. Let $\bnu$ be a given vector. When
summing over the third index, we obtain

\begin{equation}
  \TT_{\cdot,\cdot,k} \nu_k = -\bnu \otimes \nabla \Glap
  + \partial_\nu \left ( -\Delta \Gbh \II
  + \nabla \otimes \nabla \Gbh \right)
  + \nabla \otimes \left ( -\Delta \Gbh \bnu
  + \partial_{\nu} \nabla \Gbh \right) \; . \nonumber
\end{equation}
Let $\btau = \bnu^\bot$. Then

\begin{equation}
  \TT_{\cdot,\cdot,k} \nu_k = -\bnu \otimes \nabla \Glap
  - \nabla^\bot \otimes \nabla^\bot \partial_\nu \Gbh
  +\nabla \otimes \nabla^\bot \partial_\tau \Gbh \; .
  \nonumber
\end{equation}

\subsubsection{Layer potentials}

We now use the Stokeslet and stresslet definitions
of the previous section to define the standard single
and double layer potentials of the Stokes eigenvalue
problem. For $\xx \in \R^2$, the single layer potential
is defined to be

\begin{equation} \label{eq:singlelayer}
  \bS [\bmu] (\xx) = \int_\Gamma \GG (\xx,\yy) \bmu(\yy)
  \, dS(\yy) \; .
\end{equation}
We use the notation $\bsigma_\bS[\bmu]$ to denote the
stress tensor of the single layer at any given point
$\xx \in \R^2 \setminus \Gamma$.

For $\xx \in \R^2 \setminus \Gamma$, the double layer
potential is defined to be

\begin{equation} \label{eq:doublelayer}
  \bD [\bmu] (\xx) = \int_\Gamma \left ( \TT_{\cdot,\cdot,k}(\xx,\yy)
  \nu_k(\yy)\right )^\intercal \bmu(\yy) \, dS(\yy) \; ,
\end{equation}
where $\bnu$ denotes the outward unit normal to the boundary.
Note that when $\xx \in \Gamma$ we obtain the same formula
for the double layer potential except that the integral should be
interpreted in the principal value sense. 
If we write $\bmu = \bnu \mu_\nu + \btau \mu_\tau$,
where $\btau = \bmu^\bot$ is the positively oriented unit
tangent to the curve, then we have

\begin{align} \label{eq:stokesdlkernel}
  \left ( \TT_{\cdot,\cdot,k}(\xx,\yy)\nu_k(\yy) \right )^\intercal
  \bmu(\yy) &= \left ( - \nabla \Glap(\xx,\yy) + 2 \nabla^\bot
  \partial_{\nu\tau} \Gbh(\xx,\yy) \right ) \mu_\nu(\yy) \nonumber \\
  & \qquad+
  \nabla^\bot \left (\partial_{\tau\tau}-\partial_{\nu\nu} \right )
  \Gbh(\xx,\yy) \mu_\tau(\yy) \; .
\end{align}

When we view the layer potentials as
operators which map densities on the boundary to functions
on the boundary, we use special notation. Let

\begin{equation}
  \cS [\bmu] (\xx) = \int_\Gamma \GG (\xx,\yy) \bmu(\yy)
  \, dS(\yy)
\end{equation}
and
\begin{equation}
  \cD [\bmu] (\xx) = \oint_\Gamma \left ( \TT_{\cdot,\cdot,k}\nu_k(\yy)
  \right )^\intercal \bmu(\yy) \, dS(\yy) \; ,
\end{equation}
where the $\oint$ symbol means that the integral is to be
evaluated in the principal value sense. For two vector valued
functions $\ff$ and $\bg$ defined on $\Gamma$, consider the bilinear
form

\begin{equation}
  \langle \ff , \bg \rangle = \int_\Gamma \ff \cdot \bg dS \; .
\end{equation}
The adjoint of $\cD$ with respect to this bilinear form
is of particular interest; we have $\cD^\intercal = \cN$, where

\begin{equation}
  \cN [\bmu] = \oint_\Gamma \left ( \TT_{\cdot,\cdot,k}\nu_k(\xx)
  \right ) \bmu(\yy) \, dS(\yy) \; .
\end{equation}

\subsubsection{Jump conditions for the layer potentials}

In formulating an integral equation, it is important
to understand the behavior of the layer potentials
defined in the previous section for points near and on the
boundary. The so-called jump conditions for the layer
potentials are summarized in the following theorem.


\begin{lemma}[Jump conditions]
  \label{lemma:jump-conds}
  Let $\bS$, $\bsigma_\bS$, $\bD$, $\cD$, and $\cN$ be
  the layer potentials and operators defined above and
  let $\Gamma$ be a sufficiently smooth domain boundary
  with outward pointing normal $\bnu$. Then, for
  a given density $\bmu$ defined on $\Gamma$,
  we have that $\bS \bmu$
  is continuous across $\Gamma$, the exterior and interior
  limits of the surface traction of $\bD \bmu$ are equal,
  and for each $\xx_0 \in \Gamma$,

  \begin{align}
    \lim_{h \downarrow 0} \bsigma_\bS[\bmu](\xx_0 \pm h\bnu(\xx_0)) \cdot \bnu(\xx_0)
    &= \mp \frac{1}{2} \bmu(\xx_0) + \cN[\bmu](\xx_0) \\
    \lim_{h \downarrow 0} \bD[\bmu](\xx_0 \pm h\bnu(\xx_0)) 
    &= \pm \frac{1}{2} \bmu(\xx_0) + \cD[\bmu](\xx_0)    \; .
  \end{align}
\end{lemma}

The above expressions are derived by noting that the
leading order singularity of these integral kernels
is the same as for the original Stokes case, so that
the standard jump conditions for Stokes
\cite{KimSangtae1991M:pa,Pozrikidis1992}
apply. 
