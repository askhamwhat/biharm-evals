\section{Mathematical Preliminaries}
\label{sec:prelim}
In this paper, vector-valued
quantities are denoted by bold, lower-case letters
(e.g. $\bh$), while tensor-valued quantities are bold
and upper-case (e.g. $\mathbf{T}$). 
Subscript indices of non-bold characters (e.g. $h_j$ or $T_{ij\ell}$)
are used to denote the entries within a vector ($\bh$) or tensor ($\bT$).
We use the standard Einstein summation convention; in other words, 
there is an implied sum taken over the repeated indices of 
any term (e.g. the symbol $a_{j} b_{j}$ is used to represent the sum
$\sum_{j} a_{j} b_{j}$).
If $\xx = (x_1,x_2)^\intercal$, then $\xx^\bot = (-x_2,x_1)^\intercal$.
Similarly, $\nabla^\bot = (-\partial_{x_2},\partial_{x_1})^\intercal$.
Upper-case script characters (e.g. $\mathcal{K}$) are reserved for
operators on Banach spaces, with $\mathcal{I}$ denoting the
identity. Given a set $X$, we denote the closure of $X$
by $\overline{X}$.

For a given velocity field $\bu$ and pressure $p$, let $\bsigma(\bu,p)$
denote the Cauchy stress tensor, i.e.
\begin{equation}
\bsigma(\uu,p) = -p \II + 2 \be(\bu) \, ,
\end{equation}
where $\be(\bu)$ is the strain tensor given by
\begin{equation}
e_{ij}(\bu) = \frac{1}{2} \left( \partial_{x_j} u_i + \partial_{x_i} u_j \right) \; .
\end{equation}
When it is clear from context, we will drop the dependence of
$\bsigma$ on $\bu$ and $p$.
If $\Gamma$ is the boundary of a region $\Omega$ and $\bnu$ is the outward
normal to $\Gamma$, the surface traction $\bt$ on $\Gamma$ 
is the Neumann data, i.e. 
\begin{equation}
\bt = \bsigma \cdot \bnu \, .
\end{equation}

We seek solutions of \cref{eq:ostokes} in the space
\begin{equation}
  A(\Omega) = \{ (\bu,p) \textrm{ s.t. } \bu \in
  \left ( C^2(\Omega)\times C^2(\Omega) \right ) \cap
  \left ( C(\bar{\Omega}) \times C(\bar{\Omega}) \right ) \, , \,
  p \in C^1(\Omega) \times C(\bar{\Omega})\} \; ,
\end{equation}
where $\Omega$ is an open domain.

\subsection{Green's functions}

Let $\mathcal{L}_x$ denote a linear differential operator. A fundamental
solution $G(\xx,\yy)$ of $\mathcal{L}_x$ satisfies the equation
$\mathcal{L}_x G(\xx,\yy) = \delta_y(\xx)$ in the distributional sense, i.e.
for sufficiently smooth $f$
\begin{equation}
  \mathcal{L}_x \int_{\R^2} G(\xx,\yy) f(\yy) \, d \by = f(\xx) \; .
  \nonumber
\end{equation}
We consider here
free-space Green's functions, i.e. fundamental solutions which satisfy
radiation conditions as $|\xx-\yy| \to \infty$.
The Green's function of the oscillatory biharmonic equation,
\begin{equation}
  \Delta ( \Delta + k^2 ) u = 0 \; , \label{eq:obiharm} \nonumber
\end{equation}
is given by 
\begin{equation}
  \Gbh(\xx,\yy) = \frac{1}{k^2}
  \left (\frac{1}{2\pi} \log |\xx-\yy| +
  \frac{i}{4} H_0^{(1)}(k|\xx-\yy|) \right ) \, ,
  \label{eq:Gbh}
\end{equation}
where $k$ is the Helmholtz parameter in the oscillatory biharmonic equation,
and $H_{0}^{1}(r)$ is the Hankel function of the first kind of order zero.
Note that this is a scaled difference of the Green's function for
Laplace, i.e.

\begin{equation}
  \Glap(\xx,\yy) = \frac{1}{2\pi} \log |\xx-\yy| \; , \nonumber
\end{equation}
and the Green's function for the Helmholtz equation

\begin{equation}
  \Ghelm(\xx,\yy) = -\frac{i}{4} H_0^{(1)}(k|\xx-\yy|) \; . \nonumber
\end{equation}

\subsection{The Fredholm Alternative}

We require some standard results from the theory of
Fredholm integral equations. Interested readers may
consult \cite{reed1972methods,colton1983integral,kress1989linear},
among others, for the relevant background.

We first recall some definitions.
Let $X$ and $Y$ be Banach spaces with a non-degenerate
bilinear form $\langle \cdot ,\cdot \rangle: X\times Y \to \C$.
\begin{itemize}
\item Two operators $\cA:X\to X$ and $\cB:Y\to Y$ are
adjoint operators if
$\langle A \phi,\psi \rangle = \langle \phi, B\psi \rangle$
for every $\phi \in X$ and $\psi \in Y$.
\item For an operator $\cM:X\to X$, we can define the range
  $R(\cM)$ as the set $\{\phi \in X: \exists \phi_0 \textrm{ with }
  \cM\phi_0 = \phi\}$ and the null space $N(\cM)$ as the
  set $\{\phi \in X: \cM \phi = 0 \}$.
\item An operator $\cA$ is said to be compact if
  $\overline{\cA V}$ is a compact set for any
  bounded subset $V\subset X$.
\item Given a subspace $V\subset X$,
  we can define the subspace $V^\perp\subset Y$ as the
  set $V^\perp = \{ \psi \in Y: \langle \phi,\psi \rangle = 0
  \textrm{ for each } \phi \in V \}$, with the analogous
  definition for subspaces of $Y$.
\end{itemize}

Operators of the form $\cI-\cA$
have existence and uniqueness properties analogous to
matrices. This is known as the Fredholm Alternative;
we present the version provided in \cite{colton1983integral}.

\begin{thrm}[Fredholm Alternative \cite{colton1983integral}]
  Let $X$ and $Y$ be Banach spaces and
  $\langle \cdot,\cdot \rangle: X\times Y \to \C$ be
  a bilinear form. Suppose that $\cA: X\to X$ and
  $\cB:Y \to Y$ are compact adjoint operators. Then
  $\dim N(\cI-\cA) = \dim N(\cI-\cB) \in \N$,
  $R(\cI-\cA) = N(\cI-\cB)^{\perp}$, and
  $R(\cI-\cB) = N(\cI-\cA)^\perp$.
\end{thrm}

\subsection{Properties of the oscillatory Stokes layer
  potentials}

Recall that, in the case that $k=i\alpha$ for some real-valued $\alpha$,
the oscillatory Stokes equations \cref{eq:ostokes}
are known as the modified Stokes equations and are of particular
interest for their application to the analysis and numerical
simulation of unsteady flow
\cite{Pozrikidis1992,biros2002embedded,
  jiang2013second,ladyzhenskaya1969mathematical}.
The equations are well-studied in that setting and
integral representations which lead to second kind
integral equations have been developed. We review
some of the relevant results here, translating to
the oscillatory setting.

\subsubsection{Oscillatory Stokeslets and stresslets}
Consider the solution of
\cref{eq:ostokes} where a $\delta$-mass
centered at $\yy$ with strength $\ff$
has been added to the right-hand side of \cref{eq:ostokes}, i.e.

\begin{align}
  \nabla p - \Delta \uu - k^2 \uu &= \delta_\yy \ff \; ,
  \label{eq:ostokes_charge}  \\
  \nabla \cdot \uu &= 0 \; . \nonumber
\end{align}
Recall that

\begin{equation}
 \Delta \Glap(\xx,\yy) = \delta_\yy(\xx) \; . \label{eq:lapdelta}
\end{equation}
If we substitute \eqref{eq:lapdelta} into
\eqref{eq:ostokes_charge} and take the divergence,
we obtain

\begin{equation}
  p = \nabla \Glap(\xx,\yy) \cdot \ff \; . \nonumber
\end{equation}
We then have, formally,

\begin{align}
  \uu &= - (\Delta + k^2)^{-1} ( \Delta \Glap \ff
  - \nabla (\nabla \Glap \cdot \ff ) ) \nonumber \\
  &= \left ( -\Delta + \nabla \otimes \nabla \right )
  \Gbh \ff \; . \nonumber
\end{align}
The tensor

\begin{equation} \label{eq:ostokeslet}
  \GG = - \II \Delta \Gbh + \nabla \otimes \nabla \Gbh
\end{equation}
is then the analog of a Stokeslet
\cite{Pozrikidis1992} for \eqref{eq:ostokes}.

A related object is the stresslet, which is defined
in terms of the stress tensor of the velocity, pressure
pair induced by a Stokeslet. For these tensors, we find
that it is more convenient to express them in index notation
with the Einstein index summing convention.
Recall that the stress tensor $\bsigma$ is defined as 

\begin{equation}
  \sigma_{ij} = -p \delta_{ij} + \left ( \partial_{x_j}u_i
  +\partial_{x_i} u_j \right ) \; , \nonumber
\end{equation}
where $\delta_{ij}$ is the standard Kronecker delta notation.
The stresslet $\TT$ is defined to be

\begin{align}
  T_{ij\ell} &= - \partial_{x_j} \Glap \delta_{i\ell}
  + \partial_{x_\ell} \left ( -\Delta \Gbh \delta_{ij} +
  \partial_{x_i} \left(\partial_{x_j} \Gbh \right) \right)
  \nonumber \\
  & \qquad+ \partial_{x_i} \left ( -\Delta \Gbh \delta_{\ell j} +
  \partial_{x_\ell} \left(\partial_{x_j} \Gbh \right) \right)
  \; . \label{eq:ostress} 
\end{align}
Let $u_i = G_{ij} f_j$ and $p = \partial_{x_i} \Glap f_i$ be a
solution of the Stokes equations induced by a Stokeslet.
Then the corresponding stress tensor is given by
$\sigma_{i\ell} = T_{ij\ell} f_j$.

%For the layer potentials of the next section, the following
%formulas are useful. Let $\bnu$ be a given vector. When
%summing over the third index, we obtain
%
%\begin{equation}
%  \TT_{\cdot,\cdot,\ell} \nu_\ell = -\bnu \otimes \nabla \Glap
%  + \partial_\nu \left ( -\Delta \Gbh \II
%  + \nabla \otimes \nabla \Gbh \right)
%  + \nabla \otimes \left ( -\Delta \Gbh \bnu
%  + \partial_{\nu} \nabla \Gbh \right) \; . \nonumber
%\end{equation}
%Let $\btau = \bnu^\bot$. Then
%
%\begin{equation}
%  \TT_{\cdot,\cdot,k} \nu_k = -\bnu \otimes \nabla \Glap
%  - \nabla^\bot \otimes \nabla^\bot \partial_\nu \Gbh
%  +\nabla \otimes \nabla^\bot \partial_\tau \Gbh \; .
%  \nonumber
%\end{equation}

\subsubsection{Layer potentials}

We now use the Stokeslet and stresslet 
to define the single
and double layer potentials for the oscillatory Stokes problem.
For $\xx \in \R^2$, the single layer potential with density $\bmu$
is defined to be

\begin{equation} \label{eq:singlelayer}
  \bS [\bmu] (\xx) = \int_\Gamma \GG (\xx,\yy) \bmu(\yy)
  \, dS(\yy) \; .
\end{equation}
We use the notation $\bsigma_\bS[\bmu]$ to denote the
stress tensor of the single layer at any given point
$\xx \in \R^2 \setminus \Gamma$.

For $\xx \in \R^2 \setminus \Gamma$, the double layer
potential with density $\bmu$ is defined to be

\begin{equation} \label{eq:doublelayer}
  \bD [\bmu] (\xx) = \int_\Gamma \left ( \TT_{\cdot,\cdot,\ell}(\xx,\yy)
  \nu_\ell(\yy)\right )^\intercal \bmu(\yy) \, dS(\yy) \; ,
\end{equation}
where $\bnu$ denotes the outward unit normal to the boundary.
If we write $\bmu = \bnu \mu_\nu + \btau \mu_\tau$,
where $\btau = \bnu^\bot$ is the positively oriented unit
tangent to the curve, then we have

\begin{align} \label{eq:stokesdlkernel}
  \left ( \TT_{\cdot,\cdot,\ell}(\xx,\yy)\nu_\ell(\yy) \right )^\intercal
  \bmu(\yy) &= \left ( - \nabla \Glap(\xx,\yy) + 2 \nabla^\bot
  \partial_{\nu\tau} \Gbh(\xx,\yy) \right ) \mu_\nu(\yy) \nonumber \\
  & \qquad+
  \nabla^\bot \left (\partial_{\tau\tau}-\partial_{\nu\nu} \right )
  \Gbh(\xx,\yy) \mu_\tau(\yy) \; .
\end{align}

Let $\cS[\bmu]: C(\Gamma) \to C(\Gamma)$, and
$\cD[\bmu]: C(\Gamma) \to C(\Gamma)$ 
denote the restrictions of the layer potentials 
$\bS[\bmu]$ and $\bD[\bmu]$ on the boundary $\Gamma$, i.e.
for $\bx \in \Gamma$, 

\begin{equation}
  \cS [\bmu] (\xx) = \int_\Gamma \GG (\xx,\yy) \bmu(\yy)
  \, dS(\yy)
\end{equation}
and
\begin{equation}
\label{eq:dlformula}
  \cD [\bmu] (\xx) = \pv \int_\Gamma \left ( \TT_{\cdot,\cdot,\ell}(\bx,\by)
  \nu_\ell(\yy)
  \right )^\intercal \bmu(\yy) \, dS(\yy) \; ,
\end{equation}
where the \pv indicates that the integral is to be
evaluated in the principal value sense. 

For two vector valued
functions $\ff$ and $\bg$ defined on $\Gamma$, consider the bilinear
form
\begin{equation} \label{eq:bi_form}
  \langle \ff , \bg \rangle = \int_\Gamma \ff \cdot \bg dS \; .
\end{equation}
The definition of the adjoint used throughout the paper will be
the one induced by this form.


The adjoint of $\cD$ with respect to the above bilinear form
is of particular interest; and is given by
\begin{equation}
  \cD^{\intercal} [\bmu](\bx) = 
  \pv \int_\Gamma \left ( \TT_{\cdot,\cdot,\ell}(\bx,\by)\nu_\ell(\xx)
  \right ) \bmu(\yy) \, dS(\yy) \; .
\end{equation}


In the following lemma, we review the limiting values of
the layer potentials $\bS$ and $\bD$ on the boundary $\Gamma$.

\begin{lem}[Jump conditions] \label{lem:jump-conds}
  Suppose that $\Omega$ is a bounded region with a $C^{2}$ boundary
  $\Gamma$.
  Let $\bnu(\bx)$ denote the outward pointing normal at $\bx \in \Gamma$.
  Suppose that $\bmu \in C(\Gamma)$.
  Then $\bS[\bmu]$
  is continuous across $\Gamma$, and the exterior and interior
  limits of the surface traction of $\bD[\bmu]$ are equal.
  Furthermore, for $\bx_{0} \in \Gamma$,

  \begin{align}
    \lim_{h \downarrow 0^{+}} \bsigma_\bS[\bmu](\xx_0 \pm h\bnu(\xx_0)) \cdot \bnu(\xx_0)
    &= \mp \frac{1}{2} \bmu(\xx_0) + \cDt[\bmu](\xx_0) \\
    \lim_{h \downarrow 0^{+}} \bD[\bmu](\xx_0 \pm h\bnu(\xx_0)) 
    &= \pm \frac{1}{2} \bmu(\xx_0) + \cD[\bmu](\xx_0)    \; .
  \end{align}
\end{lem}

The above expressions are derived by noting that the
leading order singularity of these integral kernels
is the same as for the original Stokes case, so that
the standard jump conditions for Stokes
\cite{KimSangtae1991,Pozrikidis1992}
apply. 

\begin{lem} \label{lem:compact-sd}
  Suppose that $\Omega$ is a bounded region with a $C^{2}$ boundary
  $\Gamma$. Then the operators $\cS$ and $\cD$ defined
  above are compact operators on $C(\Gamma)\times C(\Gamma)$
  and $\mathbb{L}^2(\Gamma)\times \mathbb{L}^2(\Gamma)$.
\end{lem}

Compactness is proved by considering the
asymptotic expansion of each kernel about
$\xx=\yy$ and noting that each is at most
weakly singular.

\subsubsection{Representation Theorem}

In the following theorem, we sketch the proof of the equivalent of the
Green's identity for oscillatory Stokes setting, which is well-known
in the Stokes and modified Stokes settings
\cite{Pozrikidis1992,biros2002embedded,ladyzhenskaya1969mathematical}.

\begin{thrm} \label{thrm:rep-theorem}
  Let $\Omega$ be a bounded domain with $C^2$ boundary and let
  the pair $(\bu,p)$ satisfy the oscillatory Stokes equations
  \cref{eq:ostokes} in $\Omega$. Let $\bt$ denote the surface
  traction associated with $(\bu,p)$. Then

  \begin{equation} \label{eq:rep-theorem}
    \bS [\bt](\xx) - \bD[\bu](\xx) = \begin{cases} 
    \bu(\xx) &\quad \xx \in \Omega \,  \\
    0 &\quad \xx \in E 
    \end{cases} \; ,
  \end{equation}
  where $E=\R^2\setminus\bar\Omega$ is the
  exterior of the domain.
\end{thrm}

\begin{proof}
  Suppose that $\xx \in \Omega$.
  By the definitions of $\GG$ and $\Glap$, we have
  \begin{equation*}
    \bu(\xx) = \int_\Omega -(\Delta_x + k^2) \GG(\xx,\yy) \bu(\yy)
    + \nabla \otimes \nabla \Glap(\xx,\yy) \bu(\yy) \, dV(\yy) \; .
  \end{equation*}
  Applying Green's identity and the divergence theorem, we
  obtain
  \begin{equation}
    \bu(\xx) = \int_\Gamma \GG(\xx,\yy) \partial_\nu \bu
    - \partial_\nu \GG(\xx,\yy) \bu
    - \nabla \Glap(\xx,\yy) (\bnu \cdot \bu) \, dS 
    - \int_\Omega \GG(\xx,\yy) (\Delta + k^2) u \, dV \; .
    \nonumber
  \end{equation}
  Substituting the definition of the PDE and applying the divergence
  theorem again, we obtain
  \begin{equation}
    \bu(\xx) = \int_\Gamma \GG \partial_\nu \bu - p \GG \bnu - \partial_\nu \GG \bu
    - \nabla \Glap (\bnu \cdot \bu) \, dS  \; . \label{eq:rep_proof_1}
  \end{equation}
  From the divergence theorem and the divergence-free properties of
  $\uu$ and $\GG$, we then get

  \begin{equation}
    \int_\Gamma \GG \nabla (\bu \cdot \bnu)
    - (\nabla \GG \bnu)^\intercal \bu \, dS = 0 \; .  \label{eq:rep_proof_2}
  \end{equation}
  Adding \cref{eq:rep_proof_1,eq:rep_proof_2}, we get the desired
  result. The argument for the case $\xx \in E$ is similar.
\end{proof}

A consequence of \cref{thrm:rep-theorem} and the analyticity
of $\Gbh$ is 
\begin{cor}
\label{cor:analytic}  
  Let $(\uu,p) \in A(\Omega)$ be a solution of \cref{eq:ostokes}.
  Then each component of $\uu(\xx)$ is an analytic function
  of the coordinates $\xx$ in $\Omega$.
\end{cor}

The proof of \cref{cor:analytic} follows the same reasoning as
that for the Helmholtz case; see \cite[Theorem 3.5]{colton1983integral}.


\subsubsection{Null-space correction \label{subsubsec:nullspacecorr}}

Without modification, the standard layer potentials
can result in rank-deficient representations for
the boundary value problems. The nature of this deficiency
is treated below but for now we introduce a standard
operator used to correct this. For any integrable density
$\bmu$, let $\cW[\bmu]$ be defined by

\begin{equation} \label{eq:ones_operator}
  \cW[\bmu](\xx) = \frac{1}{|\Gamma|} \int_\Gamma \bnu(\xx)
  \left ( \bnu(\yy) \cdot \bmu(\yy) \right )
  \, dS(\yy) \; ,
\end{equation}
for any $\xx \in \Gamma$. We have

\begin{lem}
  \label{lem:propnullspacecorr}

  Let $\Omega$ be a domain with $C^2$ boundary and $\bmu$ be
  an integrable function defined on $\Gamma$. Then
  \begin{itemize}
  \item $\cW[\cW[\bmu]] = \cW[\bmu]$,
  \item $\cW^\intercal = \cW$,
  \item $\cW[\bmu - 2 \cD[\bmu]] = 0$,
  \item $\cW[\cS[\bmu]] = 0$,
  \end{itemize}
  where the transpose is induced by the bilinear
  form \cref{eq:bi_form}.
\end{lem}

\begin{proof}
  The first two results follow from the definitions of $\cW$ and
  the normal vector. The other two follow from the fact that
  $\bS[\bmu]$ and $\bD[\bmu]$ are divergence-free and
  an application of \cref{lem:jump-conds}.
\end{proof}

\begin{remark}
The operators $\bS,\bD,\cD$ and $\cD^{\intercal}$
depend on the Helmholtz parameter $k$ of the oscillatory Stokes equation.
In places where it is essential to highlight this dependence, in a slight
abuse of notation, we will use the symbols $\bS_{k}, \bD_{k},\cD_{k}$ and $\cDt_{k}$ to 
denote this dependence.
Similarly, we will use $A^{\Gamma}$ instead of the operator $A$ to highlight 
the dependence of the operator $A$ on the boundary of the region $\Gamma$.
\end{remark}
