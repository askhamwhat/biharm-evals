
\section{Integral equation derivation} \label{sec:anapp}

%\subsection{Adapting the completed double layer representation
%to solve the Dirichlet problem}
We would like to adapt the completed double layer representation
for solutions of Stokes flow \cref{eq:biharmV1,eq:biharmV2,eq:biharmV3} to solve 
the clamped plate problem \cref{eq:biharmD1,eq:biharmD2,eq:biharmD3}.
Let $f$ and $g$ be the boundary data as in \cref{eq:biharmD1,eq:biharmD2,eq:biharmD3}. 
By computing tangential derivatives of $f$ on each
boundary component, we get the following related Helmholtz-Stokes problem:
\begin{align}
 \Delta (\Delta + \lambda^2) \tilde{w}  = 0 &\quad \bx \in D \, ,\nonumber\\
 \frac{\partial \tilde{w}}{\partial \tau} = 
 \frac{\partial f}{\partial \tau} 
&\quad \bx \in \Gamma \label{eq:biharm5} \, ,\\ 
 \frac{\partial \tilde{w}}{\partial n} = g &\quad \bx 
\in \Gamma 
\, . 
\nonumber
\end{align}
There are two main issues to be addressed in using the completed
double layer representation in this context. First,
the representation is designed for 
Helmholtz-Stokes flow, in which the quantities of interest are derivatives
of the potential $\tilde{w}$ and not $\tilde{w}$ itself; the
representation for $\tilde{w}$ may not be single-valued. 
We will establish that, in the context of \cref{eq:biharm5},
the stream function is necessarily single-valued. We also discuss
some numerical issues related to evaluating the stream function.
The second issue to address
is that the solution $\tilde{w}$ only satisfies the original boundary 
condition for the value of $\tilde{w}$ up to a constant on 
each boundary component.
In fact, for multiply connected domains, the completed double layer 
representation is incomplete for the Dirichlet
problem \cref{eq:biharmD1,eq:biharmD2,eq:biharmD3}. We present a remedy for this 
issue and provide some physical intuition.
%
%\subsection{Single-valued stream functions}
%To solve the Dirichlet problem \cref{eq:biharmD1,eq:biharmD2,eq:biharmD3},
%it is necessary to compute
%a single-valued biharmonic potential.
%In the case of a multiply connected domain,
%there is no guarantee that a single-valued stream 
%function exists for a given velocity field.
%
%Consider the following example.
%Let $(r,\theta)$ denote standard polar coordinates. It is
%easy to verify that the velocity field 
%$\bu=\frac{1}{r}\hat{r}$ 
%solves the equations of Stokes flow in an annulus centered at the 
%origin. A stream function for this flow is $w=\theta$, which 
%is not single-valued; indeed, there are no single-valued
%stream functions which generate this flow. 
%
%Let $D$ be a multiply connected domain with boundary 
%$\Gamma = \cup_{i=0}^N \Gamma_i$, as in the previous section.
%We note that the gradient of a stream function is determined 
%by the velocity field, i.e. 
%
%\begin{equation}
%\nabla w = -\bu^\perp := \begin{pmatrix} 
%  - u_2 \\ u_1 
%\end{pmatrix} \; .
%\end{equation}
%Therefore, a velocity field has single-valued stream
%functions if and only if $\bu^\perp$ is conservative.
%Using standard results from multivariable calculus, 
%we can  characterize such flows.
%
%\begin{proposition} 
%Suppose that $\bu$ is a divergence-free velocity 
%field which is $C^1$ on $D$ and continuous on 
%$D\cup \Gamma$. The field $\bu^\perp$ 
%is conservative if and only if 
%
%\begin{equation}
% \int_{\Gamma_i} \bu\cdot \bn \, dS 
%= 0 \quad i=0,1,\ldots N \, .
%\label{eq:NecSuffStreamFuncExistence}
%\end{equation}
%
%\end{proposition}
%
%The equalities \cref{eq:NecSuffStreamFuncExistence}
%constitute $N$ linearly independent constraints on the 
%boundary data because the divergence-free condition 
%\cref{eq:MassConservation} implies that
%$\int_{\Gamma} \bu \cdot \bn \, dS = 0$. 
%It turns out that 
%these conditions are satisfied when the Dirichlet 
%problem is recast as a Stokes flow \cref{eq:biharm5}, as 
%it is easily verified that
%\begin{equation}
% \int_{\Gamma_i} \bu\cdot \bn \, dS 
%= \int_{\Gamma_i} \frac{\partial f}{\partial \tau} \, dS =0 \, .
%\end{equation}
%Thus, any stream function $\tilde{w}$ obtained for the Stokes
%flow \cref{eq:biharm5} is necessarily single-valued.
%% NOT GOING THIS WAY ANYMORE
%%It can be shown using the Helmholtz Hodge decomposition that the 
%%conditions \cref{eq:NecSuffStreamFuncExistence} are sufficient for 
%%the existence of a single-valued stream function. 
%
%\subsection{Evaluating the stream function} \label{subsec:stream}
%
%Given compatible boundary data for the velocity field $\bu$, 
%the completed double layer representation for Stokes flow 
%\cref{eq:IntRepStokes} guarantees the existence of
%a solution density $\boldsymbol\mu$ 
%and a corresponding stream function $\tilde{w}$. The Goursat 
%function formula 
%for $\tilde{w}$, see \cref{sec:stokeslayer},
%is necessarily single-valued, as explained above,
%but it is not immediately obvious from the formula 
%that this should be true.
%
%The difficulty in the representation of $\tilde{w}$ 
%comes from the part
%of the stream function corresponding to the double layer potential
% \cref{eq:wDL}. The 
%second term in the expression for the double layer potential is
%\begin{equation}
% v_1(z) = \mbox{Re}\left[\frac{1}{4\pi i}\int_{\Gamma}\left(
%\overline{\rho\left(\xi\right)}d\xi+\rho\left(\xi\right)
%\overline{d\xi}\right)\log\left(\xi-z\right)\right] \, .
%\end{equation}
%To compute this term, in a na\"{i}ve numerical implementation,
%the question of which is the appropriate branch of the 
%logarithm to use would arise at many steps.
%To avoid this complication, it is possible instead to compute
%$v_1$, up to a constant, as the harmonic conjugate of the
%function 
%\begin{equation}
% v_2 = \frac{1}{4\pi}\int_{\Gamma}\left(\overline{\rho\left(\xi\right)}
%d\xi+\rho\left(\xi\right)\overline{d\xi}\right)\log
%\left(\left|\xi-z\right|\right) \, . \label{eq:harmconjg}
%\end{equation}
%
%We will use this approach to evalute $v_1$ numerically.
%As a result of the Cauchy-Riemann equations, the 
%harmonic conjugate of $v_2$, satisfies the following
%Neumann problem for the Laplace equation:
%\begin{align}
% \Delta v_1 &= 0 &\quad x\in D \, ,\\
% \frac{\partial v_1}{\partial n} &= -\frac{\partial v_2}{\partial \tau} 
%&\quad x\in\Gamma \, .
%\end{align}
%It is possible then to use standard integral equation
%methods to compute $v_1$. 
%
%Let $v_1 = S^L_\Gamma \sigma$, where $S^L_\Gamma \sigma$ 
%is the single layer potential for Laplace's equation, given by
%\begin{equation}
%S^L_\Gamma\sigma (\bx) = -\frac{1}{2\pi} \int_{\Gamma} \log 
%\left|\bx - \by \right|\sigma(\by)\, dS \left(\by \right) \, ,
%\end{equation}
%where $\sigma\in \mathcal{X}= C^{0,\alpha}\left(\Gamma\right)$, for 
%some $\alpha \in (0,1)$, is an unknown density 
%(see \cite{kress1999linear, guenther1988partial}).
%Imposing the Neumann boundary conditions results in the 
%following boundary integral equation for $\sigma$:
%\begin{align}
%-\frac{\partial v_2}{\partial \tau} (\bx) &= \frac{1}{2} \sigma \left(\bx\right) - \frac{1}{2\pi}\oint_{\Gamma} 
%\frac{\partial}{\partial n_x}  
%\log \left| \bx - \by \right |\sigma(\by)\, dS \left(\by \right) 
%\, , \\
%-\frac{\partial v_2}{\partial \tau} &= \left( \frac{1}{2}I_{\mathcal{X}} + K^L_\Gamma  \right) \sigma 
%\, ,
%\label{eq:BlockSystemRow2P1tmp}
%\end{align}
%where the operator $K^L_\Gamma$ is compact, so that the integral equation
%is second kind.
%For a derivation of this result, see \cite{kress1999linear}.
%
%It is well known that the operator $\frac{1}{2}I_{\mathcal{X}} + K^L_\Gamma$ 
%has a one dimensional null space. Thus,
%we choose to solve the above integral equation subject to 
%the constraint $\int_{\Gamma} \sigma \, dS = 0$. 
%Furthermore, it is known that solving the Neumann problem subject 
%to the above constraint is equivalent
%to solving
%\begin{align}
%\left( \frac{1}{2}I_{\mathcal{X}} + K^L_\Gamma  + W_\Gamma \right) \sigma 
%= -\frac{\partial v_2}{\partial \tau}
%\label{eq:BlockSystemRow2P1} 
%\end{align}
%where $W_\Gamma\sigma = \int_{\Gamma} \sigma \, dS$. 
%To prove this, we need 
%the following property of 
%the Green's function for Laplace's equation.
%\begin{align}
%-\frac{1}{2\pi}\oint_{\gamma} \frac{\partial}{\partial n_{\by}}  
%\log \left|\bx - \by\right|\, dS \left(\by \right) 
%&=-\frac{1}{2}\quad\bx\in\gamma \, , \label{eq:DLP2}
%\end{align}
%Using the above property, it follows that
%\begin{align}
% \int_{\Gamma} K^L_\Gamma \sigma \, dS(\x) 
%= -\frac{1}{2}\int_{\Gamma} \sigma \left(\bx\right) \, dS(\x) 
%\label{eq:KLint}
%\end{align}
%Integrating equation \cref{eq:BlockSystemRow2P1} and using 
%equation \cref{eq:KLint}, we get
%\begin{align}
%\int_{\Gamma} \left( \frac{1}{2}I_{\mathcal{X}} + 
%K^L_\Gamma  + L \right) \sigma \, dS(\x) 
%&= -\int_{\Gamma} \frac{\partial v_2}{\partial \tau} \, dS(\x) \\
%\left|\Gamma\right| L\sigma  &= 0 
%\end{align}
%which proves the result.

\subsection{Making the representation complete}
As mentioned above, the solution $\tilde{w}$ of 
the auxiliary Helmholtz-Stokes problem \cref{eq:biharm5}
only satisfies the boundary conditions of the 
original Dirichlet problem \cref{eq:biharmD1,eq:biharmD2,eq:biharmD3}
up to a constant on each boundary component. For a simply
connected domain, this constant can be recovered from the
fact that adding an arbitrary constant to a stream function
does not change the velocity field. 
Thus, in simply connected domains, there is an equivalence in the 
solutions of \cref{eq:biharm5} and \cref{eq:biharmD1,eq:biharmD2,eq:biharmD3}.

To analyze the case of a multiply connected domain,
we first consider
radially symmetric solutions on an annulus centered at the
origin. Let $w\left(r\right)$ be a 
radially symmetric biharmonic-Stokes
potential. Then $w(r)$ solves the ordinary differential
equation (ODE)
\begin{equation}
\Delta_{r} (\Delta_{r} + k^2)w = 0 \, ,
\end{equation}
where $\Delta_{r}$ is the radial component of the Laplacian in polar
coordinates given by
$$
\Delta_{r} = \frac{1}{r}\frac{d}{dr} r \frac{d}{dr} \, .
$$
Four linearly independent solutions of this ODE are $1$, $\log{(r)}$, 
$J_{0}(kr)$, and $H_{0}(kr)$, 
where $J_{0}$ is the order zero Bessel function of the first kind, and $H_{0}$ is the order zero Hankel function of the first kind.
For each solution, we can compute the associated velocity field 
$\bu = \nabla^{\perp} w$. By construction, $\bu$ satisfies 
the continuity condition \cref{eq:MassConservation}. 
For the momentum equation \cref{eq:StokesFlowEq} to be 
satisfied, we need that $\Delta\bu + k^2 \bu$ is a conservative
vector field, which 
is equivalent to the condition that 
$\int_{\gamma} (\Delta\bu + k^2 \bu) \cdot d\bl = 0$ 
for any closed 
loop $\gamma$ in the annulus. For $\tilde{w}=1$, $J_{0}(kr)$, and $H_{0}(kr)$,
the corresponding velocity field satisfies
$\Delta \bu + k^2 \bu=0$ so that $\Delta \bu$
is trivially a conservative vector field. 
On the other hand, for $\tilde{w}=\log{(r)}$, the corresponding velocity field
satisfies $\Delta \bu + k^2 \bu=\frac{k^2}{r} \hat{\theta}$.
By considering a curve $\gamma$ encircling the origin,
we see that $\Delta \bu + k^2 \bu$ is not a conservative vector
field and that any pressure for the velocity field 
associated with $\log\left(r\right)$ is not single-valued.
%
%The above analysis
%can be extended to show that any solution of the biharmonic equation 
%of the form $\log{|r-r_{j}|}$ where $r_{j}$ is located in one of the holes of a multiply connected
%domain.
%cannot be represented as a Stokes velocity field. 
%In simply connected 
%domains, since there are no holes, corresponding to each solution of the modified biharmonic equation with clamped plate
%boundary conditions, there exists a solution to . For multiply connected domains with genus $N$, 
%the set of stream functions for Stokes velocity fields
%misses an $N$ dimensional space of solutions, corresponding 
%to biharmonic charges located in the holes of the domain.
%Following this reasoning, we obtain a complete representation
%for biharmonic potentials on multiply connected domains
%by adding $N$ charges, one per each hole of the domain,
%to the representation for $w$. The details of this approach, and
%the proof that it is sufficient, is in the next section.
