
\appendix

\section{Proof of the invertibility of the Farkas representation}
In this appendix, we prove~\cref{lem:farkinv} which is restated here
as~\cref{lem:farkinv_app}.

In the following lemma, we present an integral indentity involving the
Green's function of the oscillatory biharmonic equation.
\begin{lem}
Suppose that $\Gbh$ is the Green's function of the oscillatory biharmonic equation 
with Helmholtz parameter $k$. Suppose that $v:\R^{2} \to \R$ is a compactly supported smooth
function. Then
\begin{equation}
\int_{\R^{2}} \Gbh(\xx,\yy) \Delta (\Delta + k^2) v(\yy) d\yy = v(\xx)
\end{equation}
\end{lem}
\begin{proof}
Suppose that $r>0$. 
Let $B_{r}(\xx)$ denote the ball of radius $r$ centered at the origin
and let $\partial B_{r}(\xx)$ denote its boundary.
Then applying Green's second identity twice, we get 
\begin{align}
\int_{\R^{2} \setminus B_{r}(\xx)} \Gbh(\xx,\yy) \Delta (\Delta +k^{2}) v(\yy)
&=
\int_{\R^{2} \setminus B_{r}(\xx)} \Delta \Gbh (\xx,\yy) (\Delta + k^2) 
v(\yy) + \\
& \hspace{5ex}\int_{\partial B_{r}(\xx)} \Gbh(\xx,\yy) (\Delta + k^{2})
\frac{\partial}{\partial n}v(\yy) \, dS - \\
& \hspace{5ex}\int_{\partial B_{r}(\xx)} \frac{\partial }{\partial n} \Gbh(\xx,\yy) 
(\Delta + k^2) v \, dS \\
&= \int_{\R^{2} \setminus B_{r}(\xx)} \Delta (\Delta + k^2)G(\xx,\yy) v(\yy)
d\yy  - \nonumber \\
& \hspace{5ex}\int_{\partial B_{r}(\xx)} (\Delta + k^2) 
\frac{\partial }{\partial n} \Gbh(\xx,\yy) v \, dS + \nonumber\\
& \hspace{5ex}\int_{\partial B_{r}(\xx)} (\Delta + k^2) 
\Gbh(\xx,\yy) \frac{\partial}{\partial n}v \, dS - \nonumber \\
& \hspace{5ex}\int_{\partial B_{r}(\xx)}  
\frac{\partial }{\partial n} \Gbh(\xx,\yy) \Delta v\, dS + \\
& \hspace{5ex}
\int_{\partial B_{r}(\xx)} \Gbh(\xx,\yy) \Delta \frac{\partial v}{\partial n}
\, dS \nonumber
\end{align}
For all $\yy \in \partial B_{r}(x)$, there exists a constant $C$ such that
$\Gbh$ satisfies the following equations
\begin{align}
(\Delta + k^2) \frac{\partial }{\partial n} G(\xx,\yy) &= 
-\frac{1}{2\pi r} \label{eq:estapp1}\\
\left|(\Delta + k^2) G(\xx,\yy) \right| &\leq C \log{(r)} \label{eq:estapp2}\\
\left|\frac{\partial }{\partial n} G(\xx,\yy) \right| &\leq C r\log{(r)} 
\label{eq:estapp3}\\
\left| G(\xx,\yy) \right| &\leq C r^{2} \log{(r)} \label{eq:estapp4}\, .
\end{align}
Furthermore, $\Gbh$ satisfies $\Delta (\Delta + k^2) \Gbh(\xx,\yy) = 0$
for all $\yy \in \R^{2} \setminus B_{r}(\xx)$.
Using the estimates in~\cref{eq:estapp1,eq:estapp2,eq:estapp3,eq:estapp4}, 
and the smoothness of $v$, we get
\begin{equation}
\int_{\R^{2} \setminus B_{r}(\xx)} \Gbh(\xx,\yy) \Delta (\Delta + k^2) v(\yy) 
d \yy = \frac{1}{2\pi r} \int_{\partial B_{r}(\xx)} v(\yy) dS + 
O(r \log(r)) \, \label{eq:estapp5}.
\end{equation}
The result then follows by taking the limit $r\to 0$ in~\cref{eq:estapp5}. 
\end{proof}
\label{sec:farkproof}

\section{Proof of invertibility of new representation}
The proof proceeds as follows:
\begin{itemize}
\item Show uniqueness of solutions to the impedence, velocity, and surface
traction boundary value problem in exterior domains for the oscillatory Stokes
PDEs. In order to prove this, we need the following lemmas:
\begin{itemize}
\item Deriving Green's theorem for exterior domains
\item Rellich's lemma
\item Uniqueness proof
\end{itemize}
\item The proof for the clamped plate problem proceeds in the manner
analogous to the Dirichlet biharmonic paper.
\end{itemize}

With this in mind, we first discuss the PDE theory for the impedence, 
velocity and surface traction boundary value problems on unbounded
domains for the oscillatory Stokes equation.
We focus on the proof for the velocity boundary value problem. 
The proofs for the surface traction and impedence problems is similar.

Suppose that $\Omega$ is a bounded simply connected domain and that
$E = \R^{2} \setminus \Omega$ is its exterior.
Let $\Gamma = \partial \Omega$ denote the boundary of $E$.
For a given function $\bh$ on the boundary, 
the exterior velocity boundary value problem is given by:
\begin{align}
\Delta \bu + \omega^{2} \bu &= \nabla p \quad \bx \in E \, ,\\
\nabla \cdot \bu &= 0 \quad \bx \in E \, ,\\
\bu &= \bh \quad \bx \in \Gamma \, . \\
\end{align}
Let $\sigma$ denote the stress tensor associated with the velocity
field $\bu$, i.e.
\begin{equation}
\sigma = -p I + e(\bu) \, ,
\end{equation}
where $e(\bu)$ is the strain tensor given by
\begin{equation}
e(\bu) = \frac{1}{2}\left(D\bu + D\bu^{T} \right) \, .
\end{equation}
Let $\bn$ denote the outward normal to the boundary $\Gamma$, then
the surface traction $\ff$ on the boundary $\Gamma$ is given 
by
\begin{equation}
\ff = \sigma \cdot \bn
\end{equation}
Analogous to the Helmholtz equation, we need to impose radiation conditions
at $\infty$.
We propose the following radiation conditions for the oscillatory Stokes 
equation
\begin{equation}
\lim_{r\to \infty} \sqrt{r} \left| \ff - i \omega \bu \right| \to 0 \, ,
\label{eq:radcond}
\end{equation}
uniformly in direction.

\begin{lem}
\label{lem:rep}
Suppose that $\bu$ satisfies the oscillatory Stokes equation in 
an unbounded region $E$ along with the radiaition 
condition~\cref{eq:radcond}. 
Then 
\begin{equation}
\label{eq:repinfest}
\lim_{r\to\infty}
\int_{|\by|=r} \left( |\ff|^2 + |\omega|^2 |\bu|^2 \right) ds +
2 \text{Im}(\omega) \int_{E \cap B_{r}(0)} \left(|\omega|^2 |\bu|^2 + |e(\bu)|^2 \right)
dV = -2 \text{Im} \left( \omega \int_{\Gamma} \bu \cdot \overline{\ff} ds  \right)
\end{equation}
\end{lem}

\begin{proof}
Since $\bu$ satisfies the radiation condition, we have that
\begin{equation}
\lim_{r\to\infty} \int_{|\by|=r} | \ff - i \omega \bu|^2 ds = 
\lim_{r\to\infty} \int_{|\by| =r} \left( |\ff|^2 + |\omega|^2|\bu|^2 + 2 \text{Im} 
\left( \omega \bu\cdot \ff \right) ds \label{eq:raddecayproof1}
\right) = 0 \, . 
\end{equation}
Since $\bu$ satisfies the oscillatory Stokes equation $E \cap B_{r}(0)$,
using a couple of applications of the divergence theorem, we have that
\begin{align}
\int_{E\cap B_{r}(0)} |e(\bu)|^2 dV &= \int_{E\cap B_{r}(0)} \left< e(\bu), \overline{e(\bu)} \right> dV  \\
&= \frac{1}{2}\left(\int_{E\cap B_{r}(0)} \left< e(\bu),\overline{D\bu} \right> + \left< e(\bu),\overline{D\bu}^{T} \right> \right) dV \\
&= \int_{E\cap B_{r}(0)} \left< e(\bu),\overline{D\bu} \right> dV \\
&= \int_{\partial }
-\int_{\Gamma} \bu \cdot \overline{\ff} ds
+ \int_{|\by|=r} \bu \cdot \overline{\ff} ds + \overline{\omega}^2 
\int_{E \cap B_{r}(0)} |\bu|^2 dV \,. \label{eq:raddecayproof2}
\end{equation}
Combining~\cref{eq:raddecayproof1,eq:raddecayproof2}, we get
\begin{equation}
\lim_{r\to\infty} \int_{|\by|=r}\left(|\ff|^2 + |\omega|^2 |\bu|^2 \right) ds 
+ 2\text{Im}(k)\int_{E \cap B_{r}(0)} \left(|e(\bu)|^2 + |\omega|^2 |\bu|^2 
\right) dV = -2\text{Im} \int_{\Gamma} \omega \bu \cdot \overline{\ff} dS \, .
\end{equation}
\end{proof}

In the next lemma, we prove the analogue of Rellich's lemma
for the oscillatory Stokes equation. 
\begin{lem}
\label{lem:rellich}
Suppose that $\bu$ satisfies the oscillatory Stokes equation in an unbounded
region $E$.  
Suppose further that 
\begin{equation}
\lim_{r \to \infty} \int_{|\by|=r} |\bu|^2 dS = 0 
\, . \label{eq:decayatinf}
\end{equation}
Then each component of $\bu$ is harmonic and has a convergent Laurent
expansion in $E$.
\end{lem}
\begin{proof}
We first note that each component of $\bu= (u_{1},u_{2})$ satisfies the 
oscillatory biharmonic equation in $E$, i.e.
\begin{equation}
\Delta (\Delta + \omega^2) u_{j} = 0 \quad j=1,2 \,. 
\end{equation}
For $r$ sufficiently large, we can express $u_{j}$ in the Fourier basis as
\begin{equation}
u_{j}(r,\theta) = \sum_{n=-\infty}^{\infty} a_{j,n}(r) e^{i n \theta}  \quad 
j=1,2 \, .
\end{equation}
Using Parseval's identity then
\begin{equation}
\int_{|\by|=r} |\bu|^2 ds = r\sum_{n=-\infty}^{\infty} |a_{1,n}(r)|^2  +
|a_{2,n}(r)|^2 \, .
\end{equation}
Since $\bu$ satisfies~\cref{eq:decayatinf}, we conclude that
\begin{equation}
\lim_{r\to\infty} r|a_{j,n}(r)|^2 = 0 \quad j=1,2 \, , \label{eq:adecay}
\end{equation}
Since $u_{j}$, $j=1,2$ satisfies the oscillatory biharmonic equation,
the functions $a_{j,n}$ are linear combinations of 
\begin{equation}
r^{|n|}, r^{-|n|}, H^{1}_{n}(\omega r), H^{2}_{n}(\omega r) \, , \quad
n\neq 0 \, ,
\end{equation}
and
\begin{equation}
1, \log{(r)}, H^{1}_{0}(\omega r), H^{2}_{0}(\omega r) \quad n=0 \, , 
\end{equation} 
where $H_{n}^{1,2}(\cdot)$ are the Hankel functions of the first and
second kind of order $n$.
Since $a_{j,n}(r)$ satisfy~\cref{eq:adecay}, and using the asymptotic 
expansion of $H_{n}^{1,2}(r)$, we note that the projection of 
$a_{j,n}$ on $r^{|n|}$, and $H_{n}^{j}(\omega r)$ must be zero. 
Thus, 
\begin{equation}
u_{j}(r,\theta) = \sum_{n=-\infty}^{\infty} \frac{a_{j,n} e^{i n \theta}}{r^{|n|}} 
\, .
\end{equation}
\end{proof}
\begin{remark}
Note that each component of $\bu$ is harmonic and thus $\bu$ then satisfies
\begin{align}
\omega^2 \bu &= \nabla p  \quad \label{eq:massconsred} \\
\nabla \cdot \bu &= 0 \, .
\end{align}
Taking the inner product with $\nabla^{\perp}$ in~\cref{eq:massconsred}, 
we note that $\bu$ satisfies both $\nabla^{\perp} \cdot \bu = 0$ 
and $\nabla \cdot \bu = 0$.
This remark will be useful for the impedence problem...
\end{remark}
\begin{remark}
Note that in Rellich's lemma $\bu$ need not be a radiating solution. 
All we know of $\bu$ is that it satifies the PDE in $E$.
\end{remark}

\begin{thm}
(Uniqueness Dirichlet)
Suppose that $\text{Im}(\omega)\geq 0$ and 
that $\bu$ is a radiating solution to the oscillatory Stokes
equation in $E$ with $\bu =0$ on the boundary $\Gamma$, then
$u \equiv 0$ in $E$.
\end{thm}

\begin{proof}
Since $\bu = 0$ on $\Gamma$, it follows from~\cref{eq:repinfest} that
\begin{equation}
\lim_{r\to\infty}
\int_{|\by|=r} \left( |\ff|^2 + |\omega|^2 |\bu|^2 \right) ds +
2 \text{Im}(\omega) \int_{E \cap B_{r}(0)} \left(|\omega|^2 |\bu|^2 + |e(\bu)|^2 \right)
dV = 0
\end{equation} 
In particular, this implies that
\begin{equation}
\lim_{r\to\infty} \int_{|\by|=r} |\bu|^2 ds = 0 \, .
\end{equation}
Thus, the conditions for $\bu$ in Rellich's lemma (~\cref{lem:rellich})
are satisfied, and each component of $\bu$ is a harmonic function
with $\bu \to 0$ as $r \to \infty$. Furthermore, since $\bu=0$ on
$\Gamma$, from uniqueness of solutions to the Dirichlet problem
on exterior domains, we conclude that $\bu \equiv 0$ in $E$.
\end{proof}

The proof for the Neumann problem proceeds in a similar manner. 
