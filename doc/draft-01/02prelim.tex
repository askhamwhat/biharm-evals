\section{Mathematical Preliminaries}
\label{sec:prelim}
In this paper, vector- and tensor-valued quantities
are denoted by bold letters (e.g. $\bh$ and $\mathbf{T}$). 
Subscript indices of non-bold characters (e.g. $h_j$ or $T_{ij\ell}$)
are used to denote the entries within a vector or tensor.
We use the standard Einstein summation convention; i.e., 
there is an implied sum taken over the repeated indices of 
any term (e.g. the symbol $a_{j} b_{j}$ is used to represent the sum
$\sum_{j} a_{j} b_{j}$).
If $\xx = (x_1,x_2)^\intercal$, then $\xx^\bot = (-x_2,x_1)^\intercal$.
Similarly, $\nabla^\bot = (-\partial_{x_2},\partial_{x_1})^\intercal$.
Upper-case script characters (e.g. $\mathcal{K}$) are reserved for
operators on Banach spaces, with $\mathcal{I}$ denoting the
identity. Given a set $X$, we denote the closure of $X$
by $\overline{X}$.

In the rest of the paper, 
suppose that $\Omega$ is a multiply connected region defined by
the intersection of a simply connected domain $\Omega_{0}$ 
and the exteriors of a finite collection of bounded
simply connected domains $\{ \Omega_{i} \}_{i=1}^{m}$. 
Let $\Gamma_{i}$, denote the boundary of $\Omega_{i}$, and 
let $\Gamma = \cup_{i=0}^{m} \Gamma_{i}$ denote the boundary of $\Omega$.
Let $\bnu(\bx)$ denote the outward normal at $\bx \in \Gamma$.

\section{Stokes eigenvalue problem}
The Stokes Dirichlet eigenvalue problem is to find values $k^2$ such that
\begin{equation}
\begin{aligned}
  -\Delta \uu + \nabla p &= k^2 \uu \quad \textrm{in} \quad
  \Omega \label{eq:ostokes} \; , \\
  \nabla \cdot \uu &= 0 \; , \\
  \bu &=0 \quad \textrm{on} \quad \Gamma \, ,
\end{aligned}
\end{equation}
has a non-trivial solution $(\bu,p)$.
Since the Stokes eigenfunction $\bu$ is divergence free, 
and satisfies 
$\int_{\Gamma_{i}} \bu \cdot \bnu \, ds = 0$ 
for all $i=0,1,2\ldots m$, there exists a single valued
stream function $\psi$ associated with $\bu$, i.e., 
$\bu = \nabla^{\perp} \psi$ for all $\bx \in \Omega$.  

The Stokes eigenvalue problem can be reformulated as a buckling
eigenvalue problem with gradient boundary conditions for
the stream function $\psi$.
Plugging in $\bu = \nabla^{\perp} \psi$ and applying $\nabla^{\perp} \cdot$
to~\cref{eq:ostokes}, we observe that $\psi$ satisfies the following 
differential equation.

\begin{equation}
\begin{aligned}
\Delta (\Delta + k^2) \psi &= 0 \, \quad \textrm{in} \quad \Omega \, , \\
\nabla \psi &=0 \, ,\quad \textrm{in} \quad \Gamma \, .
\end{aligned}
\end{equation}

\section{Connection to buckling eigenvalues}
It is clear that if $w$ is a buckling eigenfunction with
eigenvalue $k^2$, then $\bu = \nabla^{\perp} w$ is a Stokes eigenfunction
with the same eigenvalue. 
Thus, for a given region $\Omega$,
the buckling eigenvalues are a subset of the Stokes eigenvalue. 

However, the converse is not necessarily true. 
The reason is the following.
As before, suppose that $\bu$ is a Stokes eigenfunction
with eigenvalue $k^2$.
Note that a single valued stream function $\psi$ associated with
$\bu$ always exists. 
Moreover, on each component $\Gamma_{i}$, 
$\psi$ satisfies $\partial_{\tau} \psi = 0$, which implies that
$\psi$ is a constant on $\Gamma_{i}$, for each $i=0,1,2\ldots m$.
If the domain is simply connected, then since $\psi$ is only
well-defined upto a constant, we can adjust to constant term 
to ensure that $\psi$ also satisfies $\psi = 0$ on $\Gamma_{0}$.
However, on multiply connected domains, this is not always 
the case since $\psi$ might be different constants on different
boundary components. 
In the following lemma, we identify necessary and sufficient 
conditions on the stream function $\psi$ to guarantee that
$\psi$ (upto a constant) is also a buckling eigenfunction.

\begin{lem}
\label{lem:mainlem}
Suppose that $k^2$ is a Stokes eigenvalue for the region $\Omega$ and 
let $\bu$ be the corresponding eigenfunction. 
Then, the stream function associated with the velocity $\bu$ defined by
$\psi = \left(\nabla^{\perp} \right)^{-1} \bu$ exists and is single valued. 
Furthermore, $k^2$ is also a buckling eigenvalue if and only if
there exists a constant $c$ such that
\begin{equation}
\frac{1}{\Gamma_{j}}\int_{\Gamma_{j}} \psi\, ds = c \quad \forall j=0,1,\ldots m \, .
\end{equation}
\end{lem}

\begin{proof}
Since $\bu$ is a Stokes eigenfunction, it satisfies
\begin{equation}
\int_{\Gamma_{j}} \bu \cdot \bn ds = 0 \quad j=1,2\ldots n \, .
\end{equation}
Thus, there exists a single valued stream function associated with the
velocity field $\bu$ such that 
\begin{equation}
\nabla^{\perp} w = \bu \, .
\end{equation}
Furthermore, since $\bu$ satisfies the oscillatory Stokes equations, 
$w$ satisfies
\begin{align}
\Delta (\Delta + k^{2} ) w &= 0 \, , \quad \bx \in \Omega \\
\nabla^{\perp} w = \bu &=0 \, ,\quad \bx \in \Gamma \, ,
\end{align}
In particular $\psi= c_{j}$ for $\bx \in \Gamma_{j}$ for
some constant $c_{j}$.
Thus $k$ is also a buckling eigenvalue if and only if 
\begin{equation}
c_{j} = \frac{1}{|\Gamma_{j}|}\int_{\Gamma_{j}} w ds =  c \, .
\end{equation}
The buckling eigenfunction is then given by $\psi -c$. 
\end{proof}


\section{Computing the stream function}
Given $k^2$ to be a Stokes eigenvalue, in this section, we describe
an integral formulation for computing the eigenfunction $\bu$
and the corresponding stream function $\psi$. 

Following the approach in~\cite{askhameig2019}, we
first reformulate the oscillatory Stokes equation
as an integral equation on the boundary using the combined
field representation.

Let $\GG(\bx,\by)$ denote the Stokeslet for the oscillatory
Stokes equation and $T_{ij\ell}(\bx,\by)$ denote the corresponding
Stresslet. 
These are given by the formulae
\begin{align} 
  \GG &= - \II \Delta \Gbh + \nabla \otimes \nabla \Gbh 
\label{eq:ostokeslet} \\
  T_{ij\ell} &= - \partial_{x_j} \Glap \delta_{i\ell}
  + \partial_{x_\ell} \left ( -\Delta \Gbh \delta_{ij} +
  \partial_{x_i} \left(\partial_{x_j} \Gbh \right) \right)
  \nonumber \\
  & \qquad+ \partial_{x_i} \left ( -\Delta \Gbh \delta_{\ell j} +
  \partial_{x_\ell} \left(\partial_{x_j} \Gbh \right) \right)
  \; . \label{eq:ostress} 
\end{align}
$\Gbh$ is the Green's function for the oscillatory
biharmonic equation given by
\begin{equation}
  \Gbh(\xx,\yy) = \frac{1}{k^2}
  \left (\frac{1}{2\pi} \log |\xx-\yy| +
  \frac{i}{4} H_0^{(1)}(k|\xx-\yy|) \right ) \, ,
  \label{eq:Gbh}
\end{equation}
where $k$ is the Helmholtz parameter in the oscillatory biharmonic equation,
and $H_{0}^{1}(r)$ is the Hankel function of the first kind of order zero,
and $\Glap(\xx,\yy)$ is the Laplace Green's function given by
\begin{equation}
  \Glap(\xx,\yy) = \frac{1}{2\pi} \log |\xx-\yy| \; . \nonumber
\end{equation}

Given a density $\bmu$, the oscillatory Stokes single and double layer
potentials denoted by $\bS[\bmu](\bx)$ and $\bD[\bmu](\bx)$ respectively,
are given by
\begin{align} \label{eq:singlelayer}
  \bS [\bmu] (\xx) &= \int_\Gamma \GG (\xx,\yy) \bmu(\yy)
  \, dS(\yy) \; , \\
  \bD [\bmu] (\xx) &= \int_\Gamma \left ( \TT_{\cdot,\cdot,\ell}(\xx,\yy)
  \nu_\ell(\yy)\right )^\intercal \bmu(\yy) \, dS(\yy) \; .
\end{align}
When solving the velocity boundary value problem for the oscillatory
Stokes equation on multiply connected domains, the velocity
is represented as a bomcinbed field layer potential, i.e.
setting $\bu = (i\eta \bS_{k} + \bD_{k})[\bmu]$, where
$\eta>0$ is a constant and $\bmu$ now is an unknown density.
By construction, this velocity field satisfies the oscillatory
Stokes equation.
On imposing the velocity boundary condition, and using the jump
conditions for the layer potentials (see~\cref{askhameig2019}, 
for example)
we obtain the following integral equation for the unknown density
$\bmu$,
\begin{equation}
\label{eq:inteq0}
(\cI - 2\cD_{k} -2i \eta \cS_{k})\bmu = 0 \, \quad \textrm{on} \quad \Gamma \,,
\end{equation}
where $\cS_{k}$ and $\cD_{k}$ are the restrictions of the layer potentials
$\bS_{k}$ and $\bD_{k}$ on the boundary $\Gamma$ respectively, and are given
by
\begin{align}
  \cS [\bmu] (\xx) &= \int_\Gamma \GG (\xx,\yy) \bmu(\yy)
  \, dS(\yy) \\
  \cD [\bmu] (\xx) &= \pv \int_\Gamma \left ( \TT_{\cdot,\cdot,\ell}(\bx,\by)
  \nu_\ell(\yy)
  \right )^\intercal \bmu(\yy) \, dS(\yy) \; .
\end{align}
Here \pv indicates that the integral is to be
evaluated in the principal value sense.
The integral equation~\cref{eq:inteq0} is known to be rank-deficient
for all values of $k$ due to the divergence-free condition
on $\bu$, and a standard approach is to use the following integral 
equation instead

\begin{equation}
\label{eq:inteq}
(\cI - 2\cD_{k} -2i \eta \cS_{k}-2\cW)\bmu = 0 \, \quad \textrm{on} \quad \Gamma \,,
\end{equation}
where $\cW$ is the operator given by
\begin{equation}
\cW[\bmu](\bx) = \frac{\bnu(x)}{|\Gamma|}\int_{\Gamma} \bmu(\by) \cdot \bnu(\by) \, ds \, .
\end{equation}

In~\cite{askhameig2019}, the authors show that the operator 
$\cI -2\cD_{k} -2i \cS_{k} -2\cW$ is not invertible if and only
if $k^2$ is a Stokes eigenvalue.
Furthermore if $\bmu$ is a non-zero null-vector of $\cI - 2\cD_{k} -2i \cS_{k} 
-2 \cW$, then 
$\bu = (i \eta \bS_{k} + \bD_{k})[\bmu]$ is the Stokes eigenfunction.

In order to verify the conditions in~\cref{lem:mainlem}, we
need to be able to compute the stream function associated with
the velocity $\bmu$ defined above.
If $\bmu = \mu_{\tau} \btau + \mu_{\nu} \bnu$, where $\btau=\bnu^{\perp}$
is the positively oriented tangential vector then the
layer potentials $\bS_{k}[\bmu]$ and $\bD_{k}[\bmu]$ can be rewritten as
\begin{align}
\bS_{k}[\bmu] &=  
\nabla^{\perp} 
\int_{\Gamma} (\partial_{\tau} \Gbh(\xx,\yy) \mu_{\tau}(\yy) -
\partial_{\nu} \Gbh(\xx,\yy) \mu_{\nu}(\yy)) \, ds \\
\bD_{k}[\bmu] &= 
-\nabla 
\int_{\Gamma} \Glap(\xx,\yy) \mu_{\nu}(\yy) \, ds 
+ \nabla^{\perp} 
\int_{\Gamma} \big( 2 \partial_{\nu\tau} \Gbh(\xx,\yy) \mu_{\nu}(\yy) 
+ \\
& \hspace*{35ex}(\partial_{\tau\tau} -\partial_{\nu\nu})\Gbh(\xx,\yy) \mu_{\tau}(\yy) 
\big) \, ds \, .
\end{align}

From the expressions above, it is clear that the stream function 
$\psi_{S}[\bmu]$ associated with the single layer potential is given
by
\begin{equation}
\psi_{S}[\bmu] = 
\int_{\Gamma} (\partial_{\tau} \Gbh(\xx,\yy) \mu_{\tau}(\yy) -
\partial_{\nu} \Gbh(\xx,\yy) \mu_{\nu}(\yy)) \, ds \, . 
\end{equation}

The situation with the double layer potential is a little trickier. 
We write the double layer potential as $\bD_{k}[\bmu] = \nabla \phi_{D}[\bmu]
+ \nabla^{\perp} \xi_{D}[\bmu]$, 
where 
\begin{align}
\label{eq:phiddef}
\phi_{D}[\bmu] &= 
-\int_{\Gamma} \Glap(\xx,\yy) \mu_{\nu}(\yy) \, ds  \, , \\
\xi_{D}[\bmu] &= 
\int_{\Gamma} \big( 2 \partial_{\nu\tau} \Gbh(\xx,\yy) \mu_{\nu}(\yy) 
+ (\partial_{\tau\tau} -\partial_{\nu\nu})\Gbh(\xx,\yy) \mu_{\tau}(\yy) ) \,
ds \, . 
\end{align}

If we can find a function $\tilde{\psi}_{D}[\bmu]$ such
that 
\begin{equation}
\label{eq:perptogradperp}
\nabla^{\perp} \tilde{\psi}_{D}[\bmu] = \nabla \phi_{D} [\bmu] \, ,
\end{equation}
then the stream function $\psi_{D}[\bmu]$ 
associated with the double layer potential would
be given by $\psi_{D}[\bmu] = \tilde{\psi}_{D}[\bmu] + \xi_{D}[\bmu]$.
However, on multiply connected domains, it is not necessarily true, 
that given a function $\phi_{D}[\bmu]$ of the form~\cref{eq:phiddef} 
there exists a single valued function $\tilde{\psi}_{D}[\bmu]$ 
which satisfies the condition~\cref{eq:perptogradperp} for
every $\bmu$.
Turns out, in the case where $\bmu$ satisfies the conditions
\begin{equation}
\label{eq:inteqconst}
\int_{\Gamma_{i}} \mu_{\nu} \, ds = 0 \, ,\quad i=0,1,2\ldots m \, ,
\end{equation}
then there exists a single valued function $\tilde{\psi}_{D}[\bmu]$
which satisfies~\cref{eq:perptogradperp} (see~\cite{rachh2015integral}, 
for example).
In the following lemma, we show that every null-vector of 
$\cI - 2\cD_{k} -2i \eta \cS_{k} - 2\cW$ satisfies the above integral
constraints.

\begin{lem}
Suppose $\bmu$ is a null-vector of 
$\cI - 2\cD_{k} - 2i \eta \cS_{k} -2\cW$, then
$\bmu$ satisfies~\cref{eq:inteqconst}.
\end{lem}
\begin{proof}
Insert proof here
\end{proof}
