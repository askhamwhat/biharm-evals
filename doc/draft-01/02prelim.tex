\section{Mathematical Preliminaries}

In this paper, vector-valued
quantities are denoted by bold, lower-case letters
(e.g. $\bh$), while tensor-valued quantities are bold
and upper-case (e.g. $\mathbf{T}$). 
Subscript indices of non-bold characters (e.g. $h_j$ or $T_{jkl}$)
are used to denote the entries within a vector ($\bh$) or tensor ($\bT$).
We use the standard Einstein summation convention; in other words, 
there is an implied sum taken over the repeated indices of 
any term (e.g. the symbol $a_{j} b_{j}$ is used to represent the sum
$\sum_{j} a_{j} b_{j}$).
If $\xx = (x_1,x_2)^\intercal$, then $\xx^\bot = (-x_2,x_1)^\intercal$.
Similarly, $\nabla^\bot = (-\partial_{x_2},\partial_{x_1})^\intercal$.
Upper-case script characters (e.g. $\mathcal{K}$) are reserved for
operators on Banach spaces, with $\mathcal{I}$ denoting the
identity.


\subsection{Green's functions}

Let $L_x$ denote a linear differential operator. A fundamental
solution $G(\xx,\yy)$ of $L_x$ satisfies the equation
$L_x G(\xx,\yy) = \delta_y(\xx)$ in the distributional sense, i.e.
for sufficiently smooth $f$
\begin{equation}
  L_x \int_{\R^2} G(\xx,\yy) f(\yy) \, dy = f(\xx) \; .
\end{equation}
We consider here
``free-space'' Green's functions, i.e. fundamental solutions which satisfy
some natural growth or decay conditions as $|\xx-\yy| \to \infty$.
The Green's function of the Helmholtz-biharmonic equation
\cref{eq:buck1} is given by 
\begin{equation}
  \Gbh(\xx,\yy) = \frac{1}{k^2}
  \left (\frac{1}{2\pi} \log |\xx-\yy| +
  \frac{i}{4} H_0^{(1)}(k|\xx-\yy|) \right ) \, , 
\end{equation}
where $k$ is the Helmholtz paramater in the Helmholtz-Biharmonic equation,
and $H_{0}^{1}(r)$ is the Hankel function of the first kind of order zero.
Note that this is a scaled difference of the Green's function for
Laplace, i.e.

\begin{equation}
  \Glap(\xx,\yy) = \frac{1}{2\pi} \log |\xx-\yy| \; ,
\end{equation}
and the Green's function for the Helmholtz equation

\begin{equation}
  \Ghelm(\xx,\yy) = -\frac{i}{4} H_0^{(1)}(k|\xx-\yy|) \; .
\end{equation}

\subsection{A Farkas-style integral representation}

In \cite{Farkas89}, Farkas derived integral equation formulations
for the solution of the Dirichlet problem for the biharmonic equation ($k=0$). 
In this formulation, the solution 
$w$ is represented as a sum of two layer potentials
\begin{equation}
  w(\xx) = \int_\Gamma \Kfark_1 (\xx,\yy) \sigma_1(\yy) \, dS(\yy)
  + \int_\Gamma \Kfark_2(\xx,\yy) \sigma_2(\yy) \, dS(\yy) \; ,
  \label{eq:farkRep}
\end{equation}
where $\sigma_{1}$, and $\sigma_{2}$ are unknown densities, and the kernels
$K_{1}^{F}$, and $K_{2}^{F}$ are linear combinations of derivatives of
the biharmonic Green's function and appropriately chosen to get second kind integral equations
for the unknown densities $\sigma_{1}$ and $\sigma_{2}$ on imposing the boundary conditions.

In \cite{jiang2013second}, Jiang extended the Farkas formulation for the solution of the modified
biharmonic equation, albeit for different boundary conditions. With minor modifications, their formulation
can also be adopted for the solution of the Dirichlet problem for the Helmholtz-biharmonic equation
\begin{align}
\Delta (\Delta + k^2) w &= 0 \quad \text{in} \quad \Omega \label{eq:helmbi} \\
w &= f \quad \text{on} \quad \Gamma \label{eq:helmbibc1} \\
\dwdn &=g \quad \text{on} \quad \Gamma \label{eq:helmbibc2} \,,
\end{align}
where $f,g$ are prescribed functions on the boundary. 

For two points on the plane, $\xx$ and $\yy$, let $\br = \yy - \xx$. Let the kernels $K_{1}^{F}$, and 
$K_{2}^{F}$ be defined by 
\begin{align}
  \Kfark_1(\xx,\yy) &= \partial_{n_{y}n_{y}n_{y}}\Gbh(\xx,\yy)
  + 3 \partial_{n_{y}\tau_{y}\tau_{y}}\Gbh(\xx,\yy) + k^{2} \partial_{n_{y}}\Gbh(\xx,\yy)\; , \label{eq:kf1} \\
  \Kfark_2(\xx,\yy) &= -\partial_{n_{y}n_{y}}\Gbh(\xx,\yy)
  + \partial_{\tau_{y}\tau_{y}}\Gbh(\xx,\yy) \label{eq:kf2} \; .
\end{align}
More explicitly, we have
\begin{align}
  \Kfark_1(\xx,\yy) &=  -\dfrac{\br \cdot \bn(\yy)}{|\br|^2}  \cdot \left( \frac{3 i}{2} H_0^{(1)}(k|\br|) - \frac{5 i}{2 k|\br|} H_1^{(1)}(k|\br|) + \frac{6}{\pi k^2 |\br|^2} + \frac{1}{2\pi}  \right) +     \\
  & \quad \dfrac{\left(\br \cdot \bn(\yy)\right)^3}{|\br|^4} \cdot \left( 2i H_0^{(1)}(k|\br|) - \frac{7 i}{2 k|\br|} H_1^{(1)}(k|\br|) + \frac{8}{\pi k^2 |\br|^2}  \right) \, , \nonumber \\
  \Kfark_2(\xx,\yy) &=  -\left(\frac{1}{2}-\dfrac{\left(\br \cdot \bn(\yy)\right)^2}{|\br|^2} \right)  \cdot \left( \frac{i}{2} H_0^{(1)}(k|\br|) - \frac{i}{k|\br|} H_1^{(1)}(k|\br|) + \frac{2}{\pi k^2 |\br|^2}  \right)\; .
\end{align}

On enforcing the Dirichlet boundary conditions for
$w$, we obtain the integral equation
\begin{equation}
  \begin{split}
    \begin{pmatrix}
      f(\xx) \\
      g(\xx)
    \end{pmatrix} &= 
    \int_\Gamma
    \begin{pmatrix}
      \Kfark_{11}(\xx,\yy) & \Kfark_{12}(\xx,\yy) \\
      \Kfark_{21}(\xx,\yy) & \Kfark_{22}(\xx,\yy)
    \end{pmatrix}
    \begin{pmatrix}
      \sigma_1(\yy)\\
      \sigma_2(\yy)
    \end{pmatrix}
    \, dS(\yy) \\
    &\quad + \begin{pmatrix}
      1/2 & 0 \\
      -\kappa(\xx) & 1/2
    \end{pmatrix}
    \begin{pmatrix}
      \sigma_1(\xx)\\
      \sigma_2(\xx)
    \end{pmatrix} \; , \label{eq:farkIntEq} 
  \end{split}
\end{equation}
where $\kappa$ is the signed curvature at $\xx \in \Gamma$.
The kernels are given
by $\Kfark_{11} = \Kfark_1$, $\Kfark_{12} = \Kfark_2$,
\begin{align}
  \Kfark_{21}(\xx,\yy) &= \left(\Kfark_1(\xx,\yy)\right )_{n_x} \nonumber \\
  &= \bn(\xx) \cdot \nabla_{\xx} K_{21}^{F}(\xx,\yy) \; \label{eq:farkK21},\\
  \Kfark_{22}(\xx,\yy) &= \left(\Kfark_2(\xx,\yy)\right )_{n_x} \nonumber \\
  &= \bn(\xx) \cdot \nabla_{\xx} K_{22}^{F}(\xx,\yy) \label{eq:farkK22} \; .
\end{align}

The following lemma shows that the integral equation~\cref{eq:farkIntEq} is invertible for smooth domains
whenever $k$ is not a buckling eigenfrequency. 

\begin{lem}
Suppose that $\Omega \subset \R^2$ is a simply connected region whose boundary $\Gamma$ is $\cC^{2}$. 
Furthermore, suppose that $k$ is not a buckling eigenfrequency of $\Omega$, i.e., the only solution $w$ to~\cref{eq:buck1,eq:buck2,eq:buck3} is  $w \equiv 0$. Suppose further that $f,g \in \cC^{2}(\Gamma)$. 
Then there exists a unique solution $\sigma_{1},\sigma_{2} \in \cC^{2}$ which satisfies~\cref{eq:farkIntEq}. 
Furthermore, $w(\bx)$ defined via~\cref{eq:farkRep} satisfies the Helmholtz-Biharmonic equation~\cref{eq:helmbi}
along with the boundary conditions $w(\bx)=f(\bx)$ and $\dwdn(\bx) = g(\bx)$ for $\bx \in\Gamma$.
\label{lem:farkinv}
\end{lem}

\begin{proof}
  The proof is a minor modification of the one presented in~\ref{eq:shidongmodbi}
  and is contained in~\cref{sec:farkproof}.
\end{proof}

The kernels $\Kfark_1$ and $\Kfark_2$ are
constructed with the goal that the $\Kfark_{ij}$ are
as smooth as possible. Suppose that the boundary
$\Gamma$ is of class $C^k$. Then, the kernels,
$\Kfark_{ij}(\xx,\yy)$, are $C^{k-2}$ functions on the boundary
for each $\yy \in \Gamma$.
Therefore, on a smooth
boundary, these kernels are smooth. However, on a
domain with a corner, we note 
that the kernel $\Kfark_{21}$
has a singularity with strength $r^{-2}$ (see~\cref{fig:ker21plot}). 
This singularity, in addition to the term in \cref{eq:farkIntEq}
which explicitly involves the
curvature, makes the representation \cref{eq:farkRep}
unstable for domains with high curvature (or
corners).

\subsection{An integral representation for the Stokes eigenvalue
  problem}

The Stokes eigenvalue problem (with ``no-slip'' boundary conditions)
seeks values $k^2$ such that the equations
\begin{align}
  \nabla p - \Delta \uu - k^2 \uu &= 0 \quad \textrm{in} \quad \Omega
  \; , \label{eq:helmstokes}\\
  \nabla \cdot \uu &= 0 \quad \textrm{in} \quad \Omega
  \; , \label{eq:helmstokescty} \\
  \uu &= 0 \quad \textrm{on} \quad \Gamma \; \label{eq:helmstokesnoslip}
\end{align}
have a non-trivial solution. In the case that $k=i\alpha$ for
some real-valued $\alpha$, the equations \cref{eq:helmstokes,eq:helmstokescty}
are known as the modified Stokes equations and are of particular
interest for their application to numerical simulations of unsteady
flow
\cite{pozrikidis1992boundary,biros2002embedded,jiang2013second}.
In the absence of a preferred name for the case of real-valued $k$,
we will refer to \cref{eq:helmstokes,eq:helmstokescty} as the
Helmholtz-Stokes equations.



\subsubsection{Stokeslets and stresslets}

In this section, we review the derivation of the
modified Stokeslet, but for the imaginary parameter
regime, which is the appropriate setting for the Stokes
eigenvalue problem.
The original modified Stokeslet is more well known,
and is of particular interest for its application to
numerical simulations of the unsteady Stokes equations
\cite{pozrikidis1992boundary,biros2002embedded}.

 Consider the solution of
\cref{eq:helmstokes,eq:helmstokescty} where a $\delta$-mass
centered at $\yy$ with strength $\ff$
has been added to the right-hand side of \cref{eq:helmstokes}, i.e.

\begin{align}
  \nabla p - \Delta \uu - k^2 \uu &= \delta_\yy \ff \; ,
  \label{eq:helmstokes_charge} \\
  \nabla \cdot \uu &= 0 \; .
\end{align}
Recall that

\begin{equation}
 \Delta \Glap(\xx,\yy) = \delta_\yy(\xx) \; .
\end{equation}
If we substitute this expression into
\eqref{eq:helmstokes_charge} and take the divergence,
we obtain

\begin{equation}
  p = \nabla \Glap(\xx,\yy) \cdot \ff \; .
\end{equation}
We then have, formally,

\begin{align}
  \uu &= - (\Delta + k^2)^{-1} ( \Delta \Glap \ff
  - \nabla (\nabla \Glap \cdot \ff ) ) \\
  &= \left ( -\Delta + \nabla \otimes \nabla \right )
  \Gbh \ff \; .
\end{align}
The tensor

\begin{equation} \label{eq:stokeslet}
  \GG = - \II \Delta \Gbh + \nabla \otimes \nabla \Gbh
\end{equation}
is then the equivalent of a Stokeslet
\cite{pozrikidis1992boundary} for the problem
\eqref{eq:helmstokes}.

A related object is the stresslet, which is defined
in terms of the stress tensor of a velocity, pressure
pair induced by a Stokeslet. For these tensors, we find
that it is more convenient to express them in index notation
with the Einstein index summing convention.
Recall that the stress tensor $\bsigma$ is defined as 

\begin{equation}
  \sigma_{ij} = -p \delta_{ij} + \left ( \partial_{x_j}u_i
  +\partial_{x_i} u_j \right ) \; ,
\end{equation}
where $\delta_{ij}$ is the standard Kronecker delta notation.
The stresslet $\TT$ is defined to be

\begin{align}
  T_{ijk} &= - \partial_{x_j} \Glap \delta_{ik}
  + \partial_{x_k} \left ( -\Delta \Gbh \delta_{ij} +
  \partial_{x_i} \left(\partial_{x_j} \Gbh \right) \right)
  \nonumber \\
  & \qquad+ \partial_{x_i} \left ( -\Delta \Gbh \delta_{kj} +
  \partial_{x_k} \left(\partial_{x_j} \Gbh \right) \right)
  \; .
\end{align}
Let $u_i = G_{ij} f_j$ and $p = \partial_{x_i} \Glap f_i$ be a
solution of the Stokes equations induced by a Stokeslet.
Then the corresponding stress tensor is given by
$\sigma_{ik} = T_{ijk} f_j$.

For the layer potentials of the next section, the following
formulas are useful. Let $\bnu$ be a given vector. When
summing over the third index, we obtain

\begin{equation}
  \TT_{\cdot,\cdot,k} \nu_k = -\bnu \otimes \nabla \Glap
  + \partial_\nu \left ( -\Delta \Gbh \II
  + \nabla \otimes \nabla \Gbh \right)
  + \nabla \otimes \left ( -\Delta \Gbh \bnu
  + \partial_{\nu} \nabla \Gbh \right) \; .
\end{equation}
%\begin{align}
% -\Delta \Gbh \bnu
% + \partial_{\nu} \nabla \Gbh &=
% ( (-\partial_{x_1x_1}-\partial_{x_2x_2}) \nu_1 \Gbh +
% (\partial_{x_1x_1}\nu_1 + \partial_{x_2x_1} \nu_2 )\Gbh,
% (-\partial_{x_1x_1}-\partial_{x_2x_2}) \nu_2 \Gbh +
% (\partial_{x_1x_2}\nu_1 + \partial_{x_2x_2} \nu_2 )\Gbh) \\
% &= (-\partial_{x_2} \partial_\tau \Gbh,
% \partial_{x_1} \partial_\tau \Gbh)
%\end{align}
%\begin{equation}
%  \begin{pmatrix}
%    -\partial_{x_2x_2} & \partial_{x_1x_2} \\
%    \partial_{x_1x_2} & -\partial_{x_1x_1}
%  \end{pmatrix} = -\nabla^\bot \otimes \nabla^\bot
%\end{equation}
Let $\btau = \bnu^\bot$. Then

\begin{equation}
  \TT_{\cdot,\cdot,k} \nu_k = -\bnu \otimes \nabla \Glap
  - \nabla^\bot \otimes \nabla^\bot \partial_\nu \Gbh
  +\nabla \otimes \nabla^\bot \partial_\tau \Gbh \; .
\end{equation}

\subsubsection{Layer potentials and integral representation}

We now use the Stokeslet and stresslet definitions
of the previous section to define the standard single
and double layer potentials of the Stokes eigenvalue
problem. The single layer potential is defined to
be

\begin{equation}
  \SS \bmu (\xx) = \int_\Gamma \GG (\xx,\yy) \bmu(\yy)
  \, dS(\yy) \; .
\end{equation}
The double layer potential is defined to be

\begin{equation} \label{eq:doublelayer}
  \DD \bmu (\xx) = \int_\Gamma \left ( \TT_{\cdot,\cdot,k}\nu_k(\yy)
  \right )^\intercal \bmu(\yy) \, dS(\yy) \; ,
\end{equation}
where $\bnu$ denotes the outward unit normal to the boundary.
If we write $\bmu = \bnu \mu_\nu + \btau \mu_\tau$,
where $\btau = \bmu^\bot$ is the positively oriented unit
tangent to the curve, then we have

\begin{align} \label{eq:stokesdlkernel}
  \left ( \TT_{\cdot,\cdot,k}\nu_k(\yy) \right )^\intercal
  \bmu(\yy) &= \left ( - \nabla \Glap(\xx,\yy) + 2 \nabla^\bot
  \partial_{\nu\tau} \Gbh(\xx,\yy) \right ) \mu_\nu(\yy) \nonumber \\
  & \qquad+
  \nabla^\bot \left (\partial_{\tau\tau}-\partial_{\nu\nu} \right )
  \Gbh(\xx,\yy) \mu_\tau(\yy) \; .
\end{align}

