\section{Introduction}

Results of some quick googling:

Finite element methods.
\begin{itemize}
\item Stokes eigenvalue problem: nonconforming finite
  element methods \cite{jia2009approximation}.
\end{itemize}

Integral equation methods.
\begin{itemize}
\item Book on elastodynamic eigenvalue problems
  \cite{kitahara2014boundary}
\end{itemize}

Consider the pure bending of an isotropic and homogeneous thin clamped
plate. 
Let $\Omega$ denote the midline of the thin plate, $\Gamma$ denote the 
boundary of $\Omega$ and $w$ denote the vertical displacement.
The buckling problem consists of finding 
values $k^{2}$ such that the boundary value problem

\begin{align}
\Delta (\Delta + k^2) w &= 0 \quad \text{in} \quad \Omega \label{eq:buck1} \\
w &= 0 \quad \text{on} \quad \Gamma \label{eq:buck2}\\
\dwdn &=0 \quad \text{on} \quad \Gamma \label{eq:buck3} \,.
\end{align}
has a non-trivial solution. In a slight abuse of notation,
these values $k^2$ and the corresponding
solutions are often referred to as ``eigenvalues'' and ``eigenfunctions'',
respectively.
It is well-known from classical potential theory that, 
for the buckling problem, there exists a countable collection of
such values $0 < k_{1}^{2} \leq k_{2}^2 \leq \ldots \uparrow \infty$,
counting multiplicites.
\begin{remark}
For notational convenience,  
we will refer to the modified Biharmonic problem
with imaginary parameter as the Helmholtz-Biharmonic problem.
\end{remark}

