\section{Introduction}


Consider the pure bending of an isotropic and homogeneous thin clamped
plate. 
Let $\Omega$ denote the midline of the plate, $\Gamma$ denote the 
boundary of $\Omega$ and $w$ denote the vertical displacement.
The buckling problem consists of finding 
values $k^{2}$ such that the boundary value problem

\begin{align}
\Delta (\Delta + k^2) w &= 0 \quad \textrm{in} \quad \Omega \label{eq:buck1} \; , \\
w &= 0 \quad \textrm{on} \quad \Gamma \label{eq:buck2} \; ,\\
\dwdn &=0 \quad \textrm{on} \quad \Gamma \label{eq:buck3} 
\end{align}
has a non-trivial solution. In a slight abuse of notation,
these values and the corresponding solutions are often
referred to as ``eigenvalues'' and ``eigenfunctions'',
respectively. 
It is well known that the buckling problem has a
countable collection of such values
$0 < k_{1}^{2} \leq k_{2}^2 \leq \ldots \uparrow \infty$,
counting multiplicites.

\begin{remark}
  When $k = i\alpha$, the differential equation
  \cref{eq:buck1} is known as the modified biharmonic
  equation. As there appears to be no preferred
  name for the equation with real-valued $k$,
  we will refer to \cref{eq:buck1} as the
  Helmholtz-biharmonic equation.
\end{remark}

The eigenvalues (and eigenfunctions)
of the boundary value problem \cref{eq:buck1,eq:buck2,eq:buck3}
describe the critical buckling load (and deflection)
of a thin plate under
a compressive force applied along the edge. While the
isotropy and homogeneity assumptions are restrictive,
the equations are still of interest in materials design
(CITATIONS). The analytical properties of the eigenvalues and
eigenfunctions are also of interest from a mathematical
perspective, often as they contrast with the eigenvalues
and eigenmodes of an idealized drum (CITATIONS). For instance,
it is well-known that the first eigenfunction of the
drum problem (i.e. the eigenvalue problem of the Laplace
equation with homogeneous Dirichlet boundary conditions)
is either nonpositive or nonnegative on any domain.
In contrast, it is possible for the first buckling
eigenfunction of a convex domain with analytic boundary
to take both positive and negative values \cite{antunes2011buckling}.

Numerical simulations have long played an important
role in the analyses described above and there are a number
of available methods (note that some of the references
included here are written in terms of the related
Stokes eigenvalue problem, which has equivalent
eigenvalues on simply connected domains). We do not
seek to review the literature here but will provide
some context for the present work. Finite element methods
are ubiquitous in mechanical engineering applications
and are flexible with respect to both the governing
equations and the shape of the domain. Such a method
would first obtain a ``weak'' form of the equations
\cref{eq:buck1,eq:buck2,eq:buck3} and then 
approximate the solution space in some
finite dimensional basis, resulting in a discrete model
for the buckling problem.
The eigenvalues $k^2$ are then
approximated by the eigenvalues of the discrete
model \cite{johnson1974beam,rannacher1979nonconforming,
  jia2009approximation,carstensen2014guaranteed}.
Because of the fourth order term in \cref{eq:buck1},
the system matrices resulting from a finite element
discretization can have large condition numbers (growing
like $1/h^4$ where $h$ is the diameter of a typical element
in the discretization) which can cause numerical instability.

Integral equation methods \cite{kress1989linear,
  atkinson2009numerical,kitahara2014boundary} present
a natural alternative to the finite element method
for the buckling problem, because (a) the
integral representation acts as an analytical preconditioner
for the fourth order term and (b) the problem is reduced
to one defined on the boundary of the domain alone,
requiring fewer degrees of freedom to resolve the solution
on a complex domain. In an integral equation method,
the solution $w$ is represented in terms of a sum of
two layer potentials (which a priori satisfy
\cref{eq:buck1}) with unknown densities. Substituting this
representation into the boundary conditions
\cref{eq:buck2,eq:buck3} results in an integral equation
for the unknown densities, which
is in turn discretized using numerical quadrature.
Unlike the original equation \cref{eq:buck1} and the weak
formulation used in a finite element method, the integral
equation for the unknown densities has a nonlinear dependence
on the parameter $k^2$. In an integral equation method, then,
the eigenvalues are found by searching for the values
of $k^2$ where the discrete system for the unknown densities
is no longer invertible.

While integral equation methods for the buckling problem
have been considered previously, the existing formulations
have resulted in first kind integral equations, i.e. equations
in which the operator is compact \cite{kitahara2014boundary,
  antunes2011buckling}. This is unsatisfying from a theoretical
perspective, because such operators are never invertible
(though, they are injective precisely when $k^2$ is an
eigenvalue \cite{kitahara2014boundary}) so that the
relation between the non-invertibility of the discrete system
and the eigenvalues is obscured. Further, such representations
create numerical challenges because common measures of the
``non-invertibility'' of a matrix, like the smallest singular
value or the determinant, converge rapidly to zero for all
values of $k^2$ as the boundary is refined. Therefore, the
measure of whether $k^2$ is an approximate eigenvalue is
{\em relative to the current grid} for first kind equations.

Second kind integral equations, i.e. integral equations
of the form $\mathcal{I} + \mathcal{K}$ where $\mathcal{K}$
is compact, have a more satisfying theory
\cite{reed1972methods,kress1989linear},
which translates well to numerical implementation
\cite{atkinson2009numerical,bornemann2010numerical,
  hackbusch2012integral,zhao2015robust}. Two facts are
of particular importance here: with an appropriately
chosen representation, (a) the Fredholm determinant of the
integral equation will be a well-defined analytic function
of $k^2$ with zeros precisely at the eigenvalues and (b) the 
determinant of the matrix resulting from
a Nystr\"{o}m discretization of the integral equation
will converge to the Fredholm determinant of the operator
\cite{bornemann2010numerical}.
In \cite{zhao2015robust}, Zhao and Barnett demonstrated
that the eigenvalues of the drum problem can be
computed with such a framework, resulting in
methods which are spectrally accurate with respect
to both the sampling of the boundary and the sampling
in $k^2$. There are natural extensions of existing second
kind integral
equations for the biharmonic \cite{farkas1989mathematical}
and modified biharmonic \cite{jiang2013second} equations
which would be appropriate for the buckling problem, but
these representations suffer from instability for
high curvature domains, limiting their applicability
\cite{rachh2017integral}.

In this paper, we develop a second kind integral
equation for the buckling problem based on Stokeslet
and stresslet inspired kernels, as was done for the
$k=0$ case in \cite{rachh2017integral}. This allows
for a robust approach to computing the eigenvalues
of the buckling problem, modeled after the approach for
the drum problem in \cite{zhao2015robust}, which we
demonstrate with a number of examples from the
literature.

The rest of this paper is organized as follows ...

