\section{Introduction}


Consider the pure bending of an isotropic and homogeneous thin clamped
plate. 
Let $\Omega$ denote the midline of the plate, $\Gamma$ denote the 
boundary of $\Omega$ and $w$ denote the vertical displacement.
The buckling eigenvalue problem consists of finding 
values $k^{2}$ such that the boundary value problem

\begin{align}
\Delta (\Delta + k^2) w &= 0 \quad \textrm{in} \quad \Omega \label{eq:buck1} \; , \\
w &= 0 \quad \textrm{on} \quad \Gamma \label{eq:buck2} \; ,\\
\dwdn &=0 \quad \textrm{on} \quad \Gamma \label{eq:buck3} 
\end{align}
has a non-trivial solution $w$. 
It is well known that the values $k^2$ are necessarily real 
and positive and there is a countable collection
of such values 
$0 < k_{1}^{2} \leq k_{2}^2 \leq \ldots \uparrow \infty$,
counting multiplicites.

\begin{remark}
  When $k = i\alpha$, the differential equation
  \cref{eq:buck1} is known as the modified biharmonic
  equation. As there appears to be no preferred
  name for the equation with real-valued $k$,
  we will refer to \cref{eq:buck1} as the
  oscillatory biharmonic equation.
\end{remark}

The eigenvalues (and eigenfunctions)
of the boundary value problem \cref{eq:buck1,eq:buck2,eq:buck3}
describe the critical buckling load (and deflection)
of a thin plate under
a compressive force applied along the edge. While the
isotropy and homogeneity assumptions are restrictive,
the equations are still of interest in materials design
(CITATIONS). The analytical properties of the eigenvalues and
eigenfunctions are also of interest from a mathematical
perspective, often as they contrast with the eigenvalues
and eigenmodes of an idealized drum (CITATIONS). For instance,
it is well-known that the first eigenfunction of the
drum problem (i.e. the eigenvalue problem of the Laplace
equation with homogeneous Dirichlet boundary conditions)
is either nonpositive or nonnegative on any domain.
In contrast, it is possible for the first buckling
eigenfunction of a convex domain with analytic boundary
to take both positive and negative values \cite{antunes2011buckling}.

A related problem is that of the Stokes eigenvalue problem
which is to find values $k^2$ such that
\begin{align}
\Delta (\Delta + k^2) w &= 0 \quad \textrm{in} \quad \Omega \label{eq:stokes1} \; , \\
\nabla w &= 0 \quad \textrm{on} \quad \Gamma \label{eq:stokes2} \; ,\\
\end{align}
has a non-trivial solution $w$.
While the buckling eigenvalues are identical to the Stokes
eigenvalues on simply connected domains, on multiply
connected domains it turns out that the buckling eigenvalues
are a subset of the Stokes eigenvalues.

In this note, we derive integral conditions on the stream function 
associated with the Stokes eigenfunction
to determine whether the Stokes eigenvalue is also a 
buckling eigenvalue on multiply connected domains.
Thus, given the Stokes eigenvalues 
of the domain and the stream functions associated with the 
corresponding eigenfunctions, 
the buckling eigenvalues can be determined using a simple
post-processing step.
Given $k^2$ to be a Stokes eigenvalue, we also derive
a well-conditioned second-kind integral equation for computing 
the stream function associated with the 
Stokes eigenfunction. 


Numerical simulations have long played an important
role in the analyses described above and there are a number
of available methods (note that some of the references
included here are written in terms of the related
Stokes eigenvalue problem, which has equivalent
eigenvalues on simply connected domains). We do not
seek to review the literature here but will provide
some context for the present work. Finite element methods
are ubiquitous in mechanical engineering applications
and are flexible with respect to both the governing
equations and the shape of the domain. Such a method
would first obtain a ``weak'' form of the equations
\cref{eq:buck1,eq:buck2,eq:buck3} and then 
approximate the solution space in some
finite dimensional basis, resulting in a discrete model
for the buckling problem.
The eigenvalues $k^2$ are then
approximated by the eigenvalues of the discrete
model \cite{johnson1974beam,rannacher1979nonconforming,
  jia2009approximation,carstensen2014guaranteed}.

\note{needs work}

The rest of this paper is organized as follows ...

