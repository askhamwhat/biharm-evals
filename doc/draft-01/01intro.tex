\section{Introduction}


Consider the pure bending of an isotropic and homogeneous thin clamped
plate. 
Let $\Omega$ denote the midline of the plate, $\Gamma$ denote the 
boundary of $\Omega$ and $w$ denote the vertical displacement.
The buckling problem consists of finding 
values $k^{2}$ such that the boundary value problem

\begin{align}
\Delta (\Delta + k^2) w &= 0 \quad \text{in} \quad \Omega \label{eq:buck1} \; , \\
w &= 0 \quad \text{on} \quad \Gamma \label{eq:buck2} \; ,\\
\dwdn &=0 \quad \text{on} \quad \Gamma \label{eq:buck3} 
\end{align}
has a non-trivial solution. In a slight abuse of notation,
these values and the corresponding solutions are often
referred to as ``eigenvalues'' and ``eigenfunctions'',
respectively. 
It is well known that the buckling problem has a
countable collection of such values
$0 < k_{1}^{2} \leq k_{2}^2 \leq \ldots \uparrow \infty$,
counting multiplicites.

The eigenvalues (and eigenfunctions)
of the boundary value problem \cref{eq:buck1,eq:buck2,eq:buck3}
describe the critical buckling load (and deflection)
of a thin plate under
a compressive force applied along the edge. While the
isotropy and homogeneity assumptions are restrictive,
the equations are still of interest in materials design
(CITATIONS). The analytical properties of the eigenvalues and
eigenfunctions are also of interest from a mathematical
perspective, often as they contrast with the eigenvalues
and eigenmodes of an idealized drum (CITATIONS).

Numerical simulations have long played an important
in the analyses described above and there are a number
of available methods (note that some of the references
included here are written in terms of the related
Stokes eigenvalue problem, which has equivalent
eigenvalues on simply connected domains).
The differential equation can
be discretized directly, resulting in a discrete
generalized eigenvalue problem for the buckling
eigenvalues \cite{jia2009approximation}. Integral
equation methods \cite{kitahara2014boundary} are
a natural choice for this problem, because (a) the
fourth order term induces strong
ill-conditioning when the differential equation
is discretized directly and (b) the problem is reduced
to one defined on the domain alone, requiring fewer
degrees of freedom to resolve the solution.

\begin{remark}
  When $k = i\alpha$, the differential equation
  \cref{eq:buck1} is known as the modified biharmonic
  equation. As there appears to be no preferred
  name for the equation with real-valued $k$,
  we will refer to \cref{eq:buck1} as the
  Helmholtz-biharmonic equation.
\end{remark}

