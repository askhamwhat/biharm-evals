
\section{Numerical methods}
Recall that we compute the buckling eigenmodes via computing the zeros
of the determinant of linear system corresponding to a Nystr\"{o}m 
discretization of the integral operator $A(k)$ in~\cref{eq:inteqfinal}.
As in section~\cref{sec:spec}, let $\tilde{A}_{k}$ denote the discretized
linear system.
Thus, the three main ingredients for the fast and accurate computation
of buckling eigenvalues (and its corresponding eigenfunctions) are
high order Nystr\"{o}m discretizations of the integral operator 
$A(k)$; for a given parameter $k$, the rapid evaluation of the determinant 
$\tilde{A}_{k}$; and finding the zeros of $\text{det}(\tilde{A}_{k})$ (which
is non-linear in $k$) while minimizing the function evaluations.

For discretizing the integral operator using the
Nystr\"{o}m method, we divide the boundary $\Gamma$ into
panels and represent the unknown density and the
boundary data by their values at scaled Gauss-Legendre
nodes on each panel. 
Let $n_{p}$ denote the number of Gauss-Legendre panels.
We discretize each panel using 16 scaled Gauss-Legendre nodes.
Then $n_{d} = 16 n_{p}$ is the number of discretization points 
on the boundary.
Let $\bx_{j}$ denote the discretization nodes, $w_{j}$ denote the
appropriately scaled Gauss-Legendre quadrature weights for smooth functions,
and $\bmu_{j}$ denote the unknown density at $\bx_{j}$.
When forming the linear system, we use scaled unknowns, 
$\bmu_{j} \sqrt{w_{j}}$, so that the spectral properties of the 
discrete system with respect to the $l_2$ norm
are approximations of the spectral properties of the continuous
system as on operator on $L_2$ (for more on this point of view, 
see \cite{bremer2012}).
The integral kernels $K_{ij}$ in~\cref{eq:inteqfinal} 
are either smooth or have a weak (logarithmic) singularity. 
For the smooth kernels, we use standard 
Gauss-Legendre quadrature rule.
For kernels with a logarithmic singularity, we use order $20$ 
generalized Gaussian quadrature rule~\cite{bremer2010,bremer2010u}.
Note that the resulting discretized linear system $\tilde{A}_{k}$ 
is a dense matrix of size $M=2n_{d}+N+1$, 
where $N$ is the number of connected components.

For small problems ($M \leq 10^3$), we directly compute
$\text{det}(\tilde{A}_{k})$ using standard linear algebra routines
from LAPACK. 
However, since the matrix $\tilde{A}_{k}$ is dense,
the computational complexity of computing $\text{det}(\tilde{A}_{k})$ 
using LAPACK routines is $O(M^3)$.
Moreover, for evaluating the buckling eigenmodes, $\text{det}(\tilde{A}_{k})$
must be evaluated for many different values of $k$.
Thus, for larger problems, we use the HIFIE method for evaluating
$\text{det}(\tilde{A}_{k})$. 
This method exploits the rank structure of $\tilde{A}_{k}$ (large blocks
of $\tilde{A}_{k}$ corresponding to interactions between well-separated
clusters are low-rank) and evaluates the determinant in O($M \cdot \log{M})$ 
CPU-time.


