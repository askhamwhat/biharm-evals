\documentclass[preprint,12pt]{article} 
\usepackage{b}
\usepackage[a4paper]{geometry}
\geometry{top=1.0in, bottom=1.0in, left=1.0in, right=1.0in}
\usepackage[small,it]{caption}
\usepackage{empheq}

\usepackage{subcaption}

\usepackage{cleveref}
\crefname{equation}{}{}
\Crefname{equation}{}{}
\crefname{eqnarray}{}{}
\Crefname{eqnarray}{}{}

%%%%%%%%%%%%%%%%%%%%%%%%%%%%%%%%%%%%%%%%%%%%%%%%%%%%%%%%%%%%


\def\br{{\boldsymbol r}}
\def\bh{{\boldsymbol h}}
\def\bn{{\boldsymbol n}}
\def\bT{{\boldsymbol T}}
\def\xx{{\boldsymbol x}}
\def\bx{{\boldsymbol x}}
\def\yy{{\boldsymbol y}}
\def\by{{\boldsymbol y}}
\def\ss{{\boldsymbol s}}
\def\tt{{\boldsymbol t}}
\def\uu{{\boldsymbol u}}
\def\bu{{\boldsymbol u}}
\def\bl{{\boldsymbol \ell}}
\def\ff{{\boldsymbol f}}
\def\GG{{\bf G}}
\def\Gbh{{G_{BH}}}
\def\II{{\bf I}}
\def\TT{{\bf T}}
\def\SS{{\bf S}}
\def\DD{{\bf D}}
\def\bsigma{{\boldsymbol \sigma}}
\def\bmu{{\boldsymbol \mu}}
\def\bnu{{\boldsymbol \nu}}
\def\btau{{\boldsymbol \tau}}
\def\bigo{\mathcal{O}}
\def\littleo{o}
\newcommand\dwdn{\frac{\partial w}{\partial n}}
\newcommand{\cC}{\mathcal C}

\newtheorem{remark}{Remark}
\newtheorem{proposition}{Proposition}
\newtheorem{lem}{Lemma}


%%%%%%%%%%%%%%%%%%%%%%%%%%%%%%%%%%%%%%%%%%%%%%%%%%%%%%%%%%%%%

\title{A Fredholm operator approach to the plate
buckling eigenvalue problem}
\author{Askham and Rachh}
\date{\today}

\begin{document}

\maketitle

%%%%%%%%%%%%%%%%%%%%%%%%%%%%%%%%%%%%%%%%%%%%%%%%%%%%%%%%%%%%

\section{Introduction}

Results of some quick googling:

Finite element methods.
\begin{itemize}
\item Stokes eigenvalue problem: nonconforming finite
  element methods \cite{jia2009approximation}.
\end{itemize}

Integral equation methods.
\begin{itemize}
\item Book on elastodynamic eigenvalue problems
  \cite{kitahara2014boundary}
\end{itemize}

Consider the pure bending of an isotropic and homogeneous thin clamped
plate. 
Let $\Omega$ denote the midline of the thin plate, $\Gamma$ denote the 
boundary of $\Omega$ and $w$ denote the vertical displacement.
In this setup, the buckling eigenvalue problem consists of finding 
eigenvalues $k^{2}$ and corresponding non-trivial eigenfunctions
(or buckling modes) $w$ which satisfy the homogeneous Dirichlet
modified Biharmonic problem (with an imaginary parameter $k$) given by
\begin{align}
\Delta (\Delta + k^2) w &= 0 \quad \text{in} \quad \Omega \label{eq:buck1} \\
w &= 0 \quad \text{on} \quad \Gamma \label{eq:buck2}\\
\dwdn &=0 \quad \text{on} \quad \Gamma \label{eq:buck3} \,.
\end{align}
It is well-known from classical potential theory that, 
for the buckling eigenvalue problem, there
exists a countable collection of eigenfrequencies 
$0 < k_{1}^{2} \leq k_{2}^2 \leq \ldots \uparrow \infty$
counting multiplicites.
\begin{remark}
For notational convenience,  
we will refer to the modified Biharmonic problem
with imaginary parameter as the Helmholtz-Biharmonic problem.
\end{remark}


\section{Mathematical Preliminaries}

In this paper, vector-valued
quantities are denoted by bold, lower-case letters
(e.g. $\bh$), while tensor-valued quantities are bold
and upper-case (e.g. $\mathbf{T}$). 
Subscript indices of non-bold characters (e.g. $h_j$ or $T_{jkl}$)
are used to denote the entries within a vector ($\bh$) or tensor ($\bT$).
We use the standard Einstein summation convention; in other words, 
there is an implied sum taken over the repeated indices of 
any term (e.g. the symbol $a_{j} b_{j}$ is used to represent the sum
$\sum_{j} a_{j} b_{j}$).
$\xx = (x_1,x_2)^\intercal$ then $\xx^\bot = (-x_2,x_1)^\intercal$.
Similarly, $\nabla^\bot = (-\partial_{x_2},\partial_{x_1})^\intercal$.


\subsection{The Farkas integral representation}
Following the treatment of \cite{askham}, the fundamental solution
of the Helmholtz-Biharmonic equation is given by
\begin{equation}
  G_{BH}(\xx,\yy) = -\frac{1}{k^2}
  \left (\frac{1}{2\pi} \log |\xx-\yy| +
  \frac{i}{4} H_0^{(1)}(k|\xx-\yy|) \right ) \, , 
\end{equation}
where $k$ is the Helmholtz paramater in the Helmholtz-Biharmonic equation,
and $H_{0}^{1}(r)$ is the Hankel function of the first kind of order zero.

In \cite{Farkas89}, Farkas derived integral equation formulations
for the solution of the Dirichlet problem for the biharmonic equation ($k=0$). 
In this formulation, the solution 
$w$ is represented as a sum of two layer potentials
\begin{equation}
  w(\xx) = \int_\Gamma K_1^F (\xx,\yy) \sigma_1(\yy) \, dS(\yy)
  + \int_\Gamma K_2^F(\xx,\yy) \sigma_2(\yy) \, dS(\yy) \; ,
  \label{eq:farkRep}
\end{equation}
where $\sigma_{1}$, and $\sigma_{2}$ are unknown densities, and the kernels
$K_{1}^{F}$, and $K_{2}^{F}$ are linear combinations of derivatives of
the biharmonic Green's function and appropriately chosen to get second kind integral equations
for the unknown densities $\sigma_{1}$ and $\sigma_{2}$ on imposing the boundary conditions.

In \cite{shidongmodbi}, Jiang extended the Farkas formulation for the solution of the modified
biharmonic equation, albeit for different boundary conditions. With minor modifications, their formulation
can also be adopted for the solution of the Dirichlet problem for the Helmholtz-biharmonic equation
\begin{align}
\Delta (\Delta + k^2) w &= 0 \quad \text{in} \quad \Omega \label{eq:helmbi} \\
w &= f \quad \text{on} \quad \Gamma \\
\dwdn &=g \quad \text{on} \quad \Gamma \,,
\end{align}
where $f,g$ are prescribed functions on the boundary. 

For two points on the plane, $\xx$ and $\yy$, let $\br = \yy - \xx$. Let the kernels $K_{1}^{F}$, and 
$K_{2}^{F}$ be defined by 
\begin{align}
  K_1^F(\xx,\yy) &= \partial_{n_{y}n_{y}n_{y}}\Gbh(\xx,\yy)
  + 3 \partial_{n_{y}\tau_{y}\tau_{y}}\Gbh(\xx,\yy) + k^{2} \partial_{n_{y}}\Gbh(\xx,\yy)\; , \label{eq:kf1} \\
  K_2^F(\xx,\yy) &= -\partial_{n_{y}n_{y}}\Gbh(\xx,\yy)
  + \partial_{\tau_{y}\tau_{y}}\Gbh(\xx,\yy) \label{eq:kf2} \; .
\end{align}
More explicitly, we have
\begin{align}
  K_1^F(\xx,\yy) &=  -\dfrac{\br \cdot \bn(\yy)}{|\br|^2}  \cdot \left( \frac{3 i}{2} H_0^{(1)}(k|\br|) - \frac{5 i}{2 k|\br|} H_1^{(1)}(k|\br|) + \frac{6}{\pi k^2 |\br|^2} + \frac{1}{2\pi}  \right) +     \\
  & \quad \dfrac{\left(\br \cdot \bn(\yy)\right)^3}{|\br|^4} \cdot \left( 2i H_0^{(1)}(k|\br|) - \frac{7 i}{2 k|\br|} H_1^{(1)}(k|\br|) + \frac{8}{\pi k^2 |\br|^2}  \right) \, , \nonumber \\
  K_2^F(\xx,\yy) &=  -\left(\frac{1}{2}-\dfrac{\left(\br \cdot \bn(\yy)\right)^2}{|\br|^2} \right)  \cdot \left( \frac{i}{2} H_0^{(1)}(k|\br|) - \frac{i}{k|\br|} H_1^{(1)}(k|\br|) + \frac{2}{\pi k^2 |\br|^2}  \right)\; .
\end{align}

On enforcing the Dirichlet boundary conditions for
$w$, we obtain the integral equation
\begin{equation}
  \begin{split}
    \begin{pmatrix}
      f(\xx) \\
      g(\xx)
    \end{pmatrix} &= 
    \int_\Gamma
    \begin{pmatrix}
      K_{11}^F(\xx,\yy) & K_{12}^F(\xx,\yy) \\
      K_{21}^F(\xx,\yy) & K_{22}^F(\xx,\yy)
    \end{pmatrix}
    \begin{pmatrix}
      \sigma_1(\yy)\\
      \sigma_2(\yy)
    \end{pmatrix}
    \, dS(\yy) \\
    &\quad + \begin{pmatrix}
      1/2 & 0 \\
      -\kappa(\xx) & 1/2
    \end{pmatrix}
    \begin{pmatrix}
      \sigma_1(\xx)\\
      \sigma_2(\xx)
    \end{pmatrix} \; , \label{eq:farkIntEq} 
  \end{split}
\end{equation}
where $\kappa$ is the signed curvature at $\xx \in \Gamma$.
The kernels are given
by $K_{11}^F = K_1^F$, $K_{12}^F = K_2^F$,
\begin{align}
  K_{21}^F(\xx,\yy) &= \left(K_1^F(\xx,\yy)\right )_{n_x} \nonumber \\
  &= \bn(\xx) \cdot \nabla_{\xx} K_{21}^{F}(\xx,\yy) \; \label{eq:farkK21},\\
  K_{22}^F(\xx,\yy) &= \left(K_2^F(\xx,\yy)\right )_{n_x} \nonumber \\
  &= \bn(\xx) \cdot \nabla_{\xx} K_{22}^{F}(\xx,\yy) \label{eq:farkK22} \; .
\end{align}

The following lemma shows that the integral equation~\cref{eq:farkIntEq} is invertible for smooth domains
whenever $k$ is not a buckling eigenfrequency. 

\begin{lem}
Suppose that $\Omega \subset \R^2$ is a simply connected region whose boundary $\Gamma$ is $\cC^{2}$. 
Furthermore, suppose that $k$ is not a buckling eigenfrequency of $\Omega$, i.e., the only solution $w$ to~\cref{eq:buck1,eq:buck2,eq:buck3} is  $w \equiv 0$. Suppose further that $f,g \in \cC^{2}(\Gamma)$. 
Then there exists a unique solution $\sigma_{1},\sigma_{2} \in \cC^{2}$ which satisfies~\cref{eq:farkIntEq}. 
Furthermore, $w(\bx)$ defined via~\cref{eq:farkRep} satisfies the Helmholtz-Biharmonic equation~\cref{eq:helmbi}
along with the boundary conditions $w(\bx)=f(\bx)$ and $\dwdn(\bx) = g(\bx)$ for $\bx \in\Gamma$. 
\end{lem}

\begin{proof}
The proof is a minor modification of the one presented in~\ref{eq:shidongmodbi} and is contained in~\cref{sec:app}.
\end{proof}

The kernels $K^F_1$ and $K^F_2$ are
constructed with the goal that the $K^F_{ij}$ are
as smooth as possible. Suppose that the boundary
$\Gamma$ is of class $C^k$. Then, the kernels,
$K^F_{ij}(\xx,\yy)$, are $C^{k-2}$ functions on the boundary
for each $\yy \in \Gamma$.
Therefore, on a smooth
boundary, these kernels are smooth. However, on a
domain with a corner, we note 
that the kernel $K_{21}^F$
has a singularity with strength $r^{-2}$ (see~\cref{fig:ker21plot}). 
This singularity, in addition to the term in \cref{eq:farkIntEq}
which explicitly involves the
curvature, makes the representation \cref{eq:farkRep}
unstable for domains with high curvature (or
corners).

\subsection{Helmholtz-Stokes flow in a plane}

In this section, we review the derivation of the
modified Stokeslet, but for the imaginary parameter
regime --- which we will call the Helmholtz Stokeslet ---
which is the appropriate setting for the Stokes
eigenvalue problem.
The original modified Stokeslet is more well known,
and is of particular interest for its application to
numerical simulations of the unsteady Stokes equations
\cite{pozrikidis1992boundary,biros2002embedded}.

To begin, we restate the Stokes eigenvalue
problem as
\begin{align}
  \nabla p - \Delta \uu - k^2 \uu &= 0 \; , \label{eq:helmstokes}\\
  \nabla \cdot \uu &= 0 \; . \label{eq:helmstokescty}
\end{align}
Consider the solution of this problem where a
$\delta$ centered at $\yy$ with strength $\ff$
has been added to the right-hand side, i.e.

\begin{align}
  \nabla p - \Delta \uu - k^2 \uu &= \delta_\yy \ff \; ,
  \label{eq:helmstokes_charge} \\
  \nabla \cdot \uu &= 0 \; .
\end{align}
Note that

\begin{equation}
  \Delta G_L(\xx,\yy) = \delta_\yy(\xx) \; .
\end{equation}
If we substitute this expression into
\eqref{eq:helmstokes_charge} and take the divergence,
we obtain

\begin{equation}
  p = \nabla G_L(\xx,\yy) \cdot \ff \; .
\end{equation}
We then have, formally,

\begin{align}
  \uu &= - (\Delta + k^2)^{-1} ( \Delta G_L \ff
  - \nabla (\nabla G_L \cdot \ff ) ) \\
  &= \left ( -\Delta + \nabla \otimes \nabla \right )
  G_{BH} \ff \; .
\end{align}
The tensor

\begin{equation}
  \GG = - \II \Delta G_{BH} + \nabla \otimes \nabla G_{BH}
\end{equation}
is then the equivalent of a Stokeslet
\cite{pozrikidis1992boundary} for the problem
\eqref{eq:helmstokes}.

A related object is the stresslet, which is defined
in terms of the stress tensor of a velocity, pressure
pair induced by a Stokeslet. For these tensors, we find
that it is more convenient to express them in index notation
with the Einstein index summing convention.
Recall that the stress tensor $\bsigma$ is defined as 

\begin{equation}
  \sigma_{ij} = -p \delta_{ij} + \left ( \partial_{x_j}u_i
  +\partial_{x_i} u_j \right ) \; ,
\end{equation}
where $\delta_{ij}$ is the standard Kronecker delta notation.
The stresslet $\TT$ is defined to be

\begin{align}
  T_{ijk} &= - \partial_{x_j} G_L \delta_{ik}
  + \partial_{x_k} \left ( -\Delta G_{BH} \delta_{ij} +
  \partial_{x_i} \left(\partial_{x_j} G_{BH} \right) \right)
  \nonumber \\
  & \qquad+ \partial_{x_i} \left ( -\Delta G_{BH} \delta_{kj} +
  \partial_{x_k} \left(\partial_{x_j} G_{BH} \right) \right)
  \; .
\end{align}
Let $u_i = G_{ij} f_j$ and $p = \partial_{x_i} G_L f_i$ be a
solution of the Stokes equations induced by a Stokeslet.
Then the corresponding stress tensor is given by
$\sigma_{ik} = T_{ijk} f_j$.

For the layer potentials of the next section, the following
formulas are useful. Let $\bnu$ be a given vector. When
summing over the third index, we obtain

\begin{equation}
  \TT_{\cdot,\cdot,k} \nu_k = -\bnu \otimes \nabla G_L
  + \partial_\nu \left ( -\Delta G_{BH} \II
  + \nabla \otimes \nabla G_{BH} \right)
  + \nabla \otimes \left ( -\Delta G_{BH} \bnu
  + \partial_{\nu} \nabla G_{BH} \right) \; .
\end{equation}
%\begin{align}
% -\Delta G_{BH} \bnu
% + \partial_{\nu} \nabla G_{BH} &=
% ( (-\partial_{x_1x_1}-\partial_{x_2x_2}) \nu_1 G_{BH} +
% (\partial_{x_1x_1}\nu_1 + \partial_{x_2x_1} \nu_2 )G_{BH},
% (-\partial_{x_1x_1}-\partial_{x_2x_2}) \nu_2 G_{BH} +
% (\partial_{x_1x_2}\nu_1 + \partial_{x_2x_2} \nu_2 )G_{BH}) \\
% &= (-\partial_{x_2} \partial_\tau G_{BH},
% \partial_{x_1} \partial_\tau G_{BH})
%\end{align}
%\begin{equation}
%  \begin{pmatrix}
%    -\partial_{x_2x_2} & \partial_{x_1x_2} \\
%    \partial_{x_1x_2} & -\partial_{x_1x_1}
%  \end{pmatrix} = -\nabla^\bot \otimes \nabla^\bot
%\end{equation}
Let $\btau = \bnu^\bot$. Then

\begin{equation}
  \TT_{\cdot,\cdot,k} \nu_k = -\bnu \otimes \nabla G_L
  - \nabla^\bot \otimes \nabla^\bot \partial_\nu G_{BH}
  +\nabla \otimes \nabla^\bot \partial_\tau G_{BH} \; .
\end{equation}

\subsection{Layer potentials}

We now use the Stokeslet and stresslet definitions
of the previous section to define the standard single
and double layer potentials of the Stokes eigenvalue
problem. The single layer potential is defined to
be

\begin{equation}
  \SS \bmu (\xx) = \int_\Gamma \GG (\xx,\yy) \bmu(\yy)
  \, dS(\yy) \; .
\end{equation}
The double layer potential is defined to be

\begin{equation}
  \DD \bmu (\xx) = \int_\Gamma \left ( \TT_{\cdot,\cdot,k}\nu_k(\yy)
  \right )^\intercal \bmu(\yy) \, dS(\yy) \; ,
\end{equation}
where $\bnu$ denotes the outward unit normal to the boundary.
If we write $\bmu = \bnu \mu_\nu + \btau \mu_\tau$,
where $\btau = \bmu^\bot$ is the positively oriented unit
tangent to the curve, then we have

\begin{align}
  \left ( \TT_{\cdot,\cdot,k}\nu_k(\yy) \right )^\intercal
  \bmu(\yy) &= \left ( - \nabla G_L(\xx,\yy) + 2 \nabla^\bot
  \partial_{\nu\tau} G_{BH}(\xx,\yy) \right ) \mu_\nu(\yy) \nonumber \\
  & \qquad+
  \nabla^\bot \left (\partial_{\tau\tau}-\partial_{\nu\nu} \right )
  G_{BH}(\xx,\yy) \mu_\tau(\yy) \; .
\end{align}


\subsection{The Farkas integral representation} 

As mentioned in the introduction, there are existing 
integral representations for the solution of the Dirichlet problem for the Helmholtz-biharmonic equation
\begin{align}
 \Delta (\Delta + k^2) w &= 0 \mbox{ in } D \; , \label{eq:biharmD1}\\
 w &= f \mbox{ on }
 \Gamma \label{eq:biharmD2} \; , \\ 
 \frac{\partial w}{\partial n} &= g \mbox{ on } 
 \Gamma \, , \label{eq:biharmD3}
\end{align}
In \cite{Farkas89}, 
the solution is given as the sum of two layer potentials, i.e.
where $\sigma_1$ and $\sigma_2$ are unknown densities.

The integral kernels, $K_1^F$ and $K_2^F$ are based on
derivatives of the Green's function for the biharmonic equation.

\section{Integral equation derivation} \label{sec:anapp}

%\subsection{Adapting the completed double layer representation
%to solve the Dirichlet problem}
We would like to adapt the completed double layer representation
for solutions of Stokes flow \cref{eq:biharmV1,eq:biharmV2,eq:biharmV3} to solve 
the clamped plate problem \cref{eq:biharmD1,eq:biharmD2,eq:biharmD3}.
Let $f$ and $g$ be the boundary data as in \cref{eq:biharmD1,eq:biharmD2,eq:biharmD3}. 
By computing tangential derivatives of $f$ on each
boundary component, we get the following related Helmholtz-Stokes problem:
\begin{align}
 \Delta (\Delta + \lambda^2) \tilde{w}  = 0 &\quad \bx \in D \, ,\nonumber\\
 \frac{\partial \tilde{w}}{\partial \tau} = 
 \frac{\partial f}{\partial \tau} 
&\quad \bx \in \Gamma \label{eq:biharm5} \, ,\\ 
 \frac{\partial \tilde{w}}{\partial n} = g &\quad \bx 
\in \Gamma 
\, . 
\nonumber
\end{align}
There are two main issues to be addressed in using the completed
double layer representation in this context. First,
the representation is designed for 
Helmholtz-Stokes flow, in which the quantities of interest are derivatives
of the potential $\tilde{w}$ and not $\tilde{w}$ itself; the
representation for $\tilde{w}$ may not be single-valued. 
We will establish that, in the context of \cref{eq:biharm5},
the stream function is necessarily single-valued. We also discuss
some numerical issues related to evaluating the stream function.
The second issue to address
is that the solution $\tilde{w}$ only satisfies the original boundary 
condition for the value of $\tilde{w}$ up to a constant on 
each boundary component.
In fact, for multiply connected domains, the completed double layer 
representation is incomplete for the Dirichlet
problem \cref{eq:biharmD1,eq:biharmD2,eq:biharmD3}. We present a remedy for this 
issue and provide some physical intuition.
%
%\subsection{Single-valued stream functions}
%To solve the Dirichlet problem \cref{eq:biharmD1,eq:biharmD2,eq:biharmD3},
%it is necessary to compute
%a single-valued biharmonic potential.
%In the case of a multiply connected domain,
%there is no guarantee that a single-valued stream 
%function exists for a given velocity field.
%
%Consider the following example.
%Let $(r,\theta)$ denote standard polar coordinates. It is
%easy to verify that the velocity field 
%$\bu=\frac{1}{r}\hat{r}$ 
%solves the equations of Stokes flow in an annulus centered at the 
%origin. A stream function for this flow is $w=\theta$, which 
%is not single-valued; indeed, there are no single-valued
%stream functions which generate this flow. 
%
%Let $D$ be a multiply connected domain with boundary 
%$\Gamma = \cup_{i=0}^N \Gamma_i$, as in the previous section.
%We note that the gradient of a stream function is determined 
%by the velocity field, i.e. 
%
%\begin{equation}
%\nabla w = -\bu^\perp := \begin{pmatrix} 
%  - u_2 \\ u_1 
%\end{pmatrix} \; .
%\end{equation}
%Therefore, a velocity field has single-valued stream
%functions if and only if $\bu^\perp$ is conservative.
%Using standard results from multivariable calculus, 
%we can  characterize such flows.
%
%\begin{proposition} 
%Suppose that $\bu$ is a divergence-free velocity 
%field which is $C^1$ on $D$ and continuous on 
%$D\cup \Gamma$. The field $\bu^\perp$ 
%is conservative if and only if 
%
%\begin{equation}
% \int_{\Gamma_i} \bu\cdot \bn \, dS 
%= 0 \quad i=0,1,\ldots N \, .
%\label{eq:NecSuffStreamFuncExistence}
%\end{equation}
%
%\end{proposition}
%
%The equalities \cref{eq:NecSuffStreamFuncExistence}
%constitute $N$ linearly independent constraints on the 
%boundary data because the divergence-free condition 
%\cref{eq:MassConservation} implies that
%$\int_{\Gamma} \bu \cdot \bn \, dS = 0$. 
%It turns out that 
%these conditions are satisfied when the Dirichlet 
%problem is recast as a Stokes flow \cref{eq:biharm5}, as 
%it is easily verified that
%\begin{equation}
% \int_{\Gamma_i} \bu\cdot \bn \, dS 
%= \int_{\Gamma_i} \frac{\partial f}{\partial \tau} \, dS =0 \, .
%\end{equation}
%Thus, any stream function $\tilde{w}$ obtained for the Stokes
%flow \cref{eq:biharm5} is necessarily single-valued.
%% NOT GOING THIS WAY ANYMORE
%%It can be shown using the Helmholtz Hodge decomposition that the 
%%conditions \cref{eq:NecSuffStreamFuncExistence} are sufficient for 
%%the existence of a single-valued stream function. 
%
%\subsection{Evaluating the stream function} \label{subsec:stream}
%
%Given compatible boundary data for the velocity field $\bu$, 
%the completed double layer representation for Stokes flow 
%\cref{eq:IntRepStokes} guarantees the existence of
%a solution density $\boldsymbol\mu$ 
%and a corresponding stream function $\tilde{w}$. The Goursat 
%function formula 
%for $\tilde{w}$, see \cref{sec:stokeslayer},
%is necessarily single-valued, as explained above,
%but it is not immediately obvious from the formula 
%that this should be true.
%
%The difficulty in the representation of $\tilde{w}$ 
%comes from the part
%of the stream function corresponding to the double layer potential
% \cref{eq:wDL}. The 
%second term in the expression for the double layer potential is
%\begin{equation}
% v_1(z) = \mbox{Re}\left[\frac{1}{4\pi i}\int_{\Gamma}\left(
%\overline{\rho\left(\xi\right)}d\xi+\rho\left(\xi\right)
%\overline{d\xi}\right)\log\left(\xi-z\right)\right] \, .
%\end{equation}
%To compute this term, in a na\"{i}ve numerical implementation,
%the question of which is the appropriate branch of the 
%logarithm to use would arise at many steps.
%To avoid this complication, it is possible instead to compute
%$v_1$, up to a constant, as the harmonic conjugate of the
%function 
%\begin{equation}
% v_2 = \frac{1}{4\pi}\int_{\Gamma}\left(\overline{\rho\left(\xi\right)}
%d\xi+\rho\left(\xi\right)\overline{d\xi}\right)\log
%\left(\left|\xi-z\right|\right) \, . \label{eq:harmconjg}
%\end{equation}
%
%We will use this approach to evalute $v_1$ numerically.
%As a result of the Cauchy-Riemann equations, the 
%harmonic conjugate of $v_2$, satisfies the following
%Neumann problem for the Laplace equation:
%\begin{align}
% \Delta v_1 &= 0 &\quad x\in D \, ,\\
% \frac{\partial v_1}{\partial n} &= -\frac{\partial v_2}{\partial \tau} 
%&\quad x\in\Gamma \, .
%\end{align}
%It is possible then to use standard integral equation
%methods to compute $v_1$. 
%
%Let $v_1 = S^L_\Gamma \sigma$, where $S^L_\Gamma \sigma$ 
%is the single layer potential for Laplace's equation, given by
%\begin{equation}
%S^L_\Gamma\sigma (\bx) = -\frac{1}{2\pi} \int_{\Gamma} \log 
%\left|\bx - \by \right|\sigma(\by)\, dS \left(\by \right) \, ,
%\end{equation}
%where $\sigma\in \mathcal{X}= C^{0,\alpha}\left(\Gamma\right)$, for 
%some $\alpha \in (0,1)$, is an unknown density 
%(see \cite{kress1999linear, guenther1988partial}).
%Imposing the Neumann boundary conditions results in the 
%following boundary integral equation for $\sigma$:
%\begin{align}
%-\frac{\partial v_2}{\partial \tau} (\bx) &= \frac{1}{2} \sigma \left(\bx\right) - \frac{1}{2\pi}\oint_{\Gamma} 
%\frac{\partial}{\partial n_x}  
%\log \left| \bx - \by \right |\sigma(\by)\, dS \left(\by \right) 
%\, , \\
%-\frac{\partial v_2}{\partial \tau} &= \left( \frac{1}{2}I_{\mathcal{X}} + K^L_\Gamma  \right) \sigma 
%\, ,
%\label{eq:BlockSystemRow2P1tmp}
%\end{align}
%where the operator $K^L_\Gamma$ is compact, so that the integral equation
%is second kind.
%For a derivation of this result, see \cite{kress1999linear}.
%
%It is well known that the operator $\frac{1}{2}I_{\mathcal{X}} + K^L_\Gamma$ 
%has a one dimensional null space. Thus,
%we choose to solve the above integral equation subject to 
%the constraint $\int_{\Gamma} \sigma \, dS = 0$. 
%Furthermore, it is known that solving the Neumann problem subject 
%to the above constraint is equivalent
%to solving
%\begin{align}
%\left( \frac{1}{2}I_{\mathcal{X}} + K^L_\Gamma  + W_\Gamma \right) \sigma 
%= -\frac{\partial v_2}{\partial \tau}
%\label{eq:BlockSystemRow2P1} 
%\end{align}
%where $W_\Gamma\sigma = \int_{\Gamma} \sigma \, dS$. 
%To prove this, we need 
%the following property of 
%the Green's function for Laplace's equation.
%\begin{align}
%-\frac{1}{2\pi}\oint_{\gamma} \frac{\partial}{\partial n_{\by}}  
%\log \left|\bx - \by\right|\, dS \left(\by \right) 
%&=-\frac{1}{2}\quad\bx\in\gamma \, , \label{eq:DLP2}
%\end{align}
%Using the above property, it follows that
%\begin{align}
% \int_{\Gamma} K^L_\Gamma \sigma \, dS(\x) 
%= -\frac{1}{2}\int_{\Gamma} \sigma \left(\bx\right) \, dS(\x) 
%\label{eq:KLint}
%\end{align}
%Integrating equation \cref{eq:BlockSystemRow2P1} and using 
%equation \cref{eq:KLint}, we get
%\begin{align}
%\int_{\Gamma} \left( \frac{1}{2}I_{\mathcal{X}} + 
%K^L_\Gamma  + L \right) \sigma \, dS(\x) 
%&= -\int_{\Gamma} \frac{\partial v_2}{\partial \tau} \, dS(\x) \\
%\left|\Gamma\right| L\sigma  &= 0 
%\end{align}
%which proves the result.

\subsection{Making the representation complete}
As mentioned above, the solution $\tilde{w}$ of 
the auxiliary Helmholtz-Stokes problem \cref{eq:biharm5}
only satisfies the boundary conditions of the 
original Dirichlet problem \cref{eq:biharmD1,eq:biharmD2,eq:biharmD3}
up to a constant on each boundary component. For a simply
connected domain, this constant can be recovered from the
fact that adding an arbitrary constant to a stream function
does not change the velocity field. 
Thus, in simply connected domains, there is an equivalence in the 
solutions of \cref{eq:biharm5} and \cref{eq:biharmD1,eq:biharmD2,eq:biharmD3}.

To analyze the case of a multiply connected domain,
we first consider
radially symmetric solutions on an annulus centered at the
origin. Let $w\left(r\right)$ be a 
radially symmetric biharmonic-Stokes
potential. Then $w(r)$ solves the ordinary differential
equation (ODE)
\begin{equation}
\Delta_{r} (\Delta_{r} + k^2)w = 0 \, ,
\end{equation}
where $\Delta_{r}$ is the radial component of the Laplacian in polar
coordinates given by
$$
\Delta_{r} = \frac{1}{r}\frac{d}{dr} r \frac{d}{dr} \, .
$$
Four linearly independent solutions of this ODE are $1$, $\log{(r)}$, 
$J_{0}(kr)$, and $H_{0}(kr)$, 
where $J_{0}$ is the order zero Bessel function of the first kind, and $H_{0}$ is the order zero Hankel function of the first kind.
For each solution, we can compute the associated velocity field 
$\bu = \nabla^{\perp} w$. By construction, $\bu$ satisfies 
the continuity condition \cref{eq:MassConservation}. 
For the momentum equation \cref{eq:StokesFlowEq} to be 
satisfied, we need that $\Delta\bu + k^2 \bu$ is a conservative
vector field, which 
is equivalent to the condition that 
$\int_{\gamma} (\Delta\bu + k^2 \bu) \cdot d\bl = 0$ 
for any closed 
loop $\gamma$ in the annulus. For $\tilde{w}=1$, $J_{0}(kr)$, and $H_{0}(kr)$,
the corresponding velocity field satisfies
$\Delta \bu + k^2 \bu=0$ so that $\Delta \bu$
is trivially a conservative vector field. 
On the other hand, for $\tilde{w}=\log{(r)}$, the corresponding velocity field
satisfies $\Delta \bu + k^2 \bu=\frac{k^2}{r} \hat{\theta}$.
By considering a curve $\gamma$ encircling the origin,
we see that $\Delta \bu + k^2 \bu$ is not a conservative vector
field and that any pressure for the velocity field 
associated with $\log\left(r\right)$ is not single-valued.
%
%The above analysis
%can be extended to show that any solution of the biharmonic equation 
%of the form $\log{|r-r_{j}|}$ where $r_{j}$ is located in one of the holes of a multiply connected
%domain.
%cannot be represented as a Stokes velocity field. 
%In simply connected 
%domains, since there are no holes, corresponding to each solution of the modified biharmonic equation with clamped plate
%boundary conditions, there exists a solution to . For multiply connected domains with genus $N$, 
%the set of stream functions for Stokes velocity fields
%misses an $N$ dimensional space of solutions, corresponding 
%to biharmonic charges located in the holes of the domain.
%Following this reasoning, we obtain a complete representation
%for biharmonic potentials on multiply connected domains
%by adding $N$ charges, one per each hole of the domain,
%to the representation for $w$. The details of this approach, and
%the proof that it is sufficient, is in the next section.




\bibliographystyle{plain}
\bibliography{refs}

\end{document}
