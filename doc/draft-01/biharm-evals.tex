\documentclass[preprint,12pt]{article} 
\usepackage{b}
\usepackage[a4paper]{geometry}
\geometry{top=1.0in, bottom=1.0in, left=1.0in, right=1.0in}
\usepackage[small,it]{caption}
\usepackage{empheq}

\usepackage{subcaption}

\usepackage{cleveref}
\crefname{equation}{}{}
\Crefname{equation}{}{}
\crefname{eqnarray}{}{}
\Crefname{eqnarray}{}{}

%%%%%%%%%%%%%%%%%%%%%%%%%%%%%%%%%%%%%%%%%%%%%%%%%%%%%%%%%%%%


\def\xx{{\bf x}}
\def\yy{{\bf y}}
\def\ss{{\bf s}}
\def\tt{{\bf t}}
\def\uu{{\bf u}}
\def\ff{{\bf f}}
\def\GG{{\bf G}}
\def\II{{\bf I}}
\def\TT{{\bf T}}
\def\bsigma{{\boldsymbol \sigma}}
\def\bnu{{\boldsymbol \nu}}
\def\btau{{\boldsymbol \tau}}
\def\bigo{\mathcal{O}}
\def\littleo{o}

\newtheorem{remark}{Remark}
\newtheorem{proposition}{Proposition}


%%%%%%%%%%%%%%%%%%%%%%%%%%%%%%%%%%%%%%%%%%%%%%%%%%%%%%%%%%%%%

\title{An integral equation method for the biharmonic
Dirichlet eigenvalue problem in two dimensions}
\author{Askham and Rachh}
\date{\today}

\begin{document}

\maketitle

%%%%%%%%%%%%%%%%%%%%%%%%%%%%%%%%%%%%%%%%%%%%%%%%%%%%%%%%%%%%

\section{Introduction}

Results of some quick googling:

Finite element methods.
\begin{itemize}
\item Stokes eigenvalue problem: nonconforming finite
  element methods \cite{jia2009approximation}.
\end{itemize}

Integral equation methods.
\begin{itemize}
\item Book on elastodynamic eigenvalue problems
  \cite{kitahara2014boundary}
\end{itemize}
\section{Mathematical Preliminaries}

\subsection{Notation}

Bold symbols are vectors or tensors.
$\xx = (x_1,x_2)^\intercal$ then $\xx^\bot = (-x_2,x_1)^\intercal$.
Similarly, $\nabla^\bot = (-\partial_{x_2},\partial_{x_1})^\intercal$

\subsection{Green's functions}

$\Delta$
\begin{equation}
  G_L(\xx,\yy) = -\frac{1}{2\pi} \log |\xx-\yy|
\end{equation}

$\Delta + k^2$
\begin{equation}
  G_H(\xx,\yy) = \frac{i}{4} H_0^{(1)}(k|\xx-\yy|)
\end{equation}

$\Delta(\Delta + k^2)$
\begin{equation}
  G_{BH}(\xx,\yy) = -\frac{1}{k^2}
  \left (\frac{1}{2\pi} \log |\xx-\yy| +
  \frac{i}{4} H_0^{(1)}(k|\xx-\yy|) \right )
\end{equation}

\section{Integral Equation Formulation}

\subsection{The Helmholtz Stokeslet formulation}

In this section, we review the derivation of the
modified Stokeslet, but for the imaginary parameter
regime --- which we will call the Helmholtz Stokeslet ---
which is the appropriate setting for the Stokes
eigenvalue problem.
The original modified Stokeslet is more well known,
and is of particular interest for its application to
numerical simulations of the unsteady Stokes equations
\cite{pozrikidis1992boundary,biros2002embedded}.

To begin, we restate the Stokes eigenvalue
problem as
\begin{align}
  \nabla p - \Delta \uu - k^2 \uu &= 0 \; , \label{eq:helmstokes}\\
  \nabla \cdot \uu &= 0 \; . \label{eq:helmstokescty}
\end{align}
Consider the solution of this problem where a
$\delta$ centered at $\yy$ with strength $\ff$
has been added to the right-hand side, i.e.

\begin{align}
  \nabla p - \Delta \uu - k^2 \uu &= \delta_\yy \ff \; ,
  \label{eq:helmstokes_charge} \\
  \nabla \cdot \uu &= 0 \; .
\end{align}
Note that

\begin{equation}
  \Delta G_L(\xx,\yy) = \delta_\yy(\xx) \; .
\end{equation}
If we substitute this expression into
\eqref{eq:helmstokes_charge} and take the divergence,
we obtain

\begin{equation}
  p = \nabla G_L(\xx,\yy) \cdot \ff \; .
\end{equation}
We then have, formally,

\begin{align}
  \uu &= - (\Delta + k^2)^{-1} ( \Delta G_L \ff
  - \nabla (\nabla G_L \cdot \ff ) ) \\
  &= \left ( -\Delta + \nabla \otimes \nabla \right )
  G_{BH} \ff \; .
\end{align}
The tensor

\begin{equation}
  \GG = - \II \Delta G_{BH} + \nabla \otimes \nabla G_{BH}
\end{equation}
is then the equivalent of a Stokeslet
\cite{pozrikidis1992boundary} for the problem
\eqref{eq:helmstokes}.

A related object is the stresslet, which is defined
in terms of the stress tensor of a velocity, pressure
pair induced by a Stokeslet. For these tensors, we find
that it is more convenient to express them in index notation
with the Einstein index summing convention.
Recall that the stress tensor $\bsigma$ is defined as 

\begin{equation}
  \sigma_{ij} = -p \delta_{ij} + \left ( \partial_{x_j}u_i
  +\partial_{x_i} u_j \right ) \; ,
\end{equation}
where $\delta_{ij}$ is the standard Kronecker delta notation.
The stresslet $\TT$ is defined to be

\begin{align}
  T_{ijk} &= - \partial_{x_j} G_L \delta_{ik}
  + \partial_{x_k} \left ( -\Delta G_{BH} \delta_{ij} +
  \partial_{x_i} \left(\partial_{x_j} G_{BH} \right) \right)
  \nonumber \\
  & \qquad+ \partial_{x_i} \left ( -\Delta G_{BH} \delta_{kj} +
  \partial_{x_k} \left(\partial_{x_j} G_{BH} \right) \right)
  \; .
\end{align}
Let $u_i = G_{ij} f_j$ and $p = \partial_{x_i} G_L f_i$ be a
solution of the Stokes equations induced by a Stokeslet.
Then the corresponding stress tensor is given by
$\sigma_{ik} = T_{ijk} f_j$.

For the layer potentials of the next section, the following
formulas are useful. Let $\bnu$ be a given vector. When
summing over the third index, we obtain

\begin{equation}
  \TT_{\cdot,\cdot,k} \nu_k = -\bnu \otimes \nabla G_L
  + \partial_\nu \left ( -\Delta G_{BH} \II
  + \nabla \otimes \nabla G_{BH} \right)
  + \nabla \otimes \left ( -\Delta G_{BH} \bnu
  + \partial_{\nu} \nabla G_{BH} \right) \; .
\end{equation}
%\begin{align}
% -\Delta G_{BH} \bnu
% + \partial_{\nu} \nabla G_{BH} &=
% ( (-\partial_{x_1x_1}-\partial_{x_2x_2}) \nu_1 G_{BH} +
% (\partial_{x_1x_1}\nu_1 + \partial_{x_2x_1} \nu_2 )G_{BH},
% (-\partial_{x_1x_1}-\partial_{x_2x_2}) \nu_2 G_{BH} +
% (\partial_{x_1x_2}\nu_1 + \partial_{x_2x_2} \nu_2 )G_{BH}) \\
% &= (-\partial_{x_2} \partial_\tau G_{BH},
% \partial_{x_1} \partial_\tau G_{BH})
%\end{align}
%\begin{equation}
%  \begin{pmatrix}
%    -\partial_{x_2x_2} & \partial_{x_1x_2} \\
%    \partial_{x_1x_2} & -\partial_{x_1x_1}
%  \end{pmatrix} = -\nabla^\bot \otimes \nabla^\bot
%\end{equation}
Let $\btau = \bnu^\bot$. Then

\begin{equation}
  \TT_{\cdot,\cdot,k} \nu_k = -\bnu \otimes \nabla G_L
  - \nabla^\bot \otimes \nabla^\bot \partial_\nu G_{BH}
  +\nabla \otimes \nabla^\bot \partial_\tau G_{BH} \; .
\end{equation}

\subsection{Layer potentials}


\bibliographystyle{plain}
\bibliography{refs}

\end{document}
