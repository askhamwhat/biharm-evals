
\appendix

\section{Proof of the invertibility of the Farkas representation}
In this appendix, we prove~\cref{lem:farkinv} which is restated here
as~\cref{lem:farkinv_app}.

In the following lemma, we present an integral indentity involving the
Green's function of the Helmholtz-biharmonic equation.
\begin{lem}
Suppose that $\Gbh$ is the Green's function of the Helmholtz-biharmonic equation 
with Helmholtz parameter $k$. Suppose that $v:\R^{2} \to \R$ is a compactly supported smooth
function. Then
\begin{equation}
\int_{\R^{2}} \Gbh(\xx,\yy) \Delta (\Delta + k^2) v(\yy) d\yy = v(\xx)
\end{equation}
\end{lem}
\begin{proof}
Suppose that $r>0$. 
Let $B_{r}(\xx)$ denote the ball of radius $r$ centered at the origin
and let $\partial B_{r}(\xx)$ denote its boundary.
Then applying Green's second identity twice, we get 
\begin{align}
\int_{\R^{2} \setminus B_{r}(\xx)} \Gbh(\xx,\yy) \Delta (\Delta +k^{2}) v(\yy)
&=
\int_{\R^{2} \setminus B_{r}(\xx)} \Delta \Gbh (\xx,\yy) (\Delta + k^2) 
v(\yy) + \\
& \hspace{5ex}\int_{\partial B_{r}(\xx)} \Gbh(\xx,\yy) (\Delta + k^{2})
\frac{\partial}{\partial n}v(\yy) \, dS - \\
& \hspace{5ex}\int_{\partial B_{r}(\xx)} \frac{\partial }{\partial n} \Gbh(\xx,\yy) 
(\Delta + k^2) v \, dS \\
&= \int_{\R^{2} \setminus B_{r}(\xx)} \Delta (\Delta + k^2)G(\xx,\yy) v(\yy)
d\yy  - \nonumber \\
& \hspace{5ex}\int_{\partial B_{r}(\xx)} (\Delta + k^2) 
\frac{\partial }{\partial n} \Gbh(\xx,\yy) v \, dS + \nonumber\\
& \hspace{5ex}\int_{\partial B_{r}(\xx)} (\Delta + k^2) 
\Gbh(\xx,\yy) \frac{\partial}{\partial n}v \, dS - \nonumber \\
& \hspace{5ex}\int_{\partial B_{r}(\xx)}  
\frac{\partial }{\partial n} \Gbh(\xx,\yy) \Delta v\, dS + \\
& \hspace{5ex}
\int_{\partial B_{r}(\xx)} \Gbh(\xx,\yy) \Delta \frac{\partial v}{\partial n}
\, dS \nonumber
\end{align}
For all $\yy \in \partial B_{r}(x)$, there exists a constant $C$ such that
$\Gbh$ satisfies the following equations
\begin{align}
(\Delta + k^2) \frac{\partial }{\partial n} G(\xx,\yy) &= 
-\frac{1}{2\pi r} \label{eq:estapp1}\\
\left|(\Delta + k^2) G(\xx,\yy) \right| &\leq C \log{(r)} \label{eq:estapp2}\\
\left|\frac{\partial }{\partial n} G(\xx,\yy) \right| &\leq C r\log{(r)} 
\label{eq:estapp3}\\
\left| G(\xx,\yy) \right| &\leq C r^{2} \log{(r)} \label{eq:estapp4}\, .
\end{align}
Furthermore, $\Gbh$ satisfies $\Delta (\Delta + k^2) \Gbh(\xx,\yy) = 0$
for all $\yy \in \R^{2} \setminus B_{r}(\xx)$.
Using the estimates in~\cref{eq:estapp1,eq:estapp2,eq:estapp3,eq:estapp4}, 
and the smoothness of $v$, we get
\begin{equation}
\int_{\R^{2} \setminus B_{r}(\xx)} \Gbh(\xx,\yy) \Delta (\Delta + k^2) v(\yy) 
d \yy = \frac{1}{2\pi r} \int_{\partial B_{r}(\xx)} v(\yy) dS + 
O(r \log(r)) \, \label{eq:estapp5}.
\end{equation}
The result then follows by taking the limit $r\to 0$ in~\cref{eq:estapp5}. 
\end{proof}
\label{sec:farkproof}

\section{Proof of invertibility of new representation}
The proof proceeds as follows:
\begin{itemize}
\item Show uniqueness of solutions to the impedence, velocity, and surface
traction boundary value problem in exterior domains for the Stokes-Helmholtz
PDEs. In order to prove this, we need the following lemmas:
\begin{itemize}
\item Deriving Green's theorem for exterior domains
\item Rellich's lemma
\item Uniqueness proof
\end{itemize}
\item The proof for the clamped plate problem proceeds in the manner
analogous to the Dirichlet biharmonic paper.
\end{itemize}

With this in mind, we first discuss the PDE theory for the impedence, 
velocity and surface traction boundary value problems on unbounded
domains for the Stokes-Helmholtz equation.
We focus on the proof for the velocity boundary value problem. 
The proofs for the surface traction and impedence problems is similar.

Suppose that $\Omega$ is a bounded simply connected domain and that
$E = \R^{2} \setminus \Omega$ is its exterior.
Let $\Gamma = \partial \Omega$ denote the boundary of $E$.
For a given function $\bh$ on the boundary, 
the exterior velocity boundary value problem is given by:
\begin{align}
\Delta \bu + k^{2} \bu &= \nabla p \quad \bx \in E \, ,\\
\nabla \cdot \bu &= 0 \quad \bx \in E \, ,\\
\bu &= \bh \quad \bx \in \Gamma \, . \\
\end{align}
Let $\bsigma$ denote the stress tensor associated with the velocity
field $\bu$, i.e.
\begin{equation}
\bsigma = -p \II + \be(\bu) \, ,
\end{equation}
where $\be(\bu)$ is the strain tensor given by
\begin{equation}
e_{ij}(\bu) = \left( \partial_{x_j} u_i + \partial_{x_i} u_j \right)
\end{equation}
Let $\bn$ denote the outward normal to the boundary $\Gamma$, then
the surface traction $\ff$ on the boundary $\Gamma$ is given 
by
\begin{equation}
\ff = \sigma \cdot \bn
\end{equation}
Analogous to the Helmholtz equation, we need to impose radiation conditions
at $\infty$.
We propose the following radiation conditions for the Stokes-Helmholtz 
equation

\begin{definition}
Let $(\bu,p)$ satisfy the Stokes-Helmholtz equations in
the exterior of a bounded domain. We say that
the pair $(\bu,p)$ satisfies the radiation condition if
$p$ is bounded, $p \to 0$ as $\xx \to \infty$, and 
\begin{equation}
\lim_{r\to \infty} \sqrt{r} \left| \ff - i k \bu \right| \to 0 \, ,
\label{eq:radcond}
\end{equation}
uniformly in direction.
\end{definition}

\begin{proposition}
Both the Helmholtz-Stokeslet \eqref{eq:stokeslet} and the
kernel of the Helmholtz-Stokes double layer potential
\eqref{eq:stokesdlkernel} satisfy the radiation condition
\eqref{eq:radcond}.
\end{proposition}

\begin{proof}
Without loss of generality, we assume that these
kernels are centered at the origin. Let $r = |\xx|$.
Consider the Stokeslet induced by an arbitrary charge
$k^2 \bmu$. We have

\begin{align}
\bu (\xx) &= k^2 \GG(\xx,0) \bmu \\
&= k^2 \left (-\II \Delta \Gbh(\xx,0)
+ \nabla \otimes \nabla \Gbh(\xx,0)\right ) \bmu \\
&= -k^2 \left (\nabla^\perp \otimes \nabla^\perp \Gbh(\xx,0) \right) \bmu \\
&= \left(\nabla^\perp \otimes \nabla^\perp \left ( \frac{1}{2\pi}
\log r + \frac{i}{4} H_0^{(1)} (kr) \right ) \right)\bmu \; .
\end{align}
Note that derivatives of $\log r$ are $o(1/\sqrt{r})$
and that the pressure associated with the Stokeslet is
$p = \nabla \Glap(\xx) \cdot \bmu$. We then have

\begin{align}
\left | \sigma \cdot \bn - i k \bu \right | &=
\left | p \bn + \partial_n \bu + \nabla (\bu \cdot \bn)
- ik \bu \right | \\
&\leq \left | \partial_n \bu - ik\bu \right | + \left | \nabla(\bu \cdot \bn) \right |
+ o(1/\sqrt{r}) \\
&\leq \frac{1}{4} \left | \partial_n \left(\nabla^\perp \otimes
\nabla^\perp \left (H_0^{(1)} (kr) \right ) \right)\bmu
- i k \left(\nabla^\perp \otimes \nabla^\perp
\left (H_0^{(1)} (kr) \right ) \right)\bmu \right | \\
& \quad + \left | \nabla \left( \partial_\tau \left ( 
\nabla^\perp \left (H_0^{(1)} (kr) \right ) \cdot \bmu  \right )
\right) \right | + o(1/\sqrt{r}) \; .
\end{align}
Because $H_0^{(1)}(kr)$ has the asymptotic expansion 

\begin{equation}
H_0^{(1)}(kr) = \sqrt{\frac{2}{\pi k r}} e^{i(rk-\pi/4)} \left ( 1 + O\left (
\frac{1}{r} \right ) \right ) \;
\end{equation}
as $r\to \infty$, we have

\begin{equation}
\left | \partial_n \left(\nabla^\perp \otimes
\nabla^\perp \left (H_0^{(1)} (kr) \right ) \right)\bmu
- ik \left(\nabla^\perp \otimes \nabla^\perp
\left (H_0^{(1)} (kr) \right ) \right)\bmu \right | =
o ( 1/\sqrt{r} ) \; .
\end{equation}
Finally, because $H_0^{(1)}(kr)$ is radially symmetric,
we have

\begin{equation}
\left | \nabla \left( \partial_\tau \left ( 
\nabla^\perp \left (H_0^{(1)} (kr) \right ) \cdot \bmu  \right )
\right) \right | = 0 \; ,
\end{equation}
so that the Stokeslet satisfies the radiation condition.

The proof for the kernel of the double layer potential
is similar.
\end{proof}



\begin{lem}
\label{lem:rep}
Suppose that $\bu$ satisfies the Stokes-Helmholtz equation in 
an unbounded region $E$ along with the radiaition 
condition~\cref{eq:radcond}. 
Then 
\begin{equation}
\label{eq:repinfest}
\lim_{r\to\infty}
\int_{|\by|=r} \left( |\ff|^2 + |k|^2 |\bu|^2 \right) ds +
2 \text{Im}(k) \int_{E \cap B_{r}(0)} \left(|k|^2 |\bu|^2 + |e(\bu)|^2 \right)
dV = -2 \text{Im} \left( k \int_{\Gamma} \bu \cdot \overline{\ff} ds  \right)
\end{equation}
\end{lem}

\begin{proof}
Since $\bu$ satisfies the radiation condition, we have that
\begin{equation}
\lim_{r\to\infty} \int_{|\by|=r} | \ff - i k \bu|^2 ds = 
\lim_{r\to\infty} \int_{|\by| =r} \left( |\ff|^2 + |k|^2|\bu|^2 + 2 \text{Im} 
\left( k \bu\cdot \ff \right) ds \label{eq:raddecayproof1}
\right) = 0 \, . 
\end{equation}
Since $\bu$ satisfies the Stokes-Helmholtz equation $E \cap B_{r}(0)$,
using a couple of applications of the divergence theorem, we have that
\begin{equation}
\int_{E\cap B_{r}(0)} |e(\bu)|^2 dV = -\int_{\Gamma} \bu \cdot \overline{\ff} ds
+ \int_{|\by|=r} \bu \cdot \overline{\ff} ds + \overline{k}^2 
\int_{E \cap B_{r}(0)} |\bu|^2 dV \,. \label{eq:raddecayproof2}
\end{equation}
Combining~\cref{eq:raddecayproof1,eq:raddecayproof2}, we get
\begin{equation}
\lim_{r\to\infty} \int_{|\by|=r}\left(|\ff|^2 + |k|^2 |\bu|^2 \right) ds 
+ 2\text{Im}(k)\int_{E \cap B_{r}(0)} \left(|e(\bu)|^2 + |k|^2 |\bu|^2 
\right) dV = -2\text{Im} \int_{\Gamma} k \bu \cdot \overline{\ff} dS \, .
\end{equation}
\end{proof}

In the next lemma, we prove the analogue of Rellich's lemma for the Stokes-
Helmholtz equation. 
\begin{lem}
\label{lem:rellich}
Suppose that $\bu$ satisfies the Stokes-Helmholtz equation in an unbounded
region $E$.  
Suppose further that 
\begin{equation}
\lim_{r \to \infty} \int_{|\by|=r} |\bu|^2 dS = 0 
\, . \label{eq:decayatinf}
\end{equation}
Then each component of $\bu$ is harmonic and has a convergent Laurent
expansion in $E$.
\end{lem}
\begin{proof}
We first note that each component of $\bu= (u_{1},u_{2})$ satisfies the 
Helmholtz-Biharmonic equation in $E$, i.e.
\begin{equation}
\Delta (\Delta + k^2) u_{j} = 0 \quad j=1,2 \,. 
\end{equation}
For $r$ sufficiently large, we can express $u_{j}$ in the Fourier basis as
\begin{equation}
u_{j}(r,\theta) = \sum_{n=-\infty}^{\infty} a_{j,n}(r) e^{i n \theta}  \quad 
j=1,2 \, .
\end{equation}
Using Parseval's identity then
\begin{equation}
\int_{|\by|=r} |\bu|^2 ds = r\sum_{n=-\infty}^{\infty} |a_{1,n}(r)|^2  +
|a_{2,n}(r)|^2 \, .
\end{equation}
Since $\bu$ satisfies~\cref{eq:decayatinf}, we conclude that
\begin{equation}
\lim_{r\to\infty} r|a_{j,n}(r)|^2 = 0 \quad j=1,2 \, , \label{eq:adecay}
\end{equation}
Since $u_{j}$, $j=1,2$ satisfies the Helmholtz-biharmonic equation,
the functions $a_{j,n}$ are linear combinations of 
\begin{equation}
r^{|n|}, r^{-|n|}, H^{1}_{n}(k r), H^{2}_{n}(k r) \, , \quad
n\neq 0 \, ,
\end{equation}
and
\begin{equation}
1, \log{(r)}, H^{1}_{0}(k r), H^{2}_{0}(k r) \quad n=0 \, , 
\end{equation} 
where $H_{n}^{1,2}(\cdot)$ are the Hankel functions of the first and
second kind of order $n$.
Since $a_{j,n}(r)$ satisfy~\cref{eq:adecay}, and using the asymptotic 
expansion of $H_{n}^{1,2}(r)$, we note that the projection of 
$a_{j,n}$ on $r^{|n|}$, and $H_{n}^{j}(k r)$ must be zero. 
Thus, 
\begin{equation}
u_{j}(r,\theta) = \sum_{n=-\infty}^{\infty} \frac{a_{j,n} e^{i n \theta}}{r^{|n|}} 
\, .
\end{equation}
\end{proof}
\begin{remark}
Note that each component of $\bu$ is harmonic and thus $\bu$ then satisfies
\begin{align}
k^2 \bu &= \nabla p  \quad \label{eq:massconsred} \\
\nabla \cdot \bu &= 0 \, .
\end{align}
Taking the inner product with $\nabla^{\perp}$ in~\cref{eq:massconsred}, 
we note that $\bu$ satisfies both $\nabla^{\perp} \cdot \bu = 0$ 
and $\nabla \cdot \bu = 0$.
This remark will be useful for the impedence problem...
\end{remark}
\begin{remark}
Note that in Rellich's lemma $\bu$ need not be a radiating solution. 
All we know of $\bu$ is that it satifies the PDE in $E$.
\end{remark}

\begin{thm}
(Uniqueness Dirichlet)
Suppose that $\text{Im}(k)>0$ and 
that $\bu$ is a radiating solution to the Stokes-Helmholtz
equation in $E$ with $\bu =0$ on the boundary $\Gamma$, then
$u \equiv 0$ in $E$.
\end{thm}

\begin{proof}
Since $\bu = 0$ on $\Gamma$, it follows from~\cref{eq:repinfest} that
\begin{equation}
\lim_{r\to\infty}
\int_{|\by|=r} \left( |\ff|^2 + |k|^2 |\bu|^2 \right) ds +
2 \text{Im}(k) \int_{E \cap B_{r}(0)} \left(|k|^2 |\bu|^2 + |e(\bu)|^2 \right)
dV = 0
\end{equation} 
In particular, this implies that
\begin{equation}
\lim_{r\to\infty} \int_{|\by|=r} |\bu|^2 ds = 0 \, .
\end{equation}
Thus, the conditions for $\bu$ in Rellich's lemma (~\cref{lem:rellich})
are satisfied, and each component of $\bu$ is a harmonic function
with $\bu \to 0$ as $r \to \infty$. Furthermore, since $\bu=0$ on
$\Gamma$, from uniqueness of solutions to the Dirichlet problem
on exterior domains, we conclude that $\bu \equiv 0$ in $E$.
\end{proof}

The proof for the Neumann problem proceeds in a similar manner. 

\begin{lem}
If $\bu = \nabla^{\perp} H_{0}(k r)$ and $p\to 0$ as $r\to \infty$, 
then
\begin{equation}
\lim_{r\to \inf} \sqrt(r) \left(\ff_{1} - ik \bu_{1} \right) = 0 \, .
\end{equation}
\end{lem}

\begin{proof}
If $\bu = \nabla^{\perp} H_{0}(k r) = (\partial_{y} H_{0}(k r), 
-\partial_{x} H_{0}(k r))$, then
\begin{equation}
\nabla  p = (\Delta + k^2) \bu = (\Delta + k^2) \nabla^{\perp} 
H_{0} (k r) = 0 \, .
\end{equation}
Since $p\to 0$ as $r\to \infty$, we conclude that $p=0$.
\begin{align}
\ff_{1} &= 2\partial_{x} u_{1} \cos(\theta) + \left( \partial_{x} u_{2} + 
\partial_{y} u_{1} \right) \sin(\theta) \\
&= 2 \partial_{xy} H_{0}(k r) \cdot \frac{x}{r} +
\left[-\partial_{xx} H_{0}(k r) + \partial_{yy} H_{0}(k r) \right]
\cdot \frac{y}{r} \, .
\end{align}
Based on the mathematica script (radcond1.nb in our mathematica folder), we
get
\begin{equation}
\ff_{1} - i k u_{1} = \frac{ky}{r^2} \left( -kr \left(H_{0}(k r) + 
i H_{1}(k r)\right) +
2 H_{1}(k r)\right) \, .
\end{equation}
The result then follows from the asymptotic properties of Hankel functions since
\begin{equation}
H_{0}(z) = c \frac{e^{iz}}{\sqrt{z}} + O(\frac{1}{z^{3/2}}) \quad 
\text{and} \quad 
H_{1}(z) = -i \cdot c \frac{e^{iz}}{\sqrt{z}} + O(\frac{1}{z^{3/2}}) \,,
\end{equation}
where $c$ is the same constant in both of these expansions.
\end{proof}

The following lemma shows uniqueness for the impedance problem under
the assumption that $\bu$ is complex analytic.

\begin{lem}
Suppose that $\bu$ satisfies the radiation
condition~\cref{eq:radcond}. 
Suppose further that the stress tensor $\ff$ and $\bu$ satisfy
\begin{equation}
\ff + i k \bu = 0 \quad \xx \in \Gamma \, .
\end{equation}
Then $\bu \equiv 0$ for $\xx \in E$.
\end{lem}

\begin{proof}
Since $\bu$ satisfies the radiation condition at $\infty$ and $\ff = -ik u$
on $\Gamma$, it follows from~\cref{eq:repinfest} that
\begin{align}
&
\int_{|\by|=r} \left( |\ff|^2 + |k|^2 |\bu|^2 \right) ds +
2 \text{Im}(k) \int_{E \cap B_{r}(0)} \left(|k|^2 |\bu|^2 + |e(\bu)|^2 \right)
dV +2 \text{Im} \left( k \int_{\Gamma} \bu \cdot \overline{\ff} ds  \right) \\
&= 
\int_{|\by|=r} \left( |\ff|^2 + |k|^2 |\bu|^2 \right) ds +
2 \text{Im}(k) \int_{E \cap B_{r}(0)} \left(|k|^2 |\bu|^2 + |e(\bu)|^2 \right)
dV +2 \left( |k|^{2} \int_{\Gamma} |\bu|^{2} \overline{\ff} ds  \right) = 0 
\, .
\end{align}
Since all the quantities are positive, this implies two things:
\begin{equation}
\int_{\Gamma} |\bu|^{2} = 0 \implies \bu = 0  \quad \xx \in \Gamma \, .
\end{equation}
along with the estimate
\begin{equation}
\lim_{r\to\infty} \int_{|\yy|=r} |\bu|^{2} ds \to 0 \, .
\end{equation}
The last equation implies the conditions for~\cref{lem:rellich}
are satisfied, from which we conclude that the components of $\bu$ 
are harmonic in $E$. 
Thus, the components of $\bu$ satisfy
\begin{equation}
\Delta u_{j}=  0 \quad \xx \in E \,, \quad \text{and} \quad
u_{j} = 0 \quad \xx \in \Gamma \, ,
\end{equation}
for $j=1,2$.
Then $\bu \equiv 0$ follows from the uniqueness of solutions to the 
Dirichlet problem for Laplace's equation.
\end{proof}
