
\appendix

\section{Proof of the invertibility of the Farkas representation}
In this appendix, we prove~\cref{lem:farkinv} which is restated here
as~\cref{lem:farkinv_app}.

In the following lemma, we present an integral indentity involving the
Green's function of the Helmholtz-biharmonic equation.
\begin{lem}
Suppose that $\Gbh$ is the Green's function of the Helmholtz-biharmonic equation 
with Helmholtz parameter $k$. Suppose that $v:\R^{2} \to \R$ is a compactly supported smooth
function. Then
\begin{equation}
\int_{\R^{2}} \Gbh(\xx,\yy) \Delta (\Delta + k^2) v(\yy) d\yy = v(\xx)
\end{equation}
\end{lem}
\begin{proof}
Suppose that $r>0$. 
Let $B_{r}(\xx)$ denote the ball of radius $r$ centered at the origin
and let $\partial B_{r}(\xx)$ denote its boundary.
Then applying Green's second identity twice, we get 
\begin{align}
\int_{\R^{2} \setminus B_{r}(\xx)} \Gbh(\xx,\yy) \Delta (\Delta +k^{2}) v(\yy)
&=
\int_{\R^{2} \setminus B_{r}(\xx)} \Delta \Gbh (\xx,\yy) (\Delta + k^2) 
v(\yy) + \\
& \hspace{5ex}\int_{\partial B_{r}(\xx)} \Gbh(\xx,\yy) (\Delta + k^{2})
\frac{\partial}{\partial n}v(\yy) \, dS - \\
& \hspace{5ex}\int_{\partial B_{r}(\xx)} \frac{\partial }{\partial n} \Gbh(\xx,\yy) 
(\Delta + k^2) v \, dS \\
&= \int_{\R^{2} \setminus B_{r}(\xx)} \Delta (\Delta + k^2)G(\xx,\yy) v(\yy)
d\yy  - \nonumber \\
& \hspace{5ex}\int_{\partial B_{r}(\xx)} (\Delta + k^2) 
\frac{\partial }{\partial n} \Gbh(\xx,\yy) v \, dS + \nonumber\\
& \hspace{5ex}\int_{\partial B_{r}(\xx)} (\Delta + k^2) 
\Gbh(\xx,\yy) \frac{\partial}{\partial n}v \, dS - \nonumber \\
& \hspace{5ex}\int_{\partial B_{r}(\xx)}  
\frac{\partial }{\partial n} \Gbh(\xx,\yy) \Delta v\, dS + \\
& \hspace{5ex}
\int_{\partial B_{r}(\xx)} \Gbh(\xx,\yy) \Delta \frac{\partial v}{\partial n}
\, dS \nonumber
\end{align}
For all $\yy \in \partial B_{r}(x)$, there exists a constant $C$ such that
$\Gbh$ satisfies the following equations
\begin{align}
(\Delta + k^2) \frac{\partial }{\partial n} G(\xx,\yy) &= 
-\frac{1}{2\pi r} \label{eq:estapp1}\\
\left|(\Delta + k^2) G(\xx,\yy) \right| &\leq C \log{(r)} \label{eq:estapp2}\\
\left|\frac{\partial }{\partial n} G(\xx,\yy) \right| &\leq C r\log{(r)} 
\label{eq:estapp3}\\
\left| G(\xx,\yy) \right| &\leq C r^{2} \log{(r)} \label{eq:estapp4}\, .
\end{align}
Furthermore, $\Gbh$ satisfies $\Delta (\Delta + k^2) \Gbh(\xx,\yy) = 0$
for all $\yy \in \R^{2} \setminus B_{r}(\xx)$.
Using the estimates in~\cref{eq:estapp1,eq:estapp2,eq:estapp3,eq:estapp4}, 
and the smoothness of $v$, we get
\begin{equation}
\int_{\R^{2} \setminus B_{r}(\xx)} \Gbh(\xx,\yy) \Delta (\Delta + k^2) v(\yy) 
d \yy = \frac{1}{2\pi r} \int_{\partial B_{r}(\xx)} v(\yy) dS + 
O(r \log(r)) \, \label{eq:estapp5}.
\end{equation}
The result then follows by taking the limit $r\to 0$ in~\cref{eq:estapp5}. 
\end{proof}
\label{sec:farkproof}

\section{Existence and uniqueness results for the Helmholtz-Stokes PDE}

In this section, we develop some existence and uniqueness
results for the Helmholtz-Stokes PDE.
%
The structure of this section follows that used
by Colton and Kress \cite[Ch. 3]{colton1983integral}
for the scalar Helmholtz equation.
%
Let $\Omega$ be a bounded simply connected
domain and $E = \R^{2} \setminus \bar{\Omega}$ denote its
exterior.
%
Let $\Gamma = \partial E$ denote the boundary of $E$ and
$\bnu(\yy)$ denote the exterior normal to the point $\yy$ on
$\Gamma$, i.e. the normal vector pointing out of $E$ into $\Omega$.
%
For a given function $\bh$ defined on $\Gamma$,
the exterior Dirichlet boundary value problem is to
find a pair $(\bu,p)$ which satisfies:
\begin{align}
\Delta \bu + k^{2} \bu &= \nabla p \quad \bx \in E \, ,\label{eq:helmstokes_ext}\\
\nabla \cdot \bu &= 0 \quad \bx \in E \, , \\
\bu &= \bh \quad \bx \in \Gamma \, .
\end{align}
In addition to the boundary condition on
$\Gamma$, we must impose radiation conditions
at $\infty$, analogous to the Helmholtz equation.
%

Let $\bsigma$ denote the stress tensor associated with the velocity
field $\bu$, i.e.
\begin{equation}
\bsigma = -p \II + 2 \be(\bu) \, ,
\end{equation}
where $\be(\bu)$ is the strain tensor given by
\begin{equation}
e_{ij}(\bu) = \frac{1}{2} \left( \partial_{x_j} u_i + \partial_{x_i} u_j \right) \; .
\end{equation}
%
The normal component of the stress,
\begin{equation}
\ff = \bsigma \cdot \bnu \; ,
\end{equation}
gives the surface traction, which is the analogue of
the normal derivative for the scalar Helmholtz equation,
i.e. specifying the surface traction on a boundary is
the Neumann problem for the Helmholtz-Stokes PDE.

%
Let $B_r(0)$ denote the disc of radius $r$ centered
at the origin and $\partial B_r(0)$ its boundary.
%
We propose the following radiation condition.

\begin{definition} \label{def:radcond}
Let $(\bu,p)$ satisfy the Helmholtz-Stokes equations in
the exterior of a bounded domain. We say that
the pair $(\bu,p)$ is {\em radiating} if
$p$ is bounded, $p \to 0$ as $r = |\xx| \to \infty$, and 
\begin{equation}
\lim_{r\to \infty} \sqrt{r} \left| \ff - i k \bu \right| \to 0 \, ,
\label{eq:radcond}
\end{equation}
uniformly in direction where $\ff = \bsigma \cdot \bnu$
with $\bnu = \xx/|\xx|$, i.e. $\ff$ is the surface
traction on $\partial B_r(0)$.     
\end{definition}

\begin{proposition}
The Helmholtz-Stokeslet, as in \eqref{eq:stokeslet}, satisfies
the radiation condition in Definition \ref{def:radcond}.
Assuming that a density $\bmu$ is integrable and satisfies
$\int_\Gamma \bmu\cdot \bnu dS = 0$, where $\bnu$ denotes the
outward normal to the curve $\Gamma$, the Helmholtz-Stokes
double layer potential $\DD\bmu$, as in \eqref{eq:doublelayer},
also satisfies the radiation condition.
\end{proposition}

\begin{proof}
Consider the Stokeslet induced by an arbitrary charge
$k^2 \bpsi$ at the origin. Let $r = |\xx|$,
$\bnu(\xx) = \xx/|\xx|$, and $\btau(\xx) = \bnu(\xx)^\perp$. We have

\begin{align}
\bu (\xx) &= k^2 \GG(\xx,0) \bpsi \\
&= k^2 \left (-\II \Delta \Gbh(\xx,0)
+ \nabla \otimes \nabla \Gbh(\xx,0)\right ) \bpsi \\
&= -k^2 \left (\nabla^\perp \otimes \nabla^\perp \Gbh(\xx,0) \right) \bpsi \\
&= \left(\nabla^\perp \otimes \nabla^\perp \left ( \frac{1}{2\pi}
\log r + \frac{i}{4} H_0^{(1)} (kr) \right ) \right)\bpsi \; .
\end{align}
Note that derivatives of $\log r$ are $\littleo (1/\sqrt{r})$
and that the pressure associated with the Stokeslet is
$p = \nabla \Glap(\xx) \cdot \bpsi$. We then have

\begin{align}
\left | \sigma \cdot \bnu(\xx) - i k \bu \right | &=
\left | p \bnu(\xx) + \partial_{{\nu_x}} \bu + \nabla (\bu \cdot \bnu(\xx))
- ik \bu \right | \\
&\leq \left | \partial_{{\nu_x}} \bu - ik\bu \right | + \left | \nabla(\bu \cdot \bnu(\xx)) \right |
+ \littleo (1/\sqrt{r}) \\
&\leq \frac{1}{4} \left | \partial_{{\nu_x}} \left(\nabla^\perp \otimes
\nabla^\perp \left (H_0^{(1)} (kr) \right ) \right)\bpsi
- i k \left(\nabla^\perp \otimes \nabla^\perp
\left (H_0^{(1)} (kr) \right ) \right)\bpsi \right | \\
& \quad + \left | \nabla \left( \partial_{\tau_x} \left ( 
\nabla^\perp \left (H_0^{(1)} (kr) \right ) \cdot \bpsi  \right )
\right) \right | + \littleo (1/\sqrt{r}) \; .
\end{align}
Because $H_0^{(1)}(kr)$ has the asymptotic expansion 

\begin{equation}
H_0^{(1)}(kr) = \sqrt{\frac{2}{\pi k r}} e^{i(rk-\pi/4)} \left ( 1 + O\left (
\frac{1}{r} \right ) \right ) \;
\end{equation}
as $r\to \infty$, we have

\begin{equation}
\left | \partial_{{\nu_x}} \left(\nabla^\perp \otimes
\nabla^\perp \left (H_0^{(1)} (kr) \right ) \right)\bpsi
- ik \left(\nabla^\perp \otimes \nabla^\perp
\left (H_0^{(1)} (kr) \right ) \right)\bpsi \right | =
o ( 1/\sqrt{r} ) \; .
\end{equation}
Finally, because $H_0^{(1)}(kr)$ is radially symmetric,
we have

\begin{equation}
\left | \nabla \left( \partial_{\tau_x} \left ( 
\nabla^\perp \left (H_0^{(1)} (kr) \right ) \cdot \bpsi  \right )
\right) \right | = 0 \; ,
\end{equation}
so that the Stokeslet satisfies the radiation condition.

For the double layer potential, we only establish the
decay of the pressure $p$; the rest of the terms
in \eqref{eq:radcond} can be bounded using an argument
like that for the Stokeslet above.
Because $p$ is harmonic in the exterior of any disc
containing $\Gamma$, it is sufficient to show that
$|\nabla p| = \bigo (1/r^2)$. Let $\btau$ denote the
positively oriented tangent to the curve $\Gamma$ and
$\mu_\nu(\yy)$ and $\mu_\tau(\yy)$ denote $\bmu(\yy) \cdot \bnu(\yy)$ and
$\bmu(\yy) \cdot \btau(\yy)$, respectively. Substituting
$\DD \bmu$ into \eqref{eq:helmstokes_ext}, we obtain

\begin{align}
\nabla p^\DD(\xx) &= (\Delta + k^2) \DD \bmu(\xx) \\
&= \int_\Gamma \left (-k^2 \nabla \Glap (\xx,\yy) +
2 \nabla^\perp \partial_{\nu\tau} \Glap (\xx,\yy) \right ) \mu_\nu(\yy)
\, dS(\yy) \\
& \qquad + \int_\Gamma \nabla^\perp (\partial_{\tau\tau}-\partial_{\nu\nu})\Glap \mu_\tau(\yy)
\, dS(\yy) \; .
\end{align}
The other terms are higher-order derivatives
of $\Glap$, so it is sufficient to show that the term

\begin{align}
|\nabla p_1(\xx)| &:= \left |-k^2 \nabla \int_\Gamma \Glap(\xx,\yy)
\mu_\nu(\yy) \, dS(\yy) \right |
\end{align}
is $\bigo (1/r^2)$. In the following, let $z = x_1 + i x_2$ be the
point corresponding to $\xx$ in the complex plane and let $R$
be the radius of some disc containing $\Gamma$. If $|\xx| > 2R$,
we can use a standard multipole expansion of $\log(z-(y_1+iy_2))$
and the assumption that $\int_\Gamma \bmu \cdot \bnu = 0$
to obtain
\begin{align}
|\nabla p_1(\xx)| &= \frac{k^2}{2\pi} \left |  \partial_z \int_\Gamma \log(z-(y_1+iy_2))
\mu_\nu(\yy) dS(\yy) \right | \\
&= \frac{k^2}{2\pi} \left |  \partial_z  \left ( \log(z) \int_\Gamma \mu_\nu(\yy) \, dS(\yy)
+ \sum_{l=1}^\infty \frac{1}{z^l} \int_\Gamma \left( y_1+iy_2 \right)^l \mu_\nu(\yy) \, dS(\yy)
\right ) \right | \\
&= \frac{k^2}{2\pi} \left | \sum_{l=1}^\infty \frac{-l}{z^{l+1}}
\int_\Gamma \left( y_1+iy_2 \right)^l \mu_\nu(\yy) \, dS(\yy) \right | \\
&= \bigo (1/r^2) \; .
\end{align}
\end{proof}

\subsection{Uniqueness results}


\begin{lem}
\label{lem:rep}
Suppose that $(\bu,p)$ satisfies the Helmholtz-Stokes equation in 
an unbounded region $E$ along with the radiation 
condition~\cref{eq:radcond}. 
Then 
\begin{equation}
\label{eq:repinfest}
\lim_{r\to\infty}
\int_{|\by|=r} \left( |\ff|^2 + |k|^2 |\bu|^2 \right) dS +
2 \Im(k) \int_{E \cap B_{r}(0)} \left(|k|^2 |\bu|^2 + |2\be(\bu)|^2 \right)
dV = 2 \Im \left( k \int_{\Gamma} \bu \cdot \overline{\ff} dS  \right)
\end{equation}
\end{lem}

\begin{proof}
Since $(\bu,p)$ satisfies the radiation condition, we have that
\begin{equation}
\lim_{r\to\infty} \int_{|\by|=r} | \ff - i k \bu|^2 dS = 
\lim_{r\to\infty} \int_{|\by| =r} \left( |\ff|^2 + |k|^2|\bu|^2 + 2 \Im 
\left( k \bu\cdot \overline{\ff} \right) dS \label{eq:raddecayproof1}
\right) = 0 \, . 
\end{equation}
Since $\bu$ satisfies the Helmholtz-Stokes equation $E \cap B_{r}(0)$,
using a couple of applications of the divergence theorem, we have that
\begin{equation}
\int_{E\cap B_{r}(0)} |2 \be(\bu)|^2 dV =
\int_{\Gamma} \bu \cdot \overline{\ff} dS
+ \int_{|\by|=r} \bu \cdot \overline{\ff} dS + \overline{k}^2 
\int_{E \cap B_{r}(0)} |\bu|^2 dV \,. \label{eq:raddecayproof2}
\end{equation}
Combining~\cref{eq:raddecayproof1,eq:raddecayproof2}, we get
\begin{equation}
\lim_{r\to\infty} \int_{|\by|=r}\left(|\ff|^2 + |k|^2 |\bu|^2 \right) dS 
+ 2 \Im(k)\int_{E \cap B_{r}(0)} \left(|2\be(\bu)|^2 + |k|^2 |\bu|^2 
\right) dV = 2\Im \left ( k \int_{\Gamma} \bu \cdot \overline{\ff} dS \right) \, .
\end{equation}
\end{proof}

In the next lemma, we prove the analogue of Rellich's lemma for the
Helmholtz-Stokes equation. 
\begin{lem}
\label{lem:rellich}
Suppose that $\bu$ satisfies the Helmholtz-Stokes equation in an unbounded
region $E$.  
Suppose further that 
\begin{equation}
\lim_{r \to \infty} \int_{|\by|=r} |\bu|^2 dS = 0 
\, . \label{eq:decayatinf}
\end{equation}
Then each component of $\bu$ is harmonic and has a convergent Laurent
expansion in $E$.
\end{lem}
\begin{proof}
We first note that each component of $\bu= (u_{1},u_{2})$ satisfies the 
Helmholtz-Biharmonic equation in $E$, i.e.
\begin{equation}
\Delta (\Delta + k^2) u_{j} = 0 \quad j=1,2 \,. 
\end{equation}
For $r$ sufficiently large, we can express $u_{j}$ in the Fourier basis as
\begin{equation}
u_{j}(r,\theta) = \sum_{n=-\infty}^{\infty} a_{j,n}(r) e^{i n \theta}  \quad 
j=1,2 \, .
\end{equation}
Using Parseval's identity then
\begin{equation}
\int_{|\by|=r} |\bu|^2 dS = r\sum_{n=-\infty}^{\infty} |a_{1,n}(r)|^2  +
|a_{2,n}(r)|^2 \, .
\end{equation}
Since $\bu$ satisfies~\cref{eq:decayatinf}, we conclude that
\begin{equation}
\lim_{r\to\infty} r|a_{j,n}(r)|^2 = 0 \quad j=1,2 \, , \label{eq:adecay}
\end{equation}
Since $u_{j}$, $j=1,2$ satisfies the Helmholtz-biharmonic equation,
the functions $a_{j,n}$ are linear combinations of 
\begin{equation}
r^{|n|}, r^{-|n|}, H^{1}_{n}(k r), H^{2}_{n}(k r) \, , \quad
n\neq 0 \, ,
\end{equation}
and
\begin{equation}
1, \log{(r)}, H^{1}_{0}(k r), H^{2}_{0}(k r) \quad n=0 \, , 
\end{equation} 
where $H_{n}^{1,2}(\cdot)$ are the Hankel functions of the first and
second kind of order $n$.
Since $a_{j,n}(r)$ satisfy~\cref{eq:adecay}, and using the asymptotic 
expansion of $H_{n}^{1,2}(r)$, we note that the projection of 
$a_{j,n}$ on $r^{|n|}$, and $H_{n}^{j}(k r)$ must be zero. 
Thus, 
\begin{equation}
u_{j}(r,\theta) = \sum_{n=-\infty}^{\infty} \frac{a_{j,n} e^{i n \theta}}{r^{|n|}} 
\, .
\end{equation}
\end{proof}
\begin{remark} \label{rmk:harmu}
Note that each component of $\bu$ is harmonic and thus $\bu$ then satisfies
\begin{align}
k^2 \bu &= \nabla p  \quad \label{eq:massconsred} \\
\nabla \cdot \bu &= 0 \, .
\end{align}
Taking the inner product with $\nabla^{\perp}$ in~\cref{eq:massconsred}, 
we note that $\bu$ satisfies both $\nabla^{\perp} \cdot \bu = 0$ 
and $\nabla \cdot \bu = 0$.
\end{remark}
\begin{remark}
Note that in Rellich's lemma $\bu$ need not be a radiating solution. 
All we know of $\bu$ is that it satifies the PDE in $E$.
\end{remark}

\begin{thrm}[Uniqueness of the Dirichlet Problem]
Suppose that $\Im(k)\geq 0$ and 
that $(\bu,p)$ is a radiating solution to the Helmholtz-Stokes
equation in $E$ with $\bu =0$ on the boundary $\Gamma$, then
$\bu \equiv 0$ in $E$.
\end{thrm}

\begin{proof}
Since $\bu = 0$ on $\Gamma$, it follows from~\cref{eq:repinfest} that
\begin{equation}
\lim_{r\to\infty}
\int_{|\by|=r} \left( |\ff|^2 + |k|^2 |\bu|^2 \right) dS +
2 \Im(k) \int_{E \cap B_{r}(0)} \left(|k|^2 |\bu|^2 + |\be(\bu)|^2 \right)
dV = 0
\end{equation} 
In particular, this implies that
\begin{equation}
\lim_{r\to\infty} \int_{|\by|=r} |\bu|^2 dS = 0 \, .
\end{equation}
Thus, the conditions for $\bu$ in Rellich's lemma (~\cref{lem:rellich})
are satisfied, and each component of $\bu$ is a harmonic function
with $\bu \to 0$ as $r \to \infty$. Furthermore, since $\bu=0$ on
$\Gamma$, from uniqueness of solutions to the
Dirichlet problem for Laplace's equation
on exterior domains, we conclude that $\bu \equiv 0$ in $E$.
\end{proof}

\begin{thrm}[Uniqueness of the Neumann Problem]
Suppose that $\Im(k)\geq 0$ and 
that $(\bu,p)$ is a radiating solution to the Helmholtz-Stokes
equation in $E$ with $\ff = 0$ on the boundary $\Gamma$, then
$\bu \equiv 0$ in $E$.
\end{thrm}

\begin{proof}
Since $\ff = 0$ on $\Gamma$, it follows
from~\cref{eq:repinfest} that
\begin{equation}
\lim_{r\to\infty}
\int_{|\by|=r} \left( |\ff|^2 + |k|^2 |\bu|^2 \right) dS +
2 \Im(k) \int_{E \cap B_{r}(0)} \left(|k|^2 |\bu|^2 + |\be(\bu)|^2 \right)
dV = 0
\end{equation} 
In particular, this implies that
\begin{equation}
\lim_{r\to\infty} \int_{|\by|=r} |\bu|^2 dS = 0 \, .
\end{equation}
Thus, the conditions for $\bu$ in Rellich's lemma (~\cref{lem:rellich})
are satisfied, and each component of $\bu$ is a harmonic function
with $\bu \to 0$ as $r \to \infty$. Furthermore, as observed
in \cref{rmk:harmu}, $k^2 \bu = \nabla p$. Then, the boundary
condition becomes $0 = \ff = -p \bnu + 2 \nabla \partial_\nu p/k^2$.
Because $0 = \btau \cdot \ff = 2\partial_{\tau\nu} p/k^2$,
$\partial_\nu p $ is a constant on $\Gamma$. Using the
fact that $(\bu,p)$ is radiating and the fact that $p$ is harmonic,
the divergence theorem implies that $\partial_\nu p=0$ on $\Gamma$.
Thus, $\ff = -p\bnu$ so that $p = 0$ on $\Gamma$. By the uniqueness
of solutions to the Dirichlet problem for Laplace's equation,
we have that $p=0$ so that $\uu \equiv 0$.

\end{proof}

\begin{thrm}[Uniqueness of the Impedance Problem]
Suppose that $\Im(k) \geq 0$ and that
$(\bu,p)$ is a radiating solution of the
Helmholtz-Stokes equation in $E$ which satisfies
the homogeneous impednace boundary condition
\begin{equation}
\ff - i k \bu = 0 \quad \xx \in \Gamma \, .
\end{equation}
Then $\bu \equiv 0$ for $\xx \in E$.
\end{thrm}

\begin{proof}
Since $\bu$ satisfies the radiation condition at $\infty$ and $\ff = ik \bu$
on $\Gamma$, it follows from~\cref{eq:repinfest} that
\begin{align}
0 &=
\int_{|\by|=r} \left( |\ff|^2 + |k|^2 |\bu|^2 \right) dS +
2 \text{Im}(k) \int_{E \cap B_{r}(0)} \left(|k|^2 |\bu|^2 + |2\be(\bu)|^2 \right)
dV \nonumber \\
& \qquad - 2 \text{Im} \left( k \int_{\Gamma} \bu \cdot \overline{\ff} dS  \right) \\
&= 
\int_{|\by|=r} \left( |\ff|^2 + |k|^2 |\bu|^2 \right) dS +
2 \text{Im}(k) \int_{E \cap B_{r}(0)} \left(|k|^2 |\bu|^2 + |2\be(\bu)|^2 \right)
dV \nonumber \\
& \qquad + 2 \left( |k|^{2} \int_{\Gamma} |\bu|^{2} dS  \right)
\, .
\end{align}
Because all of the quantities in the last expression above are
nonnegative, we have that
\begin{equation}
\int_{\Gamma} |\bu|^{2} = 0 \implies \bu = 0  \quad \xx \in \Gamma \, 
\end{equation}
and the estimate
\begin{equation}
\lim_{r\to\infty} \int_{|\yy|=r} |\bu|^{2} dS \to 0 \, .
\end{equation}
The last equation implies the conditions for~\cref{lem:rellich}
are satisfied, from which we conclude that the components
of $\bu$ are harmonic in $E$. 
Thus, the components of $\bu$ satisfy
\begin{equation}
\Delta u_{j}=  0 \quad \xx \in E \quad \text{and} \quad
u_{j} = 0 \quad \xx \in \Gamma \, ,
\end{equation}
for $j=1,2$.
Then $\bu \equiv 0$ follows from the uniqueness of solutions to the 
Dirichlet problem for Laplace's equation.
\end{proof}

\begin{lem}
Suppose that $\Omega$ is a multiply connected domain with outer boundary
$\Gamma_{0}$ and $n-$ obstacle boundaries $\Gamma_{j}$, $j=1,2,\ldots n$. 
Suppose that $k$ is a Stokes eigenvalue for the region $\Omega$ and 
let $\bu$ be the corresponding eigenfunction. 
Then, the stream function associated with the velocity $\bu$ defined by
$w = \left(\nabla^{\perp} \right)^{-1} \bu$ exists and is single valued. 
Furthermore, $k$ is also a buckling eigenvalue if and only if
\begin{equation}
\int_{\Gamma_{j}} w\, ds = 0 \quad \forall j=1,2,\ldots n \, .
\end{equation}
\end{lem}

\begin{proof}
Since $\bu$ is a Stokes eigenfunction, it satisfies
\begin{equation}
\int_{\Gamma_{j}} \bu \cdot \bn ds = 0 \quad j=1,2\ldots n \, .
\end{equation}
Thus, there exists a single valued stream function associated with the
velocity field $\bu$ such that 
\begin{equation}
\nabla^{\perp} w = \bu \, .
\end{equation}
Furthermore, since $\bu$ satisfies the Helmholtz-Stokes equations, 
$w$ satisfies
\begin{align}
\Delta (\Delta + k^{2} ) w &= 0 \, , \quad \bx \in \Omega \\
\nabla^{\perp} w = \bu &=0 \, ,\quad \bx \in \Gamma \, ,
\end{align}
where $\Gamma = \cup_{j=0}^{n} \Gamma_{j}$ is the boundary of $\Omega$.
In particular $w= c_{j}$ and $dw/dn = 0$ for $\bx \in \Gamma_{j}$ for
some constant $c_{j}$.
Thus $k$ is also a buckling eigenvalue if and only if 
\begin{equation}
c_{j} = \frac{1}{|\Gamma_{j}|}\int_{\Gamma_{j}} w ds =  0 \, .
\end{equation}
\end{proof}

The above result can be restated as a 
a uniqueness result for the 
solutions to the Helmholtz-Stokes equations whenever $k$ is
not a buckling value and the stream function associated with the
velocity satisfies additional integral constraints.
We state the result in the following lemma.
\begin{lem}
\label{lem:conshelmbieig}
Suppose that $k$ is not a buckling eigenvalue. 
Further suppose that $\bu$ satisfies
\begin{align}
\Delta \bu + k^{2} \bu &= \nabla p \quad \bx \in \Omega \\
\nabla \cdot \bu &=0 \quad \bx \in \Omega \\
\bu &=0 \quad \bx \in \Gamma \\
\frac{1}{|\Gamma_{j}|}
\int_{\Gamma_{j}} \left(\nabla^{\perp} \right)^{-1} \bu &= c \quad
j=0,1,2,\ldots n\, ,
\end{align}
where $c$ is a constant independent of the boundary component $\Gamma_{j}$.
Then $\bu \equiv 0$ for all $\bx \in \Omega$.
\end{lem}

\begin{lem}
Suppose that $k$ is not a buckling eigenvalue. 
Let $\mathbf{K}_{\Gamma}$, $\mathbf{B}$, $\mathbf{D}$, 
and $\mathbf{F}$ denote the operators defined in~\cref{eq:eqns}.
Then the block system
\begin{equation}
\begin{bmatrix}
-\frac{1}{2}\mathbf{I}_{\mathcal{X}\times\mathcal{X}} + 
\mathbf{K}_{\Gamma} & \mathbf{B} \\
\mathbf{D} & \mathbf{F} 
\end{bmatrix} \, ,
\label{eq:IntEqBiharm1newapp}
\end{equation}
is an invertible Fredholm operator.
\end{lem}
\begin{proof}
It is simple to
show that the linear system \cref{eq:IntEqBiharm1newapp}
is Fredholm. The block which contains
$-1/2 \mathbf{I}_{\mathcal{X}\times\mathcal{X}}
+ \mathbf{K}_\Gamma$ is Fredholm due to 
\cref{lem:stokehelmsrep}. 
The off-diagonal blocks,
denoted by $\mathbf{B}$ and $\mathbf{D}$, 
are trivially compact because either the domain 
or range of the operator is finite dimensional. 
Finally, $\mathbf{F}$ is 
Fredholm because it is a finite-dimensional linear
operator. 
Therefore, the full system is Fredholm.

Due to the Fredholm alternative,
it is suffices to establish the injectivity of
the system \cref{eq:IntEqBiharm1new} to prove that 
it is invertible.

Suppose that $\bmu,\bc$ satisfy
\begin{equation}
\begin{bmatrix}
-\frac{1}{2}\mathbf{I}_{\mathcal{X}\times\mathcal{X}} + 
\mathbf{K}_{\Gamma} & \mathbf{B} \\
\mathbf{D} & \mathbf{F} 
\end{bmatrix}  
\bbmat
\bmu \\
\bc
\ebmat
=
\bbmat
0 \\ 0
\ebmat \, ,
\label{eq:IntEqBiharm1newapp2}
\end{equation}
then $w$ defined by~\cref{eq:wrep}, satsifies the Helmholtz-Biharmonic
equation~\cref{eq:helmbi,eq:helmbibc1,eq:helmbibc2} 
with boundary conditions $f=g=0$.
Since $k$ is not a buckling eigenfrequency, we conclude that $w\equiv 0$ 
for $\bx \in \Omega$.


Let $\bu = \nabla^{\perp} w$ and let 
$\bu^{\mu} = ik \mathbf{S} [\bmu] + \mathbf{D}[\bmu] = u 
-\mathbf{B} \bc - \mathbf{W}[\bmu]$.
For each $\ell = 1,\ldots,N$, 
let $\tilde{\Gamma}_{\ell} \subset \Omega$ be a curve which satisfies 
$n\left(z_{j}, \tilde{\Gamma}_{\ell}\right) = \delta_{j\ell}$, 
where $n\left(z,\gamma\right)$ represents the winding number 
of the curve $\gamma$ about z. 
Because $\mathbf{u} = \nabla^{\perp} w$ and $w\equiv 0$ in $D$,
we have 
\begin{equation}
\int_{\tilde{\Gamma}_\ell} (\Delta + k^2) \bu \cdot \boldsymbol{\tau} \, dS = 0 .
\end{equation}
We observe that $\bu^\mu$ corresponds
to a Helmholtz-Stokes velocity field in $D$. Let $p$ be 
its associated pressure. Then
\begin{equation}
\int_{\tilde{\Gamma}_k} (\Delta+k^2) \bu^\mu 
\cdot \boldsymbol{\tau}
\, dS = \int_{\tilde{\Gamma}_k} \nabla p \cdot \boldsymbol{\tau} \, dS = 0 \, .
\end{equation}
Further, a simple calculation shows that 
\begin{equation}
\int_{\tilde{\Gamma}_\ell} (\Delta + k^{2}) \nabla^{\perp} c_j 
\log (r_j) \cdot \boldsymbol{\tau} \, dS = 
2 \pi k^2 c_{j} \delta_{j\ell} \, ,
\end{equation}
for $j = 1, \ldots, N$.
Combining these equations, we conclude that
\begin{equation}
0 = \int_{\tilde{\Gamma}_j} (\Delta +k^2) \bu \cdot 
\boldsymbol{\tau} \, dS =
\int_{\tilde{\Gamma}_j} (\Delta +k^2)( \mathbf{u}^\mu + \mathbf{B}\mathbf{c} )
\cdot \boldsymbol{\tau} \, dS =  
2\pi k^2 c_j \, .
\end{equation}
Thus $c_j = 0$ for $j=1,2,\ldots N$.

The system of equations~\cref{eq:IntEqBiharm1newapp2} then reads
\begin{align}
\left(-\frac{1}{2}\mathbf{I}_{\mathcal{X}\times\mathcal{X}} 
+ \mathbf{K}_\Gamma \right) 
\boldsymbol{\mu} &= 0 \, . \label{eq:IntEq1} \\
\mathbf{B} \bmu &= -c_{0} \left(|\Gamma_{0}|, |\Gamma_{2}| \ldots |\Gamma_{n}| 
\,. 
\right)
\end{align}
This implies that $\bu^{\mu}$ defined above satisfies the
Helmholtz-Stokes equation with the integral constraints
\begin{equation}
\frac{1}{|\Gamma_{j}|} 
\int_{\Gamma_{j}} \left(\nabla^{\perp}\right)^{-1} \bu ds = -c_{0} 
\, .
\end{equation}
From~\cref{lem:conshelmbieig}, we conclude that $\bu^{\mu} \equiv 0$ 
for $\bx \in \Omega$.
Using properties of the layer potentials, $\bu^{\mu}$ satisfies that
Helmholtz-Stokes equation in $\R^{2} \setminus \Omega$
along with the radiation condition~\cref{def:radcond}.
Let $\bu^{\mu,+}, \ff^{\mu,+}$ denote the exterior limits, i.e.
\begin{align}
\bu^{\mu,+}(\bx_{0}) &=
\lim_{\substack{\bx \to \bx_{0}\\ \bx \in \R^{2} \setminus \Omega}}
\bu(\bx)\ , , \\ 
\ff^{\mu,+}(\bx_{0}) &=
\lim_{\substack{\bx \to \bx_{0}\\ \bx \in \R^{2} \setminus \Omega}}
\ff(\bx) \, , \\ 
\end{align}
where $\bx_{0} \in \Gamma$.
Using the jump conditions for the layer potentials, we conclude that
\begin{equation}
\ff^{\mu,+}(\bx) = ik \bmu (\bx) \quad \text{and} \quad 
\bu^{\mu,+}(\bx) = \bmu
(\bx) \quad \bx \in \Gamma \,. 
\label{eq:bcaux}
\end{equation}
Thus, in the exterior of the outer-boundary $\Gamma_{0}$, 
$\bu^{\mu,+}$ satisfies the Helmholtz-Stokes equation with the impendance
boundary condition
\begin{equation}
\ff^{\mu,+} -ik \bu^{\mu,+} = 0 \quad \bx \in \Gamma_{0} \, .
\end{equation}
From uniqueness of solutions to the exterior impedance problem, we
conclude that $\bu^{\mu,+}\equiv 0$ for $\bx$ in the exterior of 
$\Gamma_{0}$.
Thus $\bmu(\bx) = 0$ for $\bx \in \Gamma_{0}$.
Let $D_{j}$ denote the interior region of the obstacle boundary
$\Gamma_{j}$, for $j=1,2,\ldots n$. 
Recall that $\bu^{\mu}$ satisfies the Stokes-Helmholtz equation in $D_{j}$.
Using Green's identities for the Stokes-Helmholtz equations, we get 
\begin{equation}
\int_{D_{j}} |e(\bu^{\mu,+})|^{2}  - \overline{k}^{2} |\bu^{\mu,+}|^2) dV = 
\int_{\Gamma_{j}} \bu^{\mu,+} \cdot \overline{\ff}^{\mu,+} dS \, .
\end{equation}
Taking the imaginary part of the above equation and using~\cref{eq:bcaux}, 
we get
\begin{equation}
2 \text{Re}(k) \text{Im}(k) \int_{D_{j}} |\bu^{\mu,+}|^2 dV 
+ \text{Re}(k) \int_{\Gamma_{j}} |\bmu|^2 ds = 0 \, .
\end{equation}
This implies that $\bmu =0$ for all $\bx \in \Gamma_{j}$, $j=1,2,\ldots n$.

Finally, because $\bmu =0$ for all $\bx \in \Gamma$ and 
$c_{j}= 0$, $j=1,2,\ldots n$, 
we get that $w \equiv c_{0}$ for all $\bx \in \Omega$. 
Since $w\equiv 0$, it then follows that $c_{0} = 0$ as well, proving the 
injectivity of the system.

\end{proof}
\subsection{Existence results}

